\documentclass[../../main.tex]{subfiles}
 
\begin{document}
\label{sec:scholarship}

Whittier College faculty classify scholarship using the \textit{Boyer} model \cite{boyer}.  My scholarship primarily falls within two categories: the scholarship of discovery, and the scholarship of application.  I utilize physics education research (PER) in my teaching practices.  When I tried my hand at advising a Whittier Scholars Program major, I found myself performing the scholarship of integration (see Sec. \ref{sec:advising_mentoring}).  In the following sections I reflect upon my scholarly work, and share my vision for the future.  In Sec. \ref{sec:professional}, I describe my research field, giving it proper historical context.  In Sec. \ref{sec:invite}, I share exciting news regarding my research and Whittier College.  In Sec. \ref{sec:neutrino}, I organize my scholarship of discovery into five categories and list progress made by my students and I.  I also draw a connection to my teaching and advising practices in each category, since they are interconnected.  In Sec. \ref{sec:naval_research}, I describe my new venture into the scholarship of application.  This has been a fruitful partnership with the Office of Naval Research (ONR).  Finally, in Sec. \ref{sec:naval_collaboration}, I propose an idea that for a partnership program has been shared with me by my colleagues at the ONR.

%\begin{flushleft}
%\subfile{professional/professional}
%\end{flushleft}

\begin{flushleft}
\subfile{sections/scholarship/professional/professional}
\end{flushleft}

%\begin{flushleft}
%\subfile{neutrino/neutrino}
%\end{flushleft}

\begin{flushleft}
\subfile{sections/scholarship/neutrino/neutrino}
\end{flushleft}

%\begin{flushleft}
%\subfile{naval_research/naval_research}
%\end{flushleft}

\begin{flushleft}
\subfile{sections/scholarship/naval_research/naval_research}
\end{flushleft}

%\begin{flushleft}
%\subfile{naval_collaboration/naval_collaboration}
%\end{flushleft}

\begin{flushleft}
\subfile{sections/scholarship/naval_collaboration/naval_collaboration}
\end{flushleft}

\section{Conclusion}

Since my last supplemental PEGP in 2019, my students and I have made wonderful progress, and I am proud of them.  In particular, I am excited to share my publications.  During the attempted merger of ARA and ARIANNA (Sec. \ref{sec:neutrino}), my field experienced turmoil.  The radar research has given me new insight into my field, along with a quality publication read by researchers all over the world.  The mathematical physics paper to be published in Physical Review D is a great achievement, and my first in that journal.  Below are three papers I have published since Fall 2017 as the main author:
\begin{enumerate}
\item J.C Hanson \textit{et al}.  ``Observation of Classically Forbidden Electromagnetic Wave Propagation and Implications for Neutrino Detection.''  Journal of Cosmology and Astroparticle Physics, n. 7 p. 55 (2018). \url{doi:10.1088/1475-7516/2018/07/055} \textit{and} C. Glaser \textit{et al.} ``NuRadioMC: simulating the radio emission of neutrinos from interaction to detector.'' The European Physical Journal C, vol. 80 n. 2 p. 77 (2020).  \url{doi:10.1140/epjc/s10052-020-7612-8}
\item J.C. Hanson.  ``Broadband RF Phased Array Design with MEEP: Comparisons to Array Theory in Two and Three Dimensions.''  Electronics Journal, vol. 10 n. 4 p. 415 (2021).  \url{doi:10.3390/electronics10040415}
\item J.C. Hanson and R. Hartig. ``Complex Analysis of Askaryan Radiation: A Fully Analytic Model in the Time-Domain.'' \textit{Accepted to Physical Review D.} \url{arXiv:2106.00804} (2021).
\end{enumerate}
The papers in item (1) deal with the issue of ray-tracing and radio propagation in ice.  I produced a ray-tracing solution that accounted for real ice properties, while arguing that the observation of special cases of horizontal propagation was not explained by ray-tracing.  For the second paper in item (1), I was not the corresponding author, but my results were used to create our main simulation package NuRadioMC.  I also wrote much of the appendix regarding our ray-tracing calculations.  Item (2) is my award-winning phased-array design paper using open-source software.  Item (3) is our mathematical physics paper on Askaryan radiation.  This list is by no means complete.  In my field, collaborations of 10-100 people are common, and every name goes on the author list in alphabetical order regardless of contribution level.  Since 2017, I have helped to write many papers for which my work was an integral part, but I am not the ``corresponding author.''  For a full list of publications, I have provided my CV in the supplemental material.
\\
\vspace{0.25cm}
Thankfully, my field has recovered from the turmoil, and we look forward to many discoveries ahead.  I decided to move my remarks about the Whittier Scholars Program to Sec. \ref{sec:advising_mentoring}, Advising and Mentoring, because my part of that project was much more about guidance and management than hands-on research work.  It falls under the \textit{Boyer} catagory \textit{scholarship of integration}, and deals with a holistic study of the impact and changing nature of glaciers.

\end{document}
