\documentclass[../../../main.tex]{subfiles}

\begin{document}

\section{CEM and Engineering with the ONR}
\label{sec:naval_research}

During 2019-2020, it became clear that not only were missions to Antarctica postponed, but that progress in my field will only resume once the IceCube Gen2 design is finalized.  This is expected to happen in Fall 2021.  I have been participating in the IceCube Gen2 radio task-force, which meets weekly to share results and plans for the design.  In the mean time, I decided to widen my research profile so I could make meaningful progress during quarantine.
\\
\vspace{0.25cm}
In the Fall of 2019, three separate individuals inquired if I would be interested in the Office of Naval Research (ONR) Summer Faculty Research Program.  An old friend from graduate school who works at the Naval Surface Warfare Center (NSWC) in Corona, CA, sent me a note.  A contractor working for the Navy contacted me.  Finally, my department chair, Prof. Seamus Lagan, forwarded me a message similar to that of the contractor.  There is a long tradition of cross-over and cooperation between academics in physics, math, computer science, and research sponsored by the armed services.
\\
\vspace{0.25cm}
After applying, I was contacted by Dr. Christopher Clark (NSWC), from the division at NSWC that focuses on radar applications.  The Navy relies on radar for defending ships and aircraft from missiles.  Scientifically, though, the Navy has always been at the forefront of \textit{computational electromagnetics} (CEM) \cite{cem}.  For example, CEM provides engineers a method to design 5G telecommunications systems, and scientists a framework for modelling electromagnetic effects in weather.  In my sub-field of physics, CEM will be useful for understanding the propagation of radio waves through ice in precise detail.  The Navy has created research programs much like those in civilian national labs (e.g. LANL, LBL) that provide the public with technological research (\url{https://www.onr.navy.mil/}).
\\
\vspace{0.25cm}
I have been awarded this ONR Summer Faculty grant in Summers 2020 and 2021, and the connections to my physics research have been fruitful.  In Sec. \ref{sec:cem}, I described the connection between my physics research and the CEM skill I gained from my Navy collaboration.  In Sec. \ref{sec:phased_array}, I discuss the applications to Navy radar research.  My students and I use CEM to design radar antennas with useful properties like high bandwidth, low amounts of energy loss, and the ability to steer a radar beam in the right direction.  In Sec. \ref{sec:3d_printer}, I discuss the efforts of my Navy collaboration to use 3D printers to fabricate our radar designs.  We are currently collecting data in our SLC lab, and hope to have results to report soon.  In Sec. \ref{sec:applications}, I review the potential applications for ONR, UHE-$\nu$ physics, and RF antenna design in general.

\subsection{CEM Phased Array Design for Radar}
\label{sec:phased_array}

In Sec. \ref{sec:neutrino}, I described how \textit{phased arrays} will be useful for UHE-$\nu$ physics.  Imagine a radio antenna radiating power at some frequency.  The direction corresponding to the maximum power is called the \textit{boresight.}  The antenna always radiates some power away from boresight.  The \textit{radiation pattern} is a graph of this relative power versus angle from boresight.  The radiation pattern of a single radio antenna is a function of the shape of the antenna.  The field of RF antenna design is concerned with clever ways to find shapes that have efficient radiation patterns for a given frequency range.  To track moving targets, the radar antenna must be \textit{steered} so that the target falls within the strong part of the radiation pattern.
\\
\vspace{0.25cm}
A phased array is simply an arrangement of identical radio antennas that vary in phase or timing.  By cleverly arranging the phases or relative timing of signals radiated by each array element, the overall radiation pattern \textit{steers} without any moving parts.  Imagine a line of pebbles dropped in a pond.  If they are each dropped at the same time, the water wave of each pebble helps to form one large wave proceeding in a direction perpendicular to the line.  Now imagine that the first pebble is dropped, then a fraction of a second later the second, and then a fraction of a second later the third, and so on.  The ensuing wave will proceed in a different direction.  Phased array radar is this principle but applied to electromagnetic waves instead of water waves.
\\
\vspace{0.25cm}
I summarized our CEM development in a recent publication \cite{electronics10040415}.  For our CEM code, I selected the open-source Python3 package MEEP\footnote{MIT Electromagnetics Equation Propagation: \url{https://meep.readthedocs.io/en/latest/}.}, and used it to produce phased array designs.  I explored Yagi-Uda antennas (lines of dipoles much like a TV antenna), and horn antennas (shaped like a lily).  The radiation patterns match theoretical predictions beautifully.  I experimented with \textit{one-dimensional} and \textit{two-dimensional} phased arrays.  The term one-dimensional refers to the line of identical RF elements mentioned above in the water analogy.  A two-dimensional phased array is the same idea, but a grid of antennas.  A grid allows for beem-steering in two angles (azimuth and zenith angle).  This work was remarkable for at least three reasons.
\\
\vspace{0.25cm}
First, I believe my work represents the first time it has been shown MEEP can produce RF phased array designs.  RF antennas are usually designed with expensive, proprietary software\footnote{For example, xFDTD by RemCom costs between \$5k and \$10k, depending on the desired features.}.  MEEP is open-source, so my students and I can create designs for free.  MEEP was originally designed for applications with micrometer wavelengths (optical photonics).  However, Maxwell's equations that govern electromagnetism demonstrate \textit{scale independence}.  For example, if an object is a few micrometers across, then it probably emits radiation with micrometer wavelengths when energized.  But if I scale its size to a few centimeters across, the radiation is the same pattern, but with centimeter wavelengths.  We adapted MEEP to describe phased arrays where antennas are centimeters long.  The authors of a review of open-source results for simple antennas in Electronics Journal (\cite{10.3390/electronics8121506}) contacted me out of the blue, to offer congratulations and to request code examples so they could utilize these tools\footnote{Correspondence included in supplemental material.}.
\\
\vspace{0.25cm}
Second, IceCube Gen2 will utilize phased arrays to detect UHE-$\nu$ in an environment with non-constant index of refraction (see Sec. \ref{sec:cem}).  The index of refraction is usually called $n$, and the speed of light is $c$.  In a medium with an index of refraction, electromagnetic waves travel at speed $v = c/n$.  If $n$ changes with ice depth, the path of radiation through the ice is curved.  Normally the software used by the IceCube Gen2 and RNO-G collaborations (xFDTD) to design phased arrays does not account for depth-dependent index of refraction. In our recent work \cite{electronics10040415}, I explored the effect of the depth-dependence of $n$ on our phased array designs.  In future work, I will incorporate our knowledge of the internal ice layers frozen into the \textit{firn} (upper portion of the ice sheet made of snow and ice).  These layers act like hidden mirrors and can alter the results of IceCube Gen2.  A similar work by a colleague uses a slightly different method \cite{prohira}.
\\
\vspace{0.25cm}
Third, the results show that our design forms the proper wave $\approx 10$ cm in front of the antennas.  The system would be compact enough to install in an \textit{anechoic chamber}, acting as a model radar echo to test other radars.  A large anechoic chamber is a room with no RF reflections.  Anechoic chambers have shapes built into the walls that cancel RF reflections, making testing a radar system possible without the confusion of reflections.  Anechoic chambers large enough for radar are rare\footnote{For example, the CReSIS anechoic chamber: \url{http://chamber.ku.edu}.}.  Smaller ones designed for smartphones are more common.  Our design is so compact it potentially fits in the small variety.  Our design enables the following type of experiment. Imagine bringing in a radar system that has been deployed in the field for years.  It has not been recalibrated, and has been banged around at sea.  When placed in front of our system, it would see a variety of signals changing direction and we could identify any systematic error in the radar output to recalibrate it.  This is one of a variety of applications we envision for the Navy.

\subsection{3D Printing of RF Antennas}
\label{sec:3d_printer}

This summer, my ONR colleagues and my student, Adam Wildanger, and I have begun fabricating RF antennas using 3D printing.  Adam has shown me how to use a variety of computer assisted design (CAD) programs.  We identified a CAD program that can render a design into a format that can be imported to both MEEP and 3D printers.  This was a huge leap forward, as we can now draw in the CAD program any shape we need and load it into MEEP.  Thus, I can model the radiation of the design with CEM \textit{and} manufacture the same design with a 3D printer capable of using the CAD file.  Using a 3D printer has a number of unique advantages.  Strangely shaped antennas designed with machine learning can have highly useful radiation patterns and bandwidth \cite{10.3390/electronics10121377} \cite{10.1109/access.2019.2932912}.  Manufacturing such antennas with standard shop tools is time-consuming.  With a 3D printer, CAD file, and appropriate print material, the process is automated.

\subsection{Applications to Mobile Broadband}
\label{sec:applications}

There are a variety of applications for this research.  Progress in high-gain ultra-wideband radar development is hindered because desireable parameters compete with each other.  The authors of \cite{10.3390/electronics10121377} created a highly intriguing system that appears to have beaten some of these limitations.  The design, however, is cumbersome to manufacture using standard methods.  The antenna has a very peculiar shape involving important features from millimeters to 10 centimeters.  One intriguing application would be to apply 3D printing to such antennas for 5G mobile broadband.  The mid-band designation for 5G is [2.5-3.7] GHz, right where our design and others radiate efficiently.  If we can create phased arrays of complex antennas in this band with 3D printing, we could reduce the cost of 5G mobile production.
\\
\vspace{0.25cm}
For UHE-$\nu$ physics, the 3D printing strategy is promising for phased array design and construction for at least two reasons.  First, learning to use CEM tools like MEEP will eliminate the cost of expensive proprietary software, and foster inclusion in the design process.  Our current tool, xFDTD, costs thousands of dollars for each installation.  When open-source software is selected, any collaboration member (i.e. a Whittier Undergraduate) could participate in detector design.  Requiring proprietary software implies that only students at institutions that control significant financial resources can create the designs.  Second, 3D printing addresses a peculiar challenge faced by IceCube Gen2.  We would like to drill boreholes in the ice and hang our antennas down the boreholes.  Simulations show that this makes the detector more sensitive to UHE-$\nu$.  However, this means our antennas need to be shaped like the borehole (a cylinder).  This is not a problem if the radio pulse \textit{polarization} is vertical, but a horizontal signal would not produce a strong antenna response.  With 3D printing, we can imagine fabricating novel borehole antenna shapes sensitive to horizontal polarization.

\end{document}
