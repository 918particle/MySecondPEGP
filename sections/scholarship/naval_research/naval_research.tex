\documentclass[../../../main.tex]{subfiles}

\begin{document}

\section{CEM and Engineering with the ONR}
\label{sec:naval_research}

During 2019-2020, it became clear that not only were missions to Antarctica postponed, but that progress in my field will only resume once the IceCube Gen2 design is finalized.  This is expected to happen in Fall 2021.  I have been participating in the IceCube Gen2 radio task-force, which meets weekly to share results and plans for the design.  In the mean time, I decided to widen my research profile so I could make meaningful progress during quarantine.
\\
\vspace{0.25cm}
In the Fall of 2019, three separate individuals sent me emails, inquiring if I would be interested in something called the Summer Faculty Research Program, administered by the Office of Naval Research (ONR).  An old friend from graduate school who works at the Naval Surface Warfare Center (NSWC) in Corona, CA, sent me a note.  A contractor working for the Navy contacted me.  Finally, my department chair, Prof. Seamus Lagan, forwarded me a message similar to that of the contractor.  There is a long tradition of cross-over and cooperation between academics in physics, math, computer science, and the military.  Because three separate people suggested I consider this program, I decided to apply.
\\
\vspace{0.25cm}
I was contacted by Dr. Christopher Clark, who explained to me that there is a division at NSWC that focuses on radar applications.  The Navy relies on radar for defending ships and aircraft, and defending our nation from missle attacks.  Scientifically, though, the Navy has always been at the forefront of \textit{computational electromagnetics} (CEM) \cite{cem}.  For example, CEM provides engineers a way to design and manage systems that manipulate microwaves for 5G telecommunications.  CEM can be used for modelling electromagnetic effects in weather.  In my sub-field of physics, CEM will be useful for understanding the propagation of radio waves through ice in precise detail.  The Navy has created research programs much like those in civilian national labs (e.g. LANL, LBL) that provide the public with technological research (\url{https://www.onr.navy.mil/}).
\\
\vspace{0.25cm}
The connections to my physics research have been fruitful.  In Sec. \ref{sec:cem}, I described the connection between my physics research and the CEM skill I gained from my Navy collaboration.  In Sec. \ref{sec:phased_array}, I discuss the applications to radar and how they are useful for the ONR.  My students and I use CEM to design radar antennas with useful properties like high bandwidth, low amounts of energy loss, and the ability to steer a radar beam in the right direction.  In Sec. \ref{sec:3d_printer}, I discuss the efforts of my Navy collaboration to use 3D printers to fabricate our radar designs.  We are currently collecting data in our lab in the Science and Learning Center, and hope to have results to report soon.  In Sec. \ref{sec:applications}, I review the potential applications for ONR, UHE-$\nu$ physics, and RF antenna design in general.

\subsection{CEM Phased Array Design for Radar}
\label{sec:phased_array}

In Sec. \ref{sec:neutrino}, I described how \textit{phased arrays} will be useful for UHE-$\nu$ physics.  Imagine a radio antenna radiating power at some frequency.  The direction corresponding to the maximum power is called the \textit{boresight.}  The antenna always radiates some power away from boresight.  The \textit{radiation pattern} is a graph of this relative power versus angle from boresight.  The radiation pattern of a single radio antenna is just a function of the fixed shape of the antenna.  The field of RF antenna design is concerned with clever ways to find shapes that have efficient radiation patterns for a given frequency range.  To track moving targets, the radar antenna must be \textit{steered} so that the target falls within the strong part of the radiation pattern.  There are disadvantages, however, to mechanical steering.
\\
\vspace{0.25cm}
A phased array is simply an arrangement of identical radio antennas that vary in phase or timing.  By cleverly arranging the phases or relative timing of signals radiated by each array element, the overall radiation pattern \textit{steers} without any moving parts.  Imagine a line of pebbles dropped in a pond.  If they are each dropped at the same time, the water wave of each pebble helps to form one large wave proceeding in a direction perpendicular to the line.  Now imagine that the first pebble is dropped, then a fraction of a second later the second, and then a fraction of a second later the third, and so on.  The ensuing wave will proceed in a different direction.  Phased array radar is this principle but applied to electromagnetic waves instead of water waves.
\\
\vspace{0.25cm}
I summarized our CEM development in a recent publication \cite{electronics10040415}.  I selected the open-source Python3 package MEEP\footnote{MIT Electromagnetics Equation Propagation: \url{https://meep.readthedocs.io/en/latest/}.}, and used it to produce single-frequency and broad bandwidth phased array designs.  I explored Yagi-Uda antennas (lines of dipoles much like a TV antenna), and horn antennas (shaped like a lily).  The radiation patterns match theoretical predictions beautifully.  I experimented with \textit{one-dimensional} and \textit{two-dimensional} phased arrays.  The term one-dimensional refers to the line of identical RF elements mentioned above in the water analogy.  A two-dimensional phased array is the same idea, but a grid of antennas.  A grid allows for beem-steering in two angles (azimuth and zenith angle).  This work was groundbreaking for at least three reasons.
\\
\vspace{0.25cm}
First, after performing a simple literature review, I believe my work represents the first time anyone has shown MEEP can produce phased array designs.  RF antennas are usually designed with expensive, proprietary software

\subsection{3D Printing of RF Antennas}
\label{sec:3d_printer}

\subsection{Broader Applications}
F
\end{document}
