\documentclass[../../../main.tex]{subfiles}

\begin{document}
A
\section{IceCube, Cosmic Rays, and Neutrinos from Deep Space}

\textit{Cosmic rays} are high-energy protons, electrons, and nuclei propagating through space near the speed of light.  They carry information from other regions in the galaxy, and in some case, other galaxies.  Since the discovery of extremely energetic cosmic rays more than a half century ago, the elusive quest to uncover the sources of these enigmatic particles has provided many challenges.  Despite progress in experimental capabilities and theoretical insight, we do not yet know the acceleration mechanism for those particles with energies that have been measured in excess of $10^{20}$ electron-volts \cite{10.1088/1742-6596/1766/1/012002}.  Being electrically charged, the paths of cosmic rays are curved by galactic and intergalactic magnetic fields.  By the time the cosmic ray arrives at Earth, the arrival direction no longer points back to their origin.  In addition, interactions with cosmic microwave background photons prevent ultra-high energy cosmic rays from propagating to the Earth, unless the sources are in our local galactic cluster \cite{PhysRevLett.16.748} \cite{1966JETPL...4...78Z}.
\\
\vspace{0.25cm}
Neutrino astronomy offers a new and powerful tool to provide insight into the physics associated with the acceleration process, and complements and extends measurements not accessible through the observation of other \textit{messengers}: cosmic-rays, gamma-rays, and  optical photons. Charged cosmic rays which interact with gas, dust, or radiation near an accelerating object produce gamma-rays and high-energy neutrinos.  These neutrinos are called \textit{astrophysical} neutrinos. Whereas gamma-rays can be absorbed in dense environments, astrophysical neutrinos can escape and travel unimpeded to a detector (\cite{Astro2020_1} and references therein). Neutrinos travel at the speed of light in straight lines, undeflected by magnetic fields.  This allows for identification of sources, as well as the potential for finding sources that emit both neutrinos and gravitational waves, which also travel in straight lines \cite{10.3847/2041-8213/ab9d24}.
\\
\vspace{0.25cm}
The most energetic cosmic rays which do escape their source can interact with the cosmic microwave background en route to the Earth, generating cosmogenic neutrinos with a characteristic energy distribution peaking at $10^{18}$ electron-volts \cite{10.1007/bf00645585} \cite{BERESINSKY1969423}. The distance over which these neutrinos could physically propagate given the distribution of cosmic microwave background photons is larger than the known Universe.  So despite the distance limitations of cosmic rays, neutrinos offer a window into regions of the Universe \textit{far beyond anything possible} with other messengers.
\\
\vspace{0.25cm}
The flux of neutrinos originating from outside the solar system with energies between $10^{13}$ and $10^{15}$ electron-volts has been measured by the IceCube collaboration \cite{PhysRevLett.111.021103}. Previous analyses have shown that the discovery of ultra-high energy neutrinos (UHE-$\nu$, energy greater than $10^{15}$ electron-volts) will require an upgraded detector design with a larger effective volume because the flux is expected to decrease with energy \cite{PhysRevD.98.062003}. Neutrinos with energies above $10^{15}$ are the ones that could potentially explain the origin of cosmic rays, and provide the chance to study quantum mechanical interactions at record-breaking energies \cite{Astro2020_1} \cite{Astro2020_2}.

\subsection{Why Antarctica?}

Utilizing the \textit{Askaryan effect}, in which UHE-$\nu$ creates a radio-frequency pulse, greatly expands the effective volume of UHE-$\nu$ detector designs.  This effect offers a way to detect UHE-$\nu$ with radio pulses that travel more than 1 kilometer in sufficiently RF-transparent media such as Antarctic and Greenlandic ice \cite{10.3189/2015jog14j214} \cite{10.3189/2015jog15j057} \cite{10.1002/2015rs005849} \cite{10.1016/j.astropartphys.2011.11.010}.  Because the 

\subsection{Radio Expansions: IceCube Generation 2}
D
\section{My Professional Background}
E
\subsection{Prior to Whittier College}
F
\subsection{Successes with Our Students at Whittier College}
G
\subsection{Forming Connections with the Office of Naval Research (ONR)}
H

\end{document}
