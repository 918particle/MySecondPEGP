\documentclass[../../../main.tex]{subfiles}

\begin{document}

\section{My Vision for Collaboration between ONR and Whittier College}
\label{sec:naval_collaboration}

According to an economic analysis by the Los Angeles County Economic Development Corporation in 2016, Southern California plays a vital and increasingly large economic role in the aerospace and defense sectors \cite{laedc}.  This sector is responsible for the Global Positioning System (GPS), Mars rovers, missle defense systems, and radar development for new aircraft.  Southern California is home to world reknowned engineering leaders like NASA Jet Propulsion Laboratory (JPL) and SpaceX Corporation.  Young engineers graduating from college understand that their skills are needed in this dynamic sector of the regional economy.
\\
\vspace{0.25cm}
Students know that these roles are quality employment.  The industry is expected to create 5630 new job openings over the next five years, with 3380 of those labeled replacements.  Replacements are created when a worker retires or is promoted.  Thus, the industry is creating new opportunities, but also needs young people to step forward.  Forty percent of all new jobs will require a bachelor's degree.  Over 90,000 people in Los Angeles County are employed in aerospace alone, and that figure is over 100,000 if public institutions like JPL are included.  Since 2004, the guided missle and space vehicle sectors have experienced 62 percent growth.  The average yearly salary is \$106k per year, making these employees among the highest paid in the region.  Included the entire supply chain, the Southern California aerospace sector employees approximately 300,000 people.
\\
\vspace{0.25cm}
Several partnerships between higher education and the industry are documented in the report \cite{laedc}.  The report specifically mentions The Aerospace Corporation for its involvement in STEM programs.  One of our physics graduates from Whittier College, Kaitlin Fundell, moved into a Research Associate position at the Aerospace Corporation.  Our department has a contact there, Prof. James Camparo, who is an adjunct professor at Whittier College, and who specializes in atomic clocks.  Kaitlin initially worked in Prof. Camparo's laboratory.  Two other programs of note at The Aerospace Corporation are the Greater Los Angeles Education-Aerospace Partnership (Great-LEAP) program, and the Mathematics, Engineering, and Science Achievement (MESA) program for disadvantaged students.  These specialized programs can be built into curricula so that graduating seniors are amply prepared for the workforce.  One example given is USC engineering, where the seniors participate in year-long design challenges that link them to alumni and industry experts.

\subsection{Building Student Success after Whittier College}

When I began to interact with the Office of Naval Research, my contacts raised the possibility of a more formal partnership with Whittier College.  If we consider the numbers above, the industry will need $\approx 1200$ new engineering participants over the next five years.  Adding up all aerospace engineering graduates from Southern California UC schools, CSU schools, plus CalTech and USC gives 300 graduates per year.  For mechanical engineering, the total is $\approx 1300$, but some large fraction of those will move industries other than aerospace and defense.  Thus, there is a shortage in the industry, including national laboratories like NSWC Corona where my contacts work.  They seek to build partnerships with colleges in the region, and Whittier College is just 40 miles away.  Additionally, Whittier College has an advantage as a small liberal arts college with simple bureaucracy and a diverse group of students.
\\
\vspace{0.25cm}
To date, I have advised five physics and engineering students toward graduation (not counting my WSP student)\footnote{Students: Cassady Smith, John Paul G\'{o}mez-Reed, Nicolas Clarizio, Nicolas Bakken-French (WSP), Raymond Hartig, and Adam Wildanger.}  Three of them are attempting to join the aerospace and defense sector in Southern California: John Paul G\'{o}mez-Reed, Nicolas Clarizio, and Adam Wildanger.  Noting that a large fraction of our STEM students might be considering this sector, we should reflect on the nature of a fruitful partnership program with the ONR.  In a nutshell, we work on engineering research relevant for the ONR, while they provide resources and guidance to our students.
\\
\vspace{0.25cm}
The obvious starting point is to provide research experiences with NSWC Corona staff.  If I continue working with my contacts, I plan to advise 1-2 students per academic year on radar and additive manufacturing (3D printing) projects.  Thus far, I have been awarded two ONR grants (Summers 2020 and 2021) that have provided me funding.  My students and I have been awarded \textit{internal} Keck Fellowships, Fletcher-Jones Fellowships, and the Ondrasik-Groce Fellowship, providing them with stipends (see Tab. \ref{tab:funds}).  I would like to shift the student financial support onto the ONR.  In exchange, I assume they will want students to perform research tasks related to ONR goals.  \textit{Since a large fraction of my students want to perform this research anyways,} it should be possible to assemble.  I can apply for the ONR Summer Faculty Research Fellowship for Summer 2022.  After Summer 2022 I am eligible to reapply at a higher compensation level (Senior Fellow) in Summer 2024.  The \textit{cooling-off year}, 2023, is a Navy requirement.  If I am awarded tenure, my sabbatical would coincide with the cooling off year, so the timing is perfect.

\begin{table}
\centering
\begin{tabular}{c c c c}
Student/Professor & Grant Opportunity & Amount & Dates \\ \hline
Jordan C. Hanson & ONR Summer Faculty Fellow & \$16.5k & Summer 2021 \\
Adam Wildanger & Fletcher Jones Fellowship & \$5k & Summer 2021 \\
Jordan C. Hanson & ONR Summer Faculty Fellow & \$16.5k & Summer 2020 \\
Raymond Hartig & Fletcher Jones Fellowship & \$5k & Summer 2020 \\
John Paul G\'{o}mez-Reed & Ondrasik-Groce Fellowship & \$7.5k & Summer-Fall 2019 \\
John Paul G\'{o}mez-Reed & Keck Fellowship & \$5k & Summer 2018 \\
Cassady Smith & Keck Fellowship & \$5k & Summer 2018 \\
\end{tabular}
\caption{\label{tab:funds} A listing of the grant opportunities awarded to my group.}
\end{table}
 
\subsection{Equipping Whittier College Laboratories}

My colleagues at NSWC have already provided my laboratory in the Science and Learning Center with equipment for RF measurements like characterizing radar antennas.  Table \ref{tab:equip} contains a list of all the components, their purpose, and estimated value.  This equipment is on loan for the six month period starting with August 2021.

\subsection{Financial Support}
E
\end{document}
