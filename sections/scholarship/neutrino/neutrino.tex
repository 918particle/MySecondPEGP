\documentclass[../../../main.tex]{subfiles}

\begin{document}
A
\section{Five Areas of Research Focus}
B
\subsection{Computational Electromagnetism}

Observable Askaryan signals originate in bulk ice, and then propagate to \textit{in situ} RF channels.  The RF pulse follows approximately a ray-tracing solution that depends on the index of refraction: $n(z)$.  The speed of light in the ice is $c/n$, making the speed a function of depth.  The effect is that light travels not in straight lines, but in curved paths called ray-traces.  The index depends on depth ($z$) because the surface is snow $(\rho \approx 0.4$ g cm$^{-3})$ that is compressed to solid ice $(\rho \approx 0.917$ g cm$^{-3})$ over millenia \cite{10.3189/2015jog14j214}.  A classical physics approach called the Lagrangian method minimizes the optical path length of the ray. Combining the Lagrangian approach with a smooth $n(z)$ function fit to $n(z)$ and density data results in a differential equation for the ray tracing solution.  The solution admits: (1) straight-line paths in deep ice (where the speed is constant), (2) quadratic paths near the surface (where the speed changes), and (3) a general form stitching the two together.
\\
\vspace{0.25cm}
The official simulation software used by IceCube Gen2 is named NuRadioMC \cite{10.1140/epjc/s10052-020-7612-8}.  NuRadioMC is built on four pillars: (1) UHE-$\nu$ generation, (2) Askaryan emission, (3) ray tracing, and (4) RF channel simulation.  My analytic ray-tracing solution is pillar (3).  We assume a functional fit for $n(z)$ that is motivated by glaciology.  Snow accumulates at the top of the ice sheet at a certain rate, depending on the yearly conditions.  It is slowly compressed over millenia, steadily increasing in density, which in turn decreases the speed of light within it.  The function is constrained by density and radio measurements \cite{Barwick:2018497}.  The key finding of \cite{Barwick:2018497}, however, was RF \textit{horizontal propagation} not predicted by ray-tracing.  I can correct my ray-tracing framework with a \textit{perturbed} Lagrangian to produce horizontal solutions.  I showed in my PhD dissertation that the effect depends on frequency \cite{hanson}, and therefore cannot be explained with ray-tracing alone.  Horizontal propagation has been noted as an interesting Askaryan detection scheme in the past \cite{10.1103/physrevd.71.011503}.
\\
\vspace{0.25cm}
The inaccuracy of ray-tracing is compounded by the selection of phased arrays as triggers of \textit{in situ} stations.  In this context, phased-arrays are vertical arrangements of identical antennas.  One can think of received signal in terms of relative phase between antennas.  For example, in transmitting mode, the elements of a phased array at fixed frequency are given signals slightly shifted in time, and the time shift increases linearly across the array.  The result is a plane wave emitted at the desired direction.  Conversely, the arrival direction of plane waves from Askaryan signals from UHE-$\nu$ can be deduced with ray-tracing. However, this all assumes a constant index $n(z)$, and we know it is not constant.  Knowing that sensitivity of IceCube Gen2 \textit{in situ} designs depends on the phased array precision, I have created a solution that leaves behind ray-tracing all together.

\subsubsection{Beyond Ray-Tracing: Open-Source Parallel FDTD Methods}

I have received two \textbf{Summer Faculty Research Internship grants} from the Naval Surface Warfare Center (NSWC) Corona Division, in Corona, CA.  My group focuses on phased-array radar development.  My colleagues at other institutions do not always think like scholars in the liberal arts.  Sometimes I am asked how is related to IceCube Gen2 or ARIANNA.  Phased-arrays are useful for testing and verification of all other radar systems because they can mimic a radar reflection that moves without using moving parts.  I adapted a Python3 software packaged called MEEP, originally designed for $\mu$m wavelengths, to work for radio wavelengths \cite{10.1016/j.cpc.2009.11.008}.  From there, I developed phase-array models in which I could control things like the speed of light versus depth (i.e. the problem we face in IceCube Gen2).  I performed design studies for single-frequency and broadband arrays, and the computed properties matched phased-array antenna theory beautifully.  Intriguingly, the work represents the first time MEEP, originally designed for $\mu$m-wavelength applications, had been applied to phased array design.  The results have been published in Electronics Journal \cite{electronics10040415}, where the work has been named among the Top 10 most notable articles of 2020-21\footnote{The notice of these awards is included in the supplemental material.}.
\\
\vspace{0.25cm}
I can envision a wide range of CEM applications for IceCube Gen2.  The phased array trigger for IceCube Gen2 must be designed accounting for the changing index of refraction.  \textit{I plan to combine a Python3 machine learning package with MEEP to optimize and study phased array trigger output given the index of refraction profile and Askaryan emission proerties.}  These computations could be performed in parallel on a small dedicated cluster we would assemble.  I'm currently searching for the right NSF grant to do this, however I do have some startup grant funding remaining that can be used for computer hardware.  If successful, the \textit{in situ} trigger would thus be trained to detect the smallest hint of a UHE-$\nu$ signal.  Theoretically, we expect a higher UHE-$\nu$ flux with smaller amplitudes, so in this way we would be maximizing the scientific output.
\\
\vspace{0.25cm}
My proposed CEM computation cluster would serve a separate physics collaboration called PUEO for another reason \cite{pueo}.  PUEO will seek UHE-$\nu$ signals with arrays of RF elements flown in the atmosphere.  PUEO antennas are the same class as the ones I designed for the Navy, and would be flown on a weather balloon gondola.  PUEO is therefore dubbed an \textit{in air} version of Askaryan-class detector.  PUEO faces a computational requirement that \textit{in situ} detectors do not: refraction from the snow surface to the air.  The Antarctic snow surface roughness has been measured \cite{doi:10.1142/S2251171717400025}.  Those results could serve as boundary conditions, and radiation propagating from bulk ice, refracting from the ice-air interface would be tracked to PUEO arrays with near-to-far-field projection.  In my work published recently in Electronics Journal \cite{electronics10040415}, I produced this type of refraction.  It would not be difficult to add the effect of the rough snow surface at the ice-air interface.

\subsubsection{Connection to Teaching and Academic Mentorship}

We always attempt to form connections between our research and teaching, and my work for the Navy has required me to teach.  My contacts are Dr. Christopher Clark, Dr. Eisa Osman, and Dr. Gary Yeakley, who work on active seeking radar.  Dr. Clark relayed a compliment I received from Dr. Yeakley, who had this to say about my 1-2 hour lectures given once per week in Summer 2020.
\begin{quote}
One of the most stunning compliments Prof. Hanson received was from my Senior RF Engineer, Mr. Gary Yeakley (who has been working developing, designing, and testing radar since the 1970s), where Gary stated ``every week I learn new RF Physics from Jordan [Prof. Hanson].''
\end{quote}
I have included a letter of advocacy in the supplemental materials from Dr. Clark.  Researching a new subject, mastering it, and teaching it to colleagues in the Navy was a rewarding experience, and I was grateful to serve.  My Navy contacts invited me back for Summer 2021 to teach an RF Field Engineering Course.  In practice, this amounted to creating tutorial videos my colleagues can download.  This content will also be useful for digital signal processing (DSP), my upcoming January term course.
\\
\vspace{0.25cm}
This past summer (2021), I included in this work a student named Adam Wildanger through a Fletcher-Jones Fellowship.  Adam is a great help, because he has taught me computer assisted design (CAD).  The process begins with me creating an antenna design using CEM tools.  Next, Adam ports my work as a machine-readable design using his favorite CAD programs.  Third, Adam sends the machine-readable design to our Navy colleagues, who fabricate it using a 3D printer.  Finally, they have sent the fabricated components to me.  Adam and I will test them in my laboratory in the Science and Learning Center using equipment provided by ONR.  The key is to use 3D printer material that conducts some amount of electric current at high frequencies.  If successful, this collaborative effort has applications as diverse as UHE-$\nu$ research, radar development, 5G mobile communications, and remote sensing for climate science (\url{https://cresis.ku.edu/}).  The key lesson here is that sometimes, the mentor can learn new things from the student.  I am working on securing Adam an position at the national lab where my ONR colleagues are based.  That has been Adam's dream job throughout college.

\subsection{Mathematical Physics}
\label{sec:math_phys}

In 2015, I became a CCAPP Fellow at The Ohio State University, where I began to work on an analytic Askaryan radiation model.  At the time, the simulation package for IceCube Gen2 (NuRadioMC) was under development.  The two \textit{in situ} groups, ARIANNA and the Askaryan Radio Array (ARA), relied on the MC codes ShelfMC and AraSim, respectively.  Both ShelfMC and AraSim were derived from the same legacy code.  We learned, however, that ShelfMC and AraSim did not always produce the same results, and were cumbersome to compare.  Further, the Askaryan models in both were 5-10 years old, and derived from \textit{semi-analytic} parameterizations.
\\
\vspace{0.25cm}
The results of foundational work in Askaryan effect simulated every single sub-atomic particle in the cascade initiated by the UHE-$\nu$, and the corresponding radiation \cite{zhs}.  Overall radio pulse amplitude was found to be proportional to the energy of the neutrino.  A distribution of radiated power was observed to radiate in a special direction: the Cherenkov angle.  Such models were called \textit{full-MC} models.  By tradition, physics simulations are sometimes called \textit{Monte Carlo} simulations (MC).  Semi-analytic parameterization models provide part of the electric field at the Cherenkov angle, and simulate the cascade development along the UHE-$\nu$ direction.  Mixing these two results produces the electric field (radio wave) at a variety of angles \cite{PhysRevD.101.083005}.

\subsubsection{Frequency-Domain Model}

Missing from the 2018 ShelfMC/AraSim integrations was a common analytic understanding of Askaryan radiation.  I responded by developing a fully analytic model\footnote{In this sense, analytic means a set of equations, not a simulation.} that accounted for several important effects \cite{10.1016/j.astropartphys.2017.03.008}.  The independent variable in the equations is the frequency of the radio wave, so the model is classified as a \textit{frequency-domain} model.  I was inspired by work by Prof. John Ralston and Roman Buniy \cite{10.1103/physrevd.65.016003}.  Using GEANT4 MC simulations on the Ohio Supercomputing Cluster (OSC), we determined the shape of the electric charge distribution in the UHE-$\nu$ induced cascades with total energies of $10^{17}$ electron-volts.
\\
\vspace{0.25cm}
We then wrote a function that followed this shape, and finished the ensuing electromagnetic calculations to obtain the radio wave.  My model produces template waveforms for UHE-$\nu$ searches with IceCube Gen2 \cite{10.1016/j.astropartphys.2017.03.008}.  The community began to use the model after I presented it at workshops at KICP (Univ. of Chicago), and TeVPA conferences.  Colleagues shared with me that a time-domain model at all angles relative to the Cherenkov angle would be highly useful.  In the final section of \cite{10.1016/j.astropartphys.2017.03.008}, we did provide an example, but only if the viewing angle equals the Cherenkov angle.

\subsubsection{Time-Domain Model}

There are four main advantages of analytic time-domain models. First, when they are matched to observed radio waveforms, UHE-$\nu$ cascade properties like total energy may be derived directly from waveform shapes. Second, evaluating a fully analytic modelmtechnically provides a speed advantage in software compared to the semi-analytic parameterizations. Third, when analytic models are combined with RF antenna properties (derived using CEM), the resulting template can be embedded in detector firmare to form a filter that enhances the probability that a passing UHE-$\nu$ signal is detected, rather than be mistaken for radio noise. Fourth, parameters in analytic models may be scaled to account for snow density in addition to ice density. This application is useful for understanding potential signals in the Antarctic firn, or the upper 100-meter layer of snow that rests on top of the ice.  My student, Raymond Hartig, and I are proud to announce that our work will be published in Physical Review D \cite{time}.  My vision for the future of this work at is expansive, and involves striving for progress along three tracks.
\\
\vspace{0.25cm}
The first track involves UHE-$\nu$ template analysis.  Our simulation for IceCube Gen2, NuRadioMC, is broken into four pillars (steps).  Currently, UHE-$\nu$ are simulated first as \textit{events} (NuRadioMC pillar (1)) and the \textit{RF emissions} (Askaryan signals) are generated next (NuRadioMC pillar (2)).  Our ability to match simulated waveforms to potential UHE-$\nu$ waveforms from the detector is limited because we cannot scan through properties of the simulated \textit{cascades}, only UHE-$\nu$ with a single RF emissions model.  For example, two UHE-$\nu$ with the same energy could generate different cascades with different shapes of electric charge.  The effect of the cascade shape is important for the interpretation of future IceCube Gen2 data.  Conversely, if the effect of the cascade shape is well-understood, it becomes possible to measure the UHE-$\nu$ energy by templates to observed data \cite{time}.
\\
\vspace{0.25cm}
The second track involves embedding the model itself in detector firmware.  Potential UHE-$\nu$ signals are recorded alongside noise, but all data has to be shipped with limited bandwidth.  Embedding the model on the detector would allow us to flag priority events.  The physics community expects IceCube Gen2 to provide an alert system so that other physics and astronomy collaborations could search for any UHE-$\nu$ or cosmic-ray sources we identify.  This is not possible if the data has to first shipped back to the United States and then be searched with powerful algorithms.  Sending priority events that already correlate with analytic predictions is a strategy that solves the problem.
\\
\vspace{0.25cm}
The third track involves the connection between CEM and our Askaryan model.  In NuRadioMC the simulated signal is created by code in pillars (1)-(4) sequentially.  That is, the basic Askaryan model is mixed with detector response \textit{after} ray-tracing.  In reality, the radiation flows immediately from the cascade according to the details of the index of refraction of the ice.  It is a wave that \textit{generally} follows ray-tracing, but that reflects from internal ice layers, propagates horizontally, and can change shape.  In other words, all the effects \textit{not captured} by the original index of refraction function.  In this regard, fully-analytic Askaryan models have a unique advantage: analytic equations can be implemented as MEEP sources, and MEEP \textit{can} account for all those effects, while ray-tracing cannot.  The analytic model is in a unique position to provide advanced insight into the effect of 3D propagation previously unexplored.

\subsubsection{Connection to Teaching and Academic Mentorship}

Researching mathematical physics and radio waves has sharpened my teaching of electromagnetism for obvious reasons.  I taught our version of upper-division electromagnetism this past semester: Electromagnetic Theory (PHYS330).  I included detail about this experience, which was wonderful, in Sec. \ref{sec:teaching}.
\\
\vspace{0.25cm}
I have been mentoring an undergraduate double-major in physics and mathematics named Raymond Hartig who plans to attend graduate school.  Our partnership began in Spring 2020, when Raymond approached me for training in complex analysis.  These are the mathematical tools necessary to perform some electromagnetic calculations.  We then won the Fletcher-Jones Undergraduate Research Fellowship for that summer.  The experience of coaching him in his development as a theoretical physicist has been rewarding, and I have taken note of a key skill project leaders must have.
\\
\vspace{0.25cm}
Similar to teaching courses, one has to think pedagogically when explaining projects to young researchers.  If they understand the direction at least as well as the syllabus of a course, they are more likely to succeed.  \textit{From the perspective of equity and inclusion}, my experience mentoring students pedagogically is useful.  By structuring the students' time, and explaining goals and productivity expectations in advance, students from diverse backgrounds feel called to participate in the same sense that they feel invited to participate in the classroom.  Sometimes research can make first-generation students, for example, feel that the work is not for them because it is too unfamiliar.  Providing structure pedagogically can eliminate that stress and increase productivity.

\subsection{Firmware, Software, and Hardware Development}

Askaryan-class detectors must operate autonomously due to the limited Antarctic infrastructure.  Stations must be powered sustainably with solar panels and wind turbines, and communications bandwidth is restricted to satellite modems \cite{10.1109/tns.2015.2468182}.  Every sub-system that can operate autonomously is another sub-system that does not use bandwidth.  The ARIANNA stations, for example, send text messages via satellite modem to the server in the United States.  Configuration files are sent by the server to the stations with operational instructions.  This includes channel thresholds, which control the RF thermal trigger rate.  All radio and radar systems trigger inevitably on \textit{thermal noise}.  A lower threshold increases the chance of hearing signal at the cost of recording more thermal noise.  To adjust thresholds in response to fluctuating thermal noise (which can fill up detector memory with useless data), we have to analyze the data between satellite messages, optimize thresholds, and place new instructions in the server queue.  The stations can therefore run autonomously without power-hungry, high-bandwidth ethernet.  This process, however, should be automated for hundreds of stations in IceCube Gen2.

\subsubsection{The Multi-Mode Frequency Counter (MMFC) and ARIANNA}

My student, John Paul G\'{o}mez-Reed and I developed firmware for the ARIANNA boards that would perform this automation.  This was a two-year process that began when we won the Keck Fellowship in the summer of 2018. First, we learned to design and load firmware into circuits.  John Paul named the system the Multi-Mode Frequency Counter (MMFC), because it was a sophisticated counter that measured the number of times per second a particular ARIANNA channel was triggered.  We demonstrated it could measure channel trigger rates from from 10 hits per second to 10 million hits per second.  (When a channel is triggered millions of times per second, that is the effect of just noise and no interesting RF signal).  The digital input was provided by RF lab equipment I purchased for Whittier College using my start up grant. John Paul presented the results at the Southern California Conference for Undergraduate Research (SCCUR) \cite{sccur1}.  We then received ARIANNA systems from UC Irvine for systems integration.  \textbf{In summer 2019, we won the Ondrasik-Groce Fellowsip.}  Throughout 2019 and early 2020, we used that fellowship to integrate the MMFC into the ARIANNA circuitry.  The circuit boards began to auto-adjust station thresholds.  We were in the process of final testing when the pandemic forced us to pause.  We did, however, present at SCCUR a second time \cite{sccur2}.

\subsubsection{Future Plans and Applications}

To continue this research, I have submitted a grant proposal to the Cottrell Scholars Program.  The proposal outlines in detail the next three phases of this research, broken into concrete steps within each phase.  The overall goal is to enhance the trigger capability of our detectors with my analytic Askaryan models (Sec. \ref{sec:math_phys}).
\\
\vspace{0.25cm}
Phase 1 includes two main steps: (1) completing the integrated threshold automation firmware and software, and (2) completing the analytic Askaryan model.  As of this summer, both (1) and (2) are complete.  The integration of (1) into ARIANNA systems had to be put on hold due to the pandemic, but both the software and firmware are written.  The latest version of the Askaryan model is complete, and will be published in Physical Review D \cite{time}.  \textbf{Thus, phase 1 is already complete.}
\\
\vspace{0.25cm}
Phase 2 includes two main steps: (1) learning to match templates to data in firmware, and (2) demonstrating the system in an anechoic chamber\footnote{An anechoic chamber is a space that blocks all radio noise and reflections for the testing of sensitive RF equipment.}.  The reliability of this process must be proven before the firmware is deployed in Antarctica.  The key to step (2) is to calibrate the whole planned RF chain: a signal generator programmed with my analytic Askaryan model, transmitting and receiving antennas, amplifiers, filters, and digitization circuits.  I have performed similar processes as a post-doctoral fellow at KU \cite{10.1016/j.astropartphys.2014.09.002}.
\\
\vspace{0.25cm}
Phase 3 includes three steps: (1) publication of threshold automation and Askaryan model, (2) installation of the Askaryan-trigger firmware in detectors, and (3) field deployment.  It is wise to have the components of this project peer-reviewed and published before installing them in many detectors.  As of this summer, the Askaryan has passed peer review.  Once the IceCube Gen2 collaboration has a chance to review the firmware, we could move forward with deployment.

\subsubsection{Connection to Teaching and Academic Mentorship}

There are important connections to mentorship and teaching within this work.  First, John Paul G\'{o}mez-Reed is a Whittier local, and from a background under-represented in physics and engineering.  I was able to coach him through courses, application processes, and two SCCUR conferences.  We are currently working on graduate school and job applications.  Working with him has honed my mentorship skills, including when to hammer out details in the lab with my students, and when to leave them to figure it out on their own.  Once I learned to sense that ...
\\
\vspace{0.25cm}
I have taught two courses connected to this research: Computer Logic and Digital Circuit Design (PHYS306/COSC330), and Digital Signal Processing (COSC390), which introduced an interesting synergy.  When I selected the Xilinx pynq-z1 board (\url{www.pynq.io}) for PHYS306, the system allowed students to operate a Unix-based processing system (PS) integrated with a programmable logic (PL) firmware layer using Jupyter notebooks (Python3).  Learning how to use Jupyter notebooks boosted my CEM research, because MEEP work is often done in Jupyter.

\subsection{Open-source Antenna Design}
F
\subsection{Drone Development and The Whittier Scholars Program}
G
\section{Invitation to Become a Member Institution of IceCube}
H
\end{document}
