\documentclass[../../../main.tex]{subfiles}

\begin{document}

\section{Invitation to Become a Member Institution of IceCube}
\label{sec:invite}

Recently, I was invited to become an official member of the IceCube Collaboration.  The IceCube Collaboration (\url{https://icecube.wisc.edu}) includes more than 300 physicists from 53 institutions in 12 countries. It began in 1999 with the submission of the first IceCube grant proposal, and many of the original members are still active on the project.  Scientists and engineers came together to build what is now the largest neutrino detector in the world.  A membership in the group at the forefront of neutrino physics is an honor, and it represents a great opportunity for Whittier College.  Our IceCube Membership brings several advantages for Whittier College.  Whittier College will be added to the \textit{list of member institutions} (included in the supporting materials).  Our students and professors would gain access to archived IceCube data.  Whittier College will also gain visibility as the only Title V HSI on the list.  Whittier College will be added to IceCube publications, and my students and I may submit papers on behalf of the IceCube Collaboration.  Whittier students and professors could attend the annual IceCube Collaboration Meeting, which is a physics conference specializing in neutrino physics and astrophysics.  Finally, I could use my IceCube membership status to help with grant proposals for items like computer clusters (see Sec. \ref{sec:cem_cluster}).

\section{Five Areas of Research Focus}
\label{sec:neutrino}

In the following five sections, I highlight five key areas of impact into which my UHE-$\nu$ research activities can be classified, and I explain how they are interconnected.  In each section, I also show how the scholarship connects to my teaching and mentorship.

\subsection{Computational Electromagnetism}
\label{sec:cem}

Askaryan signals originate in ice and propagate to \textit{in situ} RF channels.  The path they follow is called a ray-tracing solution, and it depends on the speed of light in ice versus depth.  The ray-tracing solution is curved because the speed of light changes according to the \textit{index of refraction}, $n$, which depends on ice and snow density.  If $v$ is the measured speed of light, and $c$ is the speed of light in air, and $n$ is the index, then $v = c/n$.  The density (and thus the index) depends on depth because the surface is snow, with density $\approx 0.4$ g cm$^{-3}$, that is compressed to solid ice $\approx 0.917$ g cm$^{-3}$ over millenia \cite{10.3189/2015jog14j214}.  I have used the \textit{Lagrangian method} and the observed $n$ versus depth to produce ray-tracing solutions.  The solution has special cases: (1) straight-line paths in deep ice (where the speed is constant), (2) quadratic paths near the surface (where the speed changes), and (3) a general form stitching the two together.
\\
\vspace{0.15cm}
The official simulation software used by IceCube Gen2 is named NuRadioMC \cite{10.1140/epjc/s10052-020-7612-8}.  NuRadioMC is built on four pillars: (1) UHE-$\nu$ generation, (2) Askaryan emission, (3) ray tracing, and (4) RF channel simulation.  My analytic ray-tracing solution is pillar (3).  Our data for $n$ makes sense from a glaciological view: snow accumulates at the top of the ice sheet and is compressed over millenia \cite{Barwick:2018497}.  The snow steadily increases in density, which in turn decreases the speed of light.  Having a liberal arts mindset allowed me to find this solution.  While at Ohio State, I took a climate science course in order to understand snow compression, which lead me to try solutions that turned out to work.  The key finding of \cite{Barwick:2018497}, however, was RF \textit{horizontal propagation} not predicted by ray-tracing.  Horizontal propagation has been noted as an interesting detection scheme for UHE-$\nu$ \cite{10.1103/physrevd.71.011503}.
\\
\vspace{0.15cm}
The inaccuracy of ray-tracing solutions is compounded by the selection of \textit{phased arrays} as the main detection component of \textit{in situ} stations.  In this context, phased-arrays are vertical arrangements of identical antennas that work together.  One can think of received signal in terms of relative timing between antennas.  Imagine an ocean wave arriving at the shoreline, and each student in a line of equally-spaced students records when the wave reaches them.  From these times, they would know the angle at which the wave hits the shore.  A wave arriving straight on hits them all simultaneously, but an angled wave hits one student first, then another, and so on.  Similarly, the arrival direction of plane waves from the Askaryan effect can be deduced from the relative timing. However, this all assumes a constant index $n$, and we know it is not constant.  Knowing that sensitivity of IceCube Gen2 \textit{in situ} designs depends on the phased array precision, I have created a solution that leaves behind ray-tracing all together.

\subsubsection{Beyond Ray-Tracing: Open-Source Parallel FDTD Methods}
\label{sec:cem_cluster}

I have received two \textit{Summer Faculty Research Internship grants} from the Office of Naval Research (ONR).  My group focuses on phased-array radar development.  This branch of research is another example of the liberal arts mindset and developing a connection: phased-arrays are useful for UHE-$\nu$ science, and also for testing radar systems because they can mimic a radar reflection that moves.  I adapted a Python3 software packaged called MEEP, originally designed for $\mu$m wavelengths, to work for radio wavelengths \cite{10.1016/j.cpc.2009.11.008}.  From there, I developed phase-array models in which I could control things like the speed of light versus depth (i.e. the problem we face in IceCube Gen2).  I performed design studies, and the computed radiation properties matched phased-array antenna theory beautifully.  Intriguingly, the work represents the first time MEEP (which is open-source) had been applied to phased array design.  The results have been published in Electronics Journal \cite{electronics10040415}, where the work has been named among the Top 10 most notable articles of 2020-21\footnote{The notice of these awards is included in the supplemental material.}.
\\
\vspace{0.15cm}
I can envision a wide range of CEM applications for IceCube Gen2.  The phased array trigger for IceCube Gen2 must be designed accounting for the changing index of refraction.  \textit{I plan to combine a Python3 machine learning package with MEEP to optimize the phased array trigger given the index of refraction and Askaryan emission properties.}  These computations could be performed in parallel on a small dedicated cluster we would assemble.  We are searching for the right NSF grant to build this, and I have startup grant funding remaining that can be used for computer hardware.  If successful, the \textit{in situ} trigger would thus be trained to detect the smallest UHE-$\nu$ signals.  Theoretically, we expect a higher UHE-$\nu$ flux with smaller amplitudes, so in this way we would be maximizing the scientific output.
%\\
%\vspace{0.15cm}
%A CEM cluster would also serve the PUEO collaboration for another reason \cite{pueo}.  PUEO will seek UHE-$\nu$ signals with arrays of RF elements flown in the atmosphere.  PUEO antennas are the same type as the ones I designed for the Navy, and would be flown on a weather balloon gondola.  PUEO is therefore dubbed an \textit{in air} version of Askaryan-class detector.  PUEO faces a computational requirement that \textit{in situ} detectors do not: the signal must refract through the rough snow surface and into the air.  The Antarctic snow surface roughness has been measured \cite{doi:10.1142/S2251171717400025}, and it can degrade the signal.  With the measured roughness as initial conditions, I can calculate the predicted signal in PUEO arrays with near-to-far-field projection.  In my work published recently in Electronics Journal \cite{electronics10040415}, I produced this type of refraction, but with a smooth surface.  It would not be difficult to add the effect of the rough snow surface.

\subsubsection{Connection to Teaching and Academic Mentorship}

We always attempt to form connections between our research and teaching, and my work for the Navy has required me to teach.  My contacts are Dr. Christopher Clark, Dr. Eisa Osman, and Dr. Gary Yeakley, who work on active seeking radar.  Dr. Clark relayed a compliment I received from Dr. Yeakley, who had this to say about my 1-2 hour lectures given once per week in Summer 2020.
\begin{quote}
``One of the most stunning compliments Prof. Hanson received was from my Senior RF Engineer, Mr. Gary Yeakley (who has been working developing, designing, and testing radar since the 1970s), where Gary stated ``every week I learn new RF Physics from Jordan [Prof. Hanson].''
\end{quote}
I have included a letter of advocacy in the supplemental materials from Dr. Clark.  Researching a new subject, mastering it, and teaching it to colleagues in the Navy was a rewarding experience, and I was grateful to serve.  My Navy contacts invited me back for Summer 2021 to teach an RF Field Engineering Course.  In practice, this amounted to creating asynchronous tutorials videos servicemen download.  This content also applies to digital signal processing (DSP), my upcoming January term course.
\\
\vspace{0.15cm}
This past summer (2021), I involved a student named Adam Wildanger through a Fletcher-Jones Fellowship.  Adam is a great help, because he has taught me computer assisted design (CAD).  The process begins with me creating an antenna design using CEM tools.  Next, Adam ports my work as a machine-readable design.  Third, our Navy colleagues fabricate it using a 3D printer.  Finally, they have sent the fabricated components to me.  Adam and I have begun to test them in our lab using equipment provided by ONR.  The key is to use \textit{conductive} 3D printer material that radiates at high frequencies.  If successful, this collaborative effort has applications as diverse as UHE-$\nu$ research, radar development, 5G mobile communications, and remote sensing for climate science (\url{https://cresis.ku.edu/}).  A key lesson is that sometimes the mentor can learn new things from the student.  I am working on securing Adam his dream position at the national lab where my ONR colleagues are based (NSWC Corona).

\subsection{Mathematical Physics}
\label{sec:math_phys}

In 2015, I became a CCAPP Fellow at The Ohio State University, where I created an analytic Askaryan radiation model.  At the time, the simulation package for IceCube Gen2 (NuRadioMC) was under development.  The two \textit{in situ} groups, ARIANNA and ARA, relied on the MC codes ShelfMC and AraSim, respectively.  Both ShelfMC and AraSim were derived from the same legacy code.  We learned, however, that ShelfMC and AraSim did not always produce the same results, and were cumbersome to compare.  Further, the Askaryan models in both were 5-10 years old, and derived from \textit{semi-analytic} parameterizations.
\\
\vspace{0.15cm}
The results of foundational work on the Askaryan effect simulated every single sub-atomic particle created by the UHE-$\nu$ \cite{zhs}.  Overall signal strength was found to be proportional to the energy of the neutrino.  A distribution of radiated power was observed to radiate in a special direction: the Cherenkov angle.  Such models were called \textit{full-MC} models.  By tradition, physics simulations are sometimes called \textit{Monte Carlo} simulations (MC).  Semi-analytic parameterization models provide part of the electric field at the Cherenkov angle, and simulate the total electric charge created by the UHE-$\nu$ in ice.  Mixing these two results produces the electric field (radio wave) at a variety of angles \cite{PhysRevD.101.083005}.

\subsubsection{Frequency-Domain Model}

Missing from the 2018 ShelfMC/AraSim integrations was a common analytic understanding of Askaryan radiation.  I responded by developing a fully analytic model\footnote{In this sense, analytic means a set of equations, not a simulation.} that accounted for several important effects \cite{10.1016/j.astropartphys.2017.03.008}.  The independent variable in the equations is the frequency of the radio wave, so the model is classified as a \textit{frequency-domain} model.  I was inspired by work by Prof. John Ralston and Roman Buniy \cite{10.1103/physrevd.65.016003}.  Using simulations on the Ohio Supercomputing Cluster (OSC), we modeled the total electric charge created by the UHE-$\nu$.
\\
\vspace{0.15cm}
We then wrote a function that modeled the charge, and finished the ensuing electromagnetic calculations to obtain the radio wave.  My model produces template waveforms for UHE-$\nu$ searches with IceCube Gen2 \cite{10.1016/j.astropartphys.2017.03.008}.  The community began to use the model after I presented it at workshops at KICP (Univ. of Chicago), and TeVPA conferences.  Colleagues shared with me that a time-domain model at all angles relative to the Cherenkov angle would be highly useful.  In the final section of \cite{10.1016/j.astropartphys.2017.03.008}, we did provide an example, but only if the viewing angle equals the Cherenkov angle.

\subsubsection{Time-Domain Model}

There are four main advantages of analytic time-domain models. First, when they are matched to observed radio waveforms, UHE-$\nu$ cascade properties like total energy may be derived directly from waveform shapes. Second, evaluating a fully analytic model provides a speed advantage in software compared to other approaches.  Third, when analytic models are combined with RF antenna properties (derived using CEM), the resulting template can be embedded in detector firmare to form a filter that blocks noise and enhances the probability that a passing UHE-$\nu$ signal is detected.  Fourth, parameters in analytic models may be scaled to account for snow density in addition to ice density.  This application is useful for understanding potential signals in the firn.  My student, Raymond Hartig, and I are proud to announce that our work will be published in Physical Review D \cite{time}.  My vision for the future of this work involves three tracks.
\\
\vspace{0.15cm}
The first track involves UHE-$\nu$ template analysis.  Our simulation for IceCube Gen2, NuRadioMC, is broken into four pillars (steps).  Currently, UHE-$\nu$ are simulated first as \textit{events} (NuRadioMC pillar (1)) and the \textit{RF emissions} (Askaryan signals) are generated next (NuRadioMC pillar (2)).  Our ability to match simulated waveforms to potential UHE-$\nu$ waveforms from the detector is limited because we cannot scan through properties of the simulated \textit{cascades} of particles created by the UHE-$\nu$, only the UHE-$\nu$ with a single RF emissions model.  For example, two UHE-$\nu$ with the same energy could generate different cascades with different shapes of electric charge.  The effect of the cascade shape is important for the interpretation of future IceCube Gen2 data.  Conversely, if the effect of the cascade shape is well-understood, it becomes possible to measure the UHE-$\nu$ energy by templates to observed data \cite{time}.
\\
\vspace{0.15cm}
The second track involves embedding the model itself in detector firmware.  The detector cannot distinguish small signals from noise.  Noise and signal data trigger detectors and are saved.  We try to isolate UHE-$\nu$ signals in large data sets comprised mostly of radio noise once the data has been transmitted to the USA.  All data has to be shipped with limited bandwidth, and we cannot ship it continuously.  Embedding the model on the detector would allow the detector \textit{itself} to locate and flag priority events.  The physics community expects IceCube Gen2 to provide this type of alert system so that other physics and astronomy detectors could search for any UHE-$\nu$ sources we identify.  This is not possible if the data has to be shipped and then searched offline, because this takes too much time.
\\
\vspace{0.15cm}
The third track involves the connection between CEM and our Askaryan model.  In NuRadioMC the simulated signal is created by code in pillars (1)-(4) sequentially.  That is, the basic Askaryan model is mixed with detector response \textit{after} ray-tracing.  In reality, the radiation flows immediately from the cascade according to the details of the index of refraction of the ice.  It is a wave that \textit{generally} follows ray-tracing, but that reflects from internal ice layers, propagates horizontally, and can change shape.  All the effects \textit{not captured} by the smooth index of refraction function, $n$, will affect the signal.  Fully-analytic Askaryan models have a unique advantage: analytic equations can be implemented as MEEP sources, and MEEP \textit{can} account for all those effects.  The analytic model is in a unique position to provide advanced insight when combined with CEM.

\subsubsection{Connection to Teaching and Academic Mentorship}

Researching mathematical physics and radio waves has sharpened my teaching of electromagnetism (PHYS330) for obvious reasons.  I included detail about this experience, which was wonderful, in Sec. \ref{sec:teaching}.
\\
\vspace{0.15cm}
I have been mentoring an undergraduate double-major in physics and mathematics named Raymond Hartig who plans to attend graduate school.  Our partnership began in Spring 2020, when Raymond approached me for training in complex analysis.  These are the mathematical tools necessary to perform some electromagnetic calculations.  We then won the Fletcher-Jones Undergraduate Research Fellowship for that summer.  The experience of coaching him in his development has been rewarding, and I have taken note of a key skill project leaders must have.
\\
\vspace{0.15cm}
Similar to teaching courses, one has to think pedagogically when explaining projects to young researchers.  If they understand the direction at least as well as the syllabus of a course, they are more likely to succeed.  \textit{From the perspective of equity and inclusion}, my experience mentoring students pedagogically is useful.  By structuring the students' time, and explaining goals and productivity expectations in advance, students from diverse backgrounds feel called to participate at least as much as they do in class.  Sometimes research can make first-generation students, for example, feel that the work is not for them.  By providing pedagogical structure, I am signaling that \textit{they do belong} in my lab.

\subsection{Firmware, Software, and Hardware Development}

Askaryan-class detectors must operate autonomously due to the limited Antarctic infrastructure.  Stations must be powered sustainably with solar panels and wind turbines, and communications bandwidth is restricted to satellite modems and LTE networks \cite{10.1109/tns.2015.2468182} \cite{Aguilar_2021}.  Every sub-system that can operate autonomously is another sub-system that does not use bandwidth.  The ARIANNA stations, for example, send text messages via satellite modem to the server in the USA.  Configuration files are sent by the server to the stations with operational instructions.  This includes channel thresholds, which control the RF thermal trigger rate.  When signal or noise is more powerful than the threshold setting, the station is \textit{triggered} to record the data present in the channels.  Otherwise, the data disappears.  All radio and radar systems trigger on \textit{thermal noise} as well as signal.  A lower threshold increases the chance of hearing signals at the cost of recording more thermal noise.  To adjust thresholds in response to fluctuating thermal noise (which can fill up detector memory), we have to analyze the data between satellite messages, optimize thresholds, and send the stations new instructions.  This process, however, should be automated for hundreds of stations in IceCube Gen2.

\subsubsection{The Multi-Mode Frequency Counter (MMFC) and ARIANNA}

My student, John Paul G\'{o}mez-Reed and I developed firmware for the ARIANNA boards that would perform this automation.  This was a two-year process that began when we won the Keck Fellowship in the summer of 2018. First, we learned to design and load firmware into circuits.  John Paul named the system the Multi-Mode Frequency Counter (MMFC), because it is a digital counter that measured the rate at which particular ARIANNA channel was triggered by thermal noise.  We demonstrated it could measure channel trigger rates from from 10 hits per second to 10 million hits per second.  (When a channel is triggered millions of times per second, that is the effect of just noise and no interesting RF signal).  The digital input was provided by RF lab equipment I purchased for Whittier College using my startup grant. John Paul presented the results at the Southern California Conference for Undergraduate Research (SCCUR) \cite{sccur1}.  We then received ARIANNA systems from UC Irvine for systems integration.  \textit{In summer 2019, we won the Ondrasik-Groce Fellowsip.}  Throughout 2019 and early 2020, we used that fellowship to integrate the MMFC into the ARIANNA circuitry.  The circuit boards began to auto-adjust station thresholds.  We were in the process of final testing when the pandemic forced us to pause.  We did, however, present at SCCUR a second time \cite{sccur2}.

\subsubsection{Future Plans and Applications}

To continue this research, I have submitted a grant proposal to the Cottrell Scholars Program\footnote{This is included in the supplemental material.}.  The proposal outlines in detail the next three phases of this research, broken into concrete steps within each phase.  The overall goal is to enhance the trigger capability of our detectors with my analytic Askaryan models (Sec. \ref{sec:math_phys}).
\\
\vspace{0.15cm}
Phase 1 includes two main steps: (1) completing the integrated threshold automation firmware and software, and (2) completing the analytic Askaryan model.  As of this summer, both (1) and (2) are complete.  The integration of (1) into ARIANNA systems had to be put on hold due to the pandemic, but both the software and firmware are written.  The latest version of the Askaryan model is complete, and will be published in Physical Review D \cite{time}.  \textbf{Thus, phase 1 is already complete.}
\\
\vspace{0.15cm}
Phase 2 includes two main steps: (1) learning to match templates to data in firmware, and (2) demonstrating the system in an anechoic chamber\footnote{An anechoic chamber is a space that blocks all radio noise and reflections for the testing of sensitive RF equipment.}.  The reliability of this process must be proven before the firmware is deployed in Antarctica.  The key to step (2) is to calibrate the whole planned RF chain: a signal generator programmed with my analytic Askaryan model, transmitting and receiving antennas, amplifiers, filters, and digitization circuits.  I have performed similar processes as a post-doctoral fellow at KU \cite{10.1016/j.astropartphys.2014.09.002}.
\\
\vspace{0.15cm}
Phase 3 includes three steps: (1) publication of threshold automation and Askaryan model, (2) installation of the Askaryan-trigger firmware in detectors, and (3) field deployment.  It is wise to have the components of this project peer-reviewed and published before installing them in many detectors.  As of this summer, the Askaryan has passed peer review.  Once the IceCube Gen2 collaboration has a chance to review the firmware, we could move forward with deployment.

\subsubsection{Connection to Teaching and Academic Mentorship}

There are important connections to mentorship and teaching within this work.  First, John Paul G\'{o}mez-Reed is a Whittier local, and from a background under-represented in physics and engineering.  I was able to coach him through courses, application processes, and two SCCUR conferences.  We are currently working on graduate school and job applications.  Working with him has honed my mentorship skills, including when to hammer out details in the lab with my students, and when to leave them to figure it out on their own.  Working with a self-motivated student like John Paul required me to become attuned to that dynamic.
\\
\vspace{0.15cm}
I have taught two courses connected to this research: Computer Logic and Digital Circuit Design (PHYS306/COSC330), and Digital Signal Processing (COSC390), which introduced an interesting synergy.  I selected the Xilinx pynq-z1 digital logic education unit (\url{www.pynq.io}) for PHYS306.  The circuit board allows students to operate a Unix-based processing system (PS) integrated with a programmable logic (PL) firmware layer using Jupyter notebooks (Python3).  Jupyter notebooks are a tool for writing software and notes in a browser that works for all systems (Windows, Mac, Linux).  Learning how to use Jupyter notebooks boosted my CEM research, because MEEP work is often done in Jupyter.  Because this resarch is a DSP project, it can serve as an important unit in my DSP course.

\subsection{Open-source Antenna Design}

The MEEP-based phased array design technique has generated enthusiastic feedback.  Currently, my phased array design paper is ranked top 10 most notable works in Electronics Journal for 2020-2021 \cite{electronics10040415}\footnote{See supplemental material.}.  Using my second ONR grant, we are exploring the possibility of 3D printing phased arrays with conductive filament.  Additionally, UHE-$\nu$ collaborators are interested in validating antenna designs created with expensive, proprietary software against MEEP designs.  Cross-checks help us to assess systematic errors.  If we find similar results from both packages, we eliminate the need for the proprietary software, which reduces costs.  Comparisons of antenna modeling software are also found in Electronics Journal \cite{10.3390/electronics8121506}, which insipired the selection of that journal in addition to it being open-access for our students.  Using the aforementioned CEM cluster in Sec. \ref{sec:cem}, I could perform calculations that compare and optimize the antennas \textit{themselves}, in addition to refining our UHE-$\nu$ signal predictions.

\subsubsection{Connection to Teaching and Academic Mentorship}

Creating RF antennas requires laboratory skill.  I have been teaching several courses with significant laboratory components: Computer Logic and Digital Circuit Design (PHYS306/COSC330), and each year-long sequence of algebra-based and calculus-based introductory physics (PHYS135A/B, and PHYS150/PHYS180).  These courses have laboratory components taught in an integrated online/lecture/laboratory format.  I have also mentored students on occasion to work with our machine shop to build antennas, and to use the 3D printer.  Although it would be a stretch, I could envision including student-led RF antenna design projects in DSP, PHYS180, or PHYS135B.  The two latter courses are our introductory courses for electromagnetism.

\subsection{Drone Development and The Whittier Scholars Program}

A gap exists in Askaryan-based UHE-$\nu$ science.  Although we have made detailed measurements of the ice properties necessary to create our detectors \cite{10.3189/2015jog14j214} \cite{10.3189/2015jog15j057} \cite{barwick_besson_gorham_saltzberg_2005}, we do not scan these same properties for kilometers of distance across the arrays.  IceCube Gen2 radio will require a glaciological understanding of the ice across a $10 \times 10$ km$^2$ area.  Though CReSIS\footnote{Center for Remote Sensing of Ice Sheets.} measurements have been used to constrain these properties across Greenland \cite{10.1002/2015rs005849}, there is little CReSIS data at the South Pole.

\subsubsection{The Open Polar Server Data Gaps, and Drones}

The Open Polar Server (OPS) is a service provided by CReSIS.  Researchers may download radio sounding data from Greenland and Antarctica.  The radio sounding data are recorded from plane flights over the ice.  Radio sounding is like sonar in water, but the echo is a radio wave and the medium is ice.  There are three disadvantages to the flight data.  First, there may not be a flight near the detector.  Second, flights only give a snapshot of the ice at the time.  Third, the bandwidth of CReSIS radar does not always overlap with the proposed IceCube Gen2 bandwidths.
\\
\vspace{0.15cm}
Even if there is flight data available, it comes with a trade-off.  A plane flight covers hundreds of kilometers, but a plane might not return for years.  Conversely, a fixed station records data over time, but only at one location.  A dedicated drone could constrain the ice properties in both regimes.  In the machine shop and my RF design lab in the Science and Learning Center, a student and I constructed a 3D printed drone with $\approx 1$ kg payload.  Before the pandemic hit, we had plans to equip it with solar charging and cold-temperature components.  A similar effort is underway at CReSIS: Prof. Emily Arnold of the KU Dept. of Aerospace Engineering has begun an NSF CAREER grant to utilize RC military drones to study the Jakobshavn glacier in Greenland.  Unlike the off-the-shelf drones, our drone design can be 3D printed and assembled from commercial parts for $< \$1$k, but we need valuable insights from the CReSIS group on retro-fitting for cold temperatures.

\subsubsection{Connection to Academic Mentorship and the Whittier Scholars Program}

This project required me to mentor a driven engineering student named Nick Clarizio.  We worked well together, and the work reminded me of designing ARIANNA.  The student became my first physics double major (business, physics) to graduate as my advisee.  Once we completed the drone, Nick was able to demonstrate it for my PHYS150 class, as an example of balancing forces.  The drone has four motors, each with controllable thrust.  Thus, if the thrust is lowered in two motors and raised in two others, the drone will move in a certain direction.  Thus, my students received a hands-on demonstration of an important lecture topic.  In Spring 2021, I advised another student to graduatation in this area, who chose to focus on glaciology as part of his Whittier Scholars Program final project.
\\
\vspace{0.15cm}
I described my Whittier Scholars Program project with Nicolas Bakken-French earlier in this report (Sec. \ref{sec:origin}).  The basic idea was to perform research in climate science as part of my connections to Antarctic expeditions and other polar research programs.  The final project was part climate science, part cultural analysis of the role of glaciers in different areas, and part photographic essay focusing on past and current glacier structure and landscape.  The project falls into the \textit{Boyer} category of the scholarship of integration.  One facet not yet mentioned was that Nicolas was able to learn some programming and help me with glaciological analysis of CReSIS data from Moore's Bay Antarctica.  The work gave me a broader understanding of the ice near ARIANNA.  I also helped him earn an internship at UC Irvine with my colleagues there.  He helped them to develop a device that can melt a slot into a snowbank so that an RF antenna can be installed inside it.  This enables more rapid deployment of both glaciology and physics experiments involving RF sensors.  I had such a great experience working with Dr. Andrea Rehn and the WSP team that I have offered to serve on the Whittier Scholars Advisory Board.  My offer has been accepted and I will begin in Fall 2021.

\end{document}
