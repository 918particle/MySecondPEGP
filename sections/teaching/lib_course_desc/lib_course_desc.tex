\documentclass[../../../main.tex]{subfiles}
 
\begin{document}

The liberal education courses I have taught thus far at Whittier College are now introduced, with connections to the teaching philosophy described in Sec. \ref{sec:methods}.

\subsubsection{Safe Return Doubtful: History and Current Status of Modern Science in Antarctica}

In 1913, an advertisement appeared in a London newspaper that read ``Men wanted for hazardous journey, small wages, bitter cold, long months of complete darkness, constant danger, \textbf{safe return doubtful}, honor and recognition in case of success. - Ernest Shackleton, 4 Burlington St.''  Ernest Shackleton, who was to set several records in Antarctic exploration in the early 20th century, was calling for volunteers for the first expedition to cross Antarctica ocean-to-ocean.  There is a wonderfully rugged and inspiring batch of historical accounts of the modern exploration of Antarctica, from first landing to the triumphant arrival of humans to the South Pole for the first time.  This literature is included within the broader category of \textit{exploration literature.}
\\
\vspace{0.25cm}
I have been to Antarctica twice.  I recall that people who succeed on missions there share something in common, even though they are as diverse a group as any: they know themselves.  They understand their goals and their personal limitations.  People who travel to McMurdo Station, United States Antarctic Program (USAP) are scientists and engineers, explorers, pilots, mechanics, medics, and workers who help the base function.  They do not mind being uncomfortable, as long as they know they are serving the right purpose.  They work \textit{hard,} often for little recognition.  They train thoroughly, but can also improvise and survive.  Wouldn't it be a great thing if all our students encountered traits like these, and learned incorporated them in their studies through \textit{shared meaning}?
\\
\vspace{0.25cm}
To that end, I have created INTD255: \textit{Safe Return Doubtful: History and Current Status of Modern Science in Antarctica}, a \textit{CON2}-style course.  We include three main themes. First, we cover the history of the race to be the first humans to arrive at the South Pole.  Second, we cover current scientific projects in Antarctica.  Third, we write weekly journal entries that invite the students to look inside themselves and see their potential for true exploration.  Themes like inclusion arise as we learn that the first group to the South Pole took indigenous science seriously.  Led by the same captain who completed the Northwest Passage, this group incorporated methods of travel and survival from indigenous people in Northern Canada.  We build \textit{order} and \textit{shared meaning} (Sec. \ref{sec:teaching_philosophy}) between topics like physics, chemistry, biology, history, and indigenous technology.  The students engage with modules that cover scientific results and endeavors currentlty active in Antarctica, and historically.  Examples of journal activities invite students to describe a time they were in a dangerous situation and how they handled it, or a time they were called to lead.  I chose to conclude the course with a lecture on leadership \cite{west_point}.
\\
\vspace{0.25cm}
Much like my physics courses, I included active learning strategies to keep the students engaged with the science modules.  One example was an activity in which the students accessed ice-core data from a NOAA database and graphed measured oxygen isotope ratios from trapped air bubbles.  This is how we estimate global average temperatures over hundreds of thousands of years.  In another example, the students planned a navigation from Ross Island (the location of McMurdo Base, USAP) where Robert Falcon Scott first landed, to the Central Trans-Antarctic Mountains that lead to the South Pole.  The exercise requires students to plan how they would move both themselves and their food, assuming people and pack animals consume a certain number of calories per day.  These activities connect environmental and biological science with historical exploration.  Students must pay attention to detail in order to not starve or get lost, hypothetically.  Eventually, we applied the navigational technique of triangulation on a group hike to the summit of Hellman Park.  The students use a compass and the baseline between SLC and the summit in Hellman Park to calculate the distance to Downtown Los Angeles\footnote{Though we had a wide range of math preparation in the class, as a group we got the right answer to within 6\%, according to Google Maps.}.  Finally, I include lecture content about Antarctic wildlife, neutrino physics (see Sec. \ref{sec:scholarship}), and climate science through glaciological observations.
\\
\vspace{0.25cm}
For reading, we tackle ``The Last Place on Earth'' by Roland Huntford.  Like an expedition, I plan out the students' trek through this book, and we complete short weekly reading quizzes to keep everyone accountable.  The book is an expansive history of polar exploration, and I wish \textit{every single one} of the physicists in my field would read it for the contrasting examples of leadership styles.  The author paints portraits of the Norwegian Roald Amundsen, who led the first group to the South Pole, and Englishman Robert Falcon Scott, who arrived second but died on the return journey.  We see them planning years in advance, and failing on purpose just to learn what works.  We see scientists assuming their navigational calculations are correct, only to find out in the field they missed something and pay dearly.  We see explorers learning to eat strange food and move across terrain in ways used by indigenous Canadians.  An example: everyone trying to explore Antarctica kept developing scurvy.  Captain Scott and the British canned limes but the Vitamin C was rendered less potent in shipment.  Captain Amundsen noticed that the \textit{Netsilik} tribe never developed scurvy, despite having no access to fruit like the British, or berries like the Scandanavians.  Where is the nutrition above the Arctic Circle?  The answer is under the sea, so the Netsilik eat things that bring the vitamins from the sea: seals.  The Norwegians did not bother bringing limes or lingenberries, but instead ate seal meat in Antarctica.
\\
\vspace{0.25cm}
We also include the wonderful writings of Barry Lopez\footnote{Rest in peace, December 2020.}.  Specifically, we read excerpts from ``Horizon'' by Lopez.  One chapter deals with his travels to the side of Antarctica explored originally by Shackleton, Scott, and Amundsen.  Lopez describes the daily lives of scientists in Antarctica by chronicling a mission to retrieve meteorites from the outer solar system that land on the ice sheets of Antarctica.  In another chapter, Lopez describes riding south on a large ice-breaking research vessel to the side of Antarctica nearer to the Southern tip of Latin America.  Lopez includes writing about the life of indigenous people in that area, and the effect the land has on culture.  We also included ``News at the Ends of the Earth'' by Hester Blum, which is a collection of essays regarding the writings and printed work of the polar explorers.  It turns out that some polar explorers transported whole printing presses to Antarctica in order to create shipboard newspapers and other scientific chronicles.  The writing deals with topics like a changing climate, and the possibility of no remaining places on Earth unexplored.  I have to thank Profs. Michelle Chihara and Warren Hansen for suggesting these books as helpful additions.  Prof. Hansen audited my course, and we enjoyed discussing the reading together.

\end{document}
