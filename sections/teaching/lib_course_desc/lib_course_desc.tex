\documentclass[../../../main.tex]{subfiles}
 
\begin{document}

My liberal arts courses are introduced below, with connections to the teaching philosophy described in Sec. \ref{sec:methods}.

\subsubsection{Safe Return Doubtful: History and Current Status of Modern Science in Antarctica}

In 1913, an advertisement appeared in a London newspaper that read ``Men wanted for hazardous journey, small wages, bitter cold, long months of complete darkness, constant danger, \textbf{safe return doubtful}, honor and recognition in case of success. - Ernest Shackleton, 4 Burlington St.''  Ernest Shackleton, who was to set several records in Antarctic exploration in the early 20th century, was calling for volunteers for the first expedition to cross Antarctica ocean-to-ocean.  There is a wonderfully rugged and inspiring batch of historical accounts of the modern exploration of Antarctica, from first landing to the triumphant arrival of humans to the South Pole for the first time.  This literature is included within the broader category of \textit{exploration literature.}
\\
\vspace{0.15cm}
I have been to Antarctica twice, and I remember that people who succeed on missions there share something in common: they know themselves, their goals, and their personal limitations.  People who travel to McMurdo Station, United States Antarctic Program (USAP) are scientists and engineers, explorers, pilots, mechanics, medics, and base workers.  They do not mind being uncomfortable if they know they are serving the right purpose.  They work \textit{hard,} often for little recognition.  My hope is that our students encounter these traits and incorporate them in their studies through \textit{shared meaning} (see Sec. \ref{sec:teaching_philosophy}).
\\
\vspace{0.15cm}
To that end, I have created INTD255: \textit{Safe Return Doubtful: History and Current Status of Modern Science in Antarctica} (CON2).  We include three main themes. First, we cover the race to be the first humans to arrive at the South Pole in 1911.  Second, we cover current scientific projects in Antarctica.  Third, we write weekly journal entries that invite the students to look inside and see their potential for true exploration.  Themes like inclusion arise as we learn that the first group to the South Pole took indigenous science seriously.  Led by the same captain who completed the Northwest Passage, this group incorporated methods of travel and survival from indigenous people in Northern Canada.  We build \textit{order} and \textit{shared meaning} (Sec. \ref{sec:teaching_philosophy}) between topics like physics, chemistry, biology, history, and indigenous technology.  The students learn about current and historical scientific endeavors in Antarctica.  Examples of journal activities invite students to describe a time they were in a dangerous situation and how they handled it, or a time they were called to lead.  I conclude the course with a lecture on leadership \cite{west_point}.
\\
\vspace{0.15cm}
I include active learning strategies to keep the students engaged.  One example was an activity in which the students accessed ice-core data from a NOAA database to understand global temperature trends.  In another example, the students plan a navigation from Ross Island (the location of McMurdo Base, USAP) to the Central Trans-Antarctic Mountains that lead to the South Pole, including how they would move both themselves and their supplies assuming a certain number of calories per day.  These activities connect environmental and biological science with historical exploration.  Eventually, we apply the navigational technique of triangulation on a group hike to the summit of Hellman Park.  The students use a compass and the baseline between SLC and the summit in Hellman Park to calculate the distance to Downtown Los Angeles\footnote{As a group we got the right answer to within 6\%, according to Google Maps.}.  I include science lecture content about Antarctic wildlife, neutrino physics (see Sec. \ref{sec:scholarship}), and climate science.
\\
\vspace{0.15cm}
For reading, we tackle ``The Last Place on Earth'' by Roland Huntford, completing short, weekly reading quizzes like checkpoints.  The book is an expansive history of polar exploration, and I wish \textit{every single one} of the physicists in my field would read it for the contrasting examples of leadership styles.  The author paints portraits of the Norwegian Roald Amundsen, who led the first group to the South Pole, and Englishman Robert Falcon Scott, who arrived second but died on the return journey.  We see them planning years in advance, and failing on purpose just to learn what works.  We see scientists assuming their navigational calculations are correct, only to find out in the field they missed something and pay dearly.  We see explorers learning to eat strange food and move across terrain in ways used by indigenous Canadians.  For example, everyone trying to explore Antarctica kept developing scurvy.  Captain Amundsen noticed that the \textit{Netsilik} tribe never developed scurvy, despite having no citrus fruit.  It turns out the Netsilik eat things that bring the vitamins from the sea: seals.  The Norwegians did not bother bringing limes or lingenberries, but instead ate seal meat in Antarctica.
\\
\vspace{0.15cm}
We also include the wonderful writings of Barry Lopez from ``Horizon.''\footnote{Rest in peace, December 2020.}  One chapter deals with his travels to the side of Antarctica explored originally by Shackleton, Scott, and Amundsen.  Lopez describes the daily lives of scientists in Antarctica by chronicling a mission to retrieve meteorites from the outer solar system that land on the ice sheets.  In another chapter, Lopez describes riding south on a large ice-breaking research vessel to the side of Antarctica nearer to the Southern tip of Latin America.  Lopez writes about the life of indigenous people in that area, and the effect the land has on culture.  We also include ``News at the Ends of the Earth'' by Hester Blum, which is a collection of essays regarding the writings and printed work of the polar explorers.  Their writing deals with topics like a changing climate, and the possibility of no remaining places on Earth unexplored.  I am grateful to Profs. Michelle Chihara and Warren Hansen for suggesting these books as helpful additions.  Prof. Hansen audited my course, and we enjoyed discussing the reading together.

\subsubsection{A History of Science in Latin America}

As I shared in Sec. \ref{sec:intro}, I was inspired by my family and our students to create a new course entitled \textit{A History of Science in Latin America}.  Twenty-six students took this course in Spring 2021 (module 1).  The high enrollment was due in part to the \textit{CON2} and \textit{CUL3} designations.  I observed that the students' selection of final project topics often involved the country of origin of their ancestors or historical anecdotes from their families.  Thus, I drew the conclusion that the students were hungry for a course that covered science in Latin America.
\\
\vspace{0.15cm}
One historical idea the students encounter is the idea of central and peripheral scientific communities. Taking STEM courses alone might lead a student to believe that European and American cities have been central to scientific progress, and that Latin American communities have been peripheral. Peripheral, in this sense, refers to a community that merely adopts discoveries from the central ones and rarely produces progress. We find many examples is physics, astronomy, chemistry, and engineering/mining in which Latin American communities were central and their European counterparts were peripheral. A more accurate description of scientific progress for Latin America and Europe would be a full two-way exchange of knowledge.
\\
\vspace{0.15cm}
For example, two cures for syphilis were brought back to Europe by Spanish colonials: chinaberry bark and sarsaparilla root.  Even Spanish sources acknowledge that these treatments (which exploded commercially starting in the 1600s) were revealed to the colonials by indigenous doctors.  There is a similar story behind quinine production from \textit{cinchona} bark to treat malaria.  Spanish colonials would have been operating from the medical theory of the \textit{four humours,} and the Mexica and other Nahua peoples had a wholly separate spiritual and scientific worldview that involved the human body.  The people who produced a working treatment, whether they were colonial or indigenous, were simply those who had data from the native plants.
\\
\vspace{0.15cm}
The students also encountered the idea that mathematical systems developed all over Latin America in the pre-Columbian era.  We learned to translate numbers from our system of Arabic numerals to Mayan, Incan, and Aztec systems.  The students performed agricultural calculations using the numerical systems of the era\footnote{A sample assignment is included in the supporting material.}.  We branched into the astronomy of the solar system, because the 1600s-1700s saw rapid development in this branch of science worldwide.  Universities founded by Jesuits and Dominicans in what are now Columbia, Per\'{u}, and Venezuela held interesting debates about the structure of the solar system.  Meanwhile, the construction of technical colleges financed by private mining guilds and the viceroyalties was underway.  Native mining techniques proved more efficient than imported ones from Europe.  As the Scientific Revolution spread through Mex\'{i}co, Columbia, Per\'{u}, and R\'{i}o de La Plata, fascinating examples of science performed by indigenous citizens emerged.
\\
\vspace{0.15cm}
Thus, the students absorb a more complex and complete picture of the Scientific Revolution that demonstrates both Latin Americans and Europeans working towards the truth. Mexican scientists participated in the first measurements of the distance to the Sun using Venus transits.  By timing the transit of Venus across the face of the Sun, and comparing times from opposite sides of the planet, the distance to the Sun can be derived.  Measurements from Europe alone would have been insufficient.  Teams were dispatched from Britain and France to Russia, Tahiti, and Baja California.  Mexican scientists, rather than European visitors, completed the Baja measurements and the data helped establish the correct (and astonishing) value to within a few percent error.  Some of the same Mexican scientists found and explained the connection between sunspots and the aurora borealis.  The connection exists because the charged cosmic rays in the solar wind (Sec. \ref{sec:scholarship}) interact with the magnetic fields of the Earth and Sun.  This discovery was based on data from a Northern aurora that flared as far South as Mexico City and Zacatecas.  One Mexican scientist even recreated the aurora light \textit{in the lab} using charged particles striking gas.  Some European scientists adopted a theory that the aurora originated at the North Pole from inside the Earth.  Humans would not reach the North Pole for more than one hundred years.
\\
\vspace{0.15cm}
Taught for the first time online, this course featured group discussions in breakout rooms, video tutorials, warm-up activities, and asynchronous activities.  Supplemented by our excellent reading, \textit{Science in Latin America, A History}, edited by Juan Jos\'{e} Salda\~{n}a, we proceeded from the 1500s up to 1900.  This book appears to be one of only a few histories of Latin American science written in the last 10 years.  As long as I provide context from physics, mathematics, and chemistry, the students seem to follow it.  We would have liked to go further, but we were restricted due to the seven week limitation of the module system.  For final project design, I contacted Sonia Chaidez from Wardman Library to introduce Digital Storytelling to our students.  The students performed brilliantly, and we learned about everything from silver mining in Mexico to cosmic rays and Mayan mathematics.

\end{document}
