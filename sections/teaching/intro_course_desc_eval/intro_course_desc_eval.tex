\documentclass[../../main.tex]{subfiles}
 
\begin{document}

\begin{table}[ht]
\footnotesize
\centering
\begin{tabular}{| p{2cm} | p{12cm} |}
\hline
\hline
Question Number & Text \\ \hline
10 & This course had clear goals and objectives. \\ \hline
11 & This course was academically challenging. \\ \hline
12 & This course offered useful learning tools (such as lectures, discussions, readings, assignments and/or examinations). \\ \hline
13 & This course had grading criteria that were clearly identified. \\ \hline
14 & This course improved my understanding of the material. \\ \hline
15 & This course increased my interest in the subject matter. \\ \hline
16 & Overall, I would recommend this course to others. \\ \hline
17 & The professor used class time effectively and demonstrated preparation for class. \\ \hline
18 & The professor's teaching style and/or enthusiasm for the material strengthened my interest in the subject matter. \\ \hline
19 & The professor was able to explain complicated ideas. \\ \hline
20 & The professor challenged students to think critically and/or imaginatively about the course material. \\ \hline
21 & The professor provided clear and timely feedback. \\ \hline
22 & The professor encouraged meaningful class discussions. \\ \hline
23 & The professor was receptive to differing views. \\ \hline
24 & The professor was available for help outside of class. \\ \hline
25 & Overall, I would recommend this professor to others. \\ \hline
\hline
\end{tabular}
\caption{\label{tab:questions} The listing of standard course evaluation quesions.}
\end{table}

The course evaluations for introductory STEM courses described in Sec. \ref{sec:intro} are shared below.  On the original course evaluation forms, questions 10-16 pertain to the course, and questions 17-25 pertain to the professor.  Table \ref{tab:questions} lists the standard questions.  In the analysis below, I discuss response rates separately from the scores.  I have also chosen to include my summer session course, as the data reveals something interesting.

\subsubsection{Algebra-Based Physics}

\begin{table}
\footnotesize
\centering
\begin{tabular}{| c | c | c | c | c |}
\hline
\hline
Question & F2019 & S2020 & S2021 (1) & S2021 (2) \\ \hline
10 & 4.8 & 4.9 & 4.6 & 5.0 \\ \hline
11 & 4.7 & 4.9 & 4.8 & 4.9 \\ \hline
12 & 4.7 & 4.7 & 4.5 & 5.0 \\ \hline
13 & 4.8 & 4.8 & 4.5 & 5.0 \\ \hline
14 & 4.5 & 4.6 & 4.3 & 5.0 \\ \hline
15 & 4.0 & 4.2 & 4.1 & 4.9 \\ \hline
16 & 4.5 & 4.7 & 3.9 & 4.9 \\ \hline
\hline
\end{tabular}
\begin{tabular}{| c | c | c | c | c |}
\hline
\hline
Question & F2019 & S2020 & S2021 (1) & S2021 (2) \\ \hline
17 & 4.6 & 4.7 & 4.5 & 4.9 \\ \hline
18 & 4.6 & 4.4 & 4.4 & 4.9 \\ \hline
19 & 4.3 & 4.5 & 4.2 & 5.0 \\ \hline
20 & 4.6 & 4.7 & 4.4 & 5.0 \\ \hline
21 & 4.8 & 4.9 & 4.0 & 5.0 \\ \hline
22 & 4.6 & 4.6 & 4.4 & 5.0 \\ \hline
23 & 4.7 & 4.8 & 4.8 & 5.0 \\ \hline
24 & 4.8 & 4.8 & 4.9 & 5.0 \\ \hline
25 & 4.7 & 4.5 & 4.5 & 4.9 \\ \hline
\hline
\end{tabular}
\caption{\label{tab:eval_135} (Left) Course evaluation results for PHYS135, course questions.  (Right) Course evaluation results for PHYS135, professor questions.  F2019: Fall 2019, PHYS135A.  S2020: Spring 2020, PHYS135B.  S2021 (1 or 2): Spring 2020, PHYS135B sections 1 and 2.}
\end{table}

The course evaluation data for algebra-based physics is shown in Tab. \ref{tab:eval_135}.  Table \ref{tab:eval_135} (left) contains the results for the course, while Tab. \ref{tab:eval_135} (right) contains the results for the professor.  The question definitions are listed in Tab. \ref{tab:questions}.  The data cover four courses in this category from Fall 2019 through Spring 2021.  The courses taught in 2021 were taught through the module system, while those in 2019 and 2020 were done through semesters.  Fall courses (135A) cover units and vectors, kinematics, forces, energy, and momentum.  Spring courses (135B) cover electromagnetism topics like charge, fields, current, DC circuits, and magnetism.
\\
\vspace{0.25cm}
In general, the results are much higher than in my first years at Whittier College.  I attribute the successes to FPC recommendations and the hard work of my students.  I am pleased to find that my section of PHYS135B section 2 (module 2) received near perfect scores from those that responded, despite being taught in the module system.  The seven week module posed serious challenges for introductory physics students.  Those near perfect scores were earned when both students and professors were experiencing burnout.  Desite all the student success, several trends are visible in the algebra-based physics data that merit discussion.
\\
\vspace{0.25cm}
I have been watching Question 15 results since Fall 2017.  The question asks the student about increasing interest for physics.  One key point to understand about algebra-based physics comes from Question 9 (not shown in Tab. \ref{tab:eval_135}): ``I had a strong desire to take this course.''  Students regularly enter 3.0/5.0 for that question.  The online PEGP reports now provide data regarding why many students take the course, which is something I knew intuitively.  Students \textit{have to take it} for their major.  I know that about nine of ten students are either biology or KNS majors, and both programs require a year of physics if the student wants to be involved in medicine.  I have to inspire a genuine appreciation for physics in students who are being forced to take a course they do not want to take.  For this reason, I introduced article bonuses and self-designed projects, to reach the \textit{curiosity} learning focus, and the overall theme of \textit{shared meaning} from my teaching philosophy.  The average for all four instances of 135 is 4.3, which is higher than my last PEGP (2019), when I found 4.25.
\\
\vspace{0.25cm}
The second trend I noticed was one I was expecting.  Although most of the results in Tab. \ref{tab:eval_135} are high, the Spring 2021 section 1 (module 3) are lower on average than the others.  There are three reasons why this occurred, and all can be remedied.  This was my module 3 course that followed the module 2 PHYS135B (section 2).  I was blocked from seeing the course evaluations for module 2 before teaching module 3.  In module 2, we \textit{barely} reached the topic of applications of magnetism.  While teaching section 2 (module 2, 135B), I kept remembering my mantra to \textit{control the pace,} learned from prior FPC recommendations and experiences.  It turns out even the module 2 students were sharing that the pace was pretty quick, especially given that the course delivery was online and only seven weeks.
\\
\vspace{0.25cm}
Module 2 came to an end before we could address all the magnetism content, but the students that responded gave me high scores.  Although I planned to cover magnetism in more detail, there was just not enough time.  The module system and the pandemic took my magnets!  (Do you remember that phrase from the curricular discussions?  Don't take my magnets!  Good times.)  Ask any physicist: can your students learn electromagnetism in seven weeks, meeting five days per week?  The answer would likely be that \textit{human beings} cannot learn physics that fast.  So the first reason the module 3 scores were lower was this: I felt that it was an issue of integrity that I try to cover more magnetism for the module 3 students.  The module 3 students, however, were burned out, and some wrote exactly that in the written response section.  Had I been able to read the module 2 data before teaching module 3, I would not have tried to gain the extra week to cover different magnetic effects.
\\
\vspace{0.25cm}
The second reason the numbers are lower for that section had to do with grading.  Module 3 turned out to be my perfect storm of course work, committee service, finishing a research paper, advising thirty students, and caring for a one-year-old.  In retrospect, I should have exchanged my written midterm for something automatically graded using OpenStax Tutor.  When the students began to submit their midterms on Moodle, I realized I was not going to be able to finish grading them quickly.  Now that we are returning to the semester format, this problem will be solved both through better midterm design and having a longer semester.
\\
\vspace{0.25cm}
The third reason Spring 2021 135B section 1 scores were lower has to do with the topic of \textit{vectors}.  An example of a vector is force, which is both an amount and a direction ($150$ lbs. \textit{downward}).  To understand electromagnetism, one needs to be able to multiply vectors.  Knowledge of electric and magnetic forces, and the ensuing effects in circuits, motors, and generators relies on this knowledge.  In PHYS135B section 1 (module 3), I noticed students struggling more than usual with vectors.  Vectors are usually practiced in 135A, but I was not the intstructor in Fall 2020 for PHYS135A.  When I polled the students in PHYS135B in module 3 to learn if vectors were covered in 135A in Fall 2020, they uniformly said ``no.''  I recalibrated my course on-the-fly, reviewing vector content.  Had I known that vectors had been dropped, I would have planned for that.  This comming Fall 2021, however, I will be teaching all sections of 135A.  I will help the students practice more with vectors, and this should make 135B go more smoothly for them.

\subsubsection{Calculus-Based Physics}

\begin{table}
\footnotesize
\centering
\begin{tabular}{| c | c | c |}
\hline
\hline
Question & F2019 & S2020 \\ \hline
10 & 4.8 & 5.0 \\ \hline
11 & 4.9 & 5.0 \\ \hline
12 & 4.9 & 4.9 \\ \hline
13 & 4.9 & 4.9 \\ \hline
14 & 4.7 & 4.9 \\ \hline
15 & 4.3 & 4.7 \\ \hline
16 & 4.8 & 4.9 \\ \hline
\hline
\end{tabular}
\begin{tabular}{| c | c | c |}
\hline
\hline
Question & F2019 & S2020 \\ \hline
17 & 4.9 & 5.0 \\ \hline
18 & 4.7 & 5.0 \\ \hline
19 & 4.6 & 4.7 \\ \hline
20 & 4.8 & 5.0 \\ \hline
21 & 4.8 & 4.9 \\ \hline
22 & 4.7 & 4.9 \\ \hline
23 & 4.8 & 5.0 \\ \hline
24 & 5.0 & 5.0 \\ \hline
25 & 4.8 & 5.0 \\ \hline
\hline
\end{tabular}
\caption{\label{tab:eval_150_180} (Left) Course evaluation results for PHYS150/180, course questions.  (Right) Course evaluation results for PHYS150/180, professor questions.  F2019: Fall 2019, PHYS150.  S2020: Spring 2020, PHYS180.}
\end{table}

The data in Tab. \ref{tab:eval_150_180} pertains to two sections of calculus-based physics.  I taught PHYs150 in Fall 2019, and PHYS180 in Spring 2020.  Like algebra-based physics, mechanics is covered in the first course and electromagnetism is covered in the second.  Upon examining the numbers, I do not identify any significant downward trends relative to my supplemental PEGP from 2019.  The scores from PHYS180 in the supplemental PEGP were near perfect, but the class size was $N = 8$, rather than $N = 26$ (PHYS150) and $N = 24$ (PHYS180) for this round.  The exception to these remarks is the usual Question 15 (PHYS150), but that is not unexpected from a large sample of students.  Some students take PHYS150 thinking they want to major in physics or 3-2 engineering, for example, and switch majors when they realize they are more interested in applications like ICS/Math.  For example, I had one advisee this year switch from physics to Digital Art and Design.
\\
\vspace{0.25cm}
After reading through the written assessments, I noticed several comments for which I can provide a remedy.  First, some students mention problems with \textit{ExpertTA}, and I have moved to OpenStax Tutor.  Some students mentioned that three shorter midterms worked well for them, and others mentioned that the third one falls too close to finals season.  I have since switched to two midterms designed to take 1 hour each.  In algebra-based physics, I am moving to one midterm, and the final project design is due when the second midterm would have occurred.  In both algebra and calculus-based physics, the final project is still presented towards the end of the class, and the final exam is optional.  Most of the other comments were positive, especially for PHYS180 during the rapid transition to remote learning.  One student suggested keeping the tutorial videos going even after returning from quarantine.  I am prepared to do this for electromagnetism, but will have to create new videos for mechanics (PHYS135A and PHYS150).

\subsubsection{Elementary Statistics}

\begin{table}
\footnotesize
\centering
\begin{tabular}{| p{6cm} | p{1cm} |}
\hline
\hline
Question & Result \\ \hline
This course had clear and objective outcomes. & 4.8 \\ \hline
This course was academically challenging. & 4.5 \\ \hline
This course offers useful learning tools. & 4.8 \\ \hline
This course had grading criteria that were clearly identified. & 4.8 \\ \hline
This course improved my understanding of the material. & 4.8 \\ \hline
This course increased my interest in the subject matter. & 4.5 \\ \hline
This course provided interactions between students that were meaningful. & 4.5 \\ \hline
This course is as rigorous as the typical on-campus course. & 4.3 \\ \hline
Overall, I would recommend this course to others. & 4.8 \\ \hline
\hline
\end{tabular}
\begin{tabular}{| p{6cm} | p{1cm} |}
\hline
\hline
Question & Result \\ \hline
The professor demonstrated preparation for the class. & 4.8 \\ \hline
The professor's teaching style and/or enthusiasm for the material strengthened my interest in the subject matter. & 4.5 \\ \hline
The professor was able to explain complicated ideas. & 4.3 \\ \hline
The professor challenged students to think critically. & 4.8 \\ \hline
The professor provided clear and timely feedback. & 4.5 \\ \hline
The professor encouraged meaningful discussions. & 4.5 \\ \hline
The professor was receptive to differing views. & 4.8 \\ \hline
The professor was available to help. & 4.8 \\ \hline
Overall, I would recommend this professor to others. & 4.8 \\ \hline
\hline
\end{tabular}
\caption{\label{tab:eval_080} (Top) Course evaluation results for MATH080, course questions.  (Bottom) Course evaluation results for MATH080, professor questions. Su2020: Summer 2020.  Note: the questions for summer online courses are slightly different than courses during the academic year.  See text for details.}
\end{table}

Teaching elementary statistics (MATH080) in Summer 2020 was a useful experience, and I really enjoyed working with my students.  I thought it would be worth while to point out the the numbers in Tab. \ref{tab:eval_080} are not that different from Tabs. \ref{tab:eval_135} and \ref{tab:eval_150_180}.  Before the pandemic, online math courses were not allowed by Math Department policy.  I was likely the first Whittier College professor to ever teach one.  I am grateful to the Math Department for trusting me with the responsibility of teaching MATH080.  I taught this course in the same way I teach PHYS135A/B: reading assessment warm-up, TT and PER modules, and PHeT simulations.  I included tutorial videos and a student-designed final project.  Before we agreed to teach the course in the Summer, there were concerns from math professors that \textit{quality control} is what stops us from teaching math online.  Similar to algebra-based physics, students are taking a must-pass course for graduation, but need help sharpening math skills.  Instructors are under pressure to reduce content and slow the pace.  I found that the students learned the material and applied it successfully in their final projects\footnote{Examples in supporting material.}.  Of the $N = 4$ students that responded, they shared that the course was \textit{almost} as rigorous as the semester-long version (4.3/5.0).  However, they also report that the course had useful learning tools, that it increased their interest, and that it improved their understanding.

\end{document}
