\documentclass[../../main.tex]{subfiles}

\begin{document}

The course evaluations for advanced STEM courses described in Sec. \ref{sec:adv} are shared below.

\subsubsection{Computer Logic and Digital Circuit Design}

\begin{table}
\footnotesize
\centering
\begin{tabular}{| c | c |}
\hline
\hline
Question & S2020 \\ \hline
10 & 4.8 \\ \hline
11 & 4.8 \\ \hline
12 & 4.7 \\ \hline
13 & 4.7 \\ \hline
14 & 4.6 \\ \hline
15 & 4.8 \\ \hline
16 & 4.8 \\ \hline
\hline
\end{tabular}
\begin{tabular}{| c | c |}
\hline
\hline
Question & S2020 \\ \hline
17 & 4.7 \\ \hline
18 & 4.8 \\ \hline
19 & 4.6 \\ \hline
20 & 4.8 \\ \hline
21 & 4.8 \\ \hline
22 & 4.4 \\ \hline
23 & 4.7 \\ \hline
24 & 4.8 \\ \hline
25 & 4.6 \\ \hline
\hline
\end{tabular}
\caption{\label{tab:eval_330_306} (Left) Course evaluation results for COSC330/PHYS306 pertaining to the course.  (Right) Course evaluation results for COSC330/PHYS306 pertaining to the professor.}
\end{table}

The course evaluation results for COSC330/PHYS306 are shown in Tab. \ref{tab:eval_330_306}.  The results are very good, in general.  I would have liked a higher result for Question 25, but this simply prompted me to look through the written responses for clues.  This course took place during the rapid transition to online teaching.  As we examine the written responses, we see that the students were inspired by the laboratory portion of the course.  They were disappointed, though, that what they see as vital skills (lab training, circuit boards, python programming) were diminished by the pandemic.  After the transition to online learning, I created a solution that enabled the students to perform some of the digital logic labs on the PYNQ-Z1 boards \textit{from their homes.}
\\
\vspace{0.25cm}
Figure \ref{fig:pynq} (left) contains a diagram showing how one can control the PYNQ-Z1 board via a laptop running Jupyter notebooks.  Jupyter notebooks allow a student to run python and other programming languages regardless of the operating system (platform independent).  I realized as long as they could control the laptops, they could control the boards, provided the boards were hooked together correctly.  I installed TeamViewer on the laboratory computers.  TeamViewer is an open-access remote-desktop solution, and I passed the access codes to the students.  I tested the system by logging in to a pynq board from my laptop through my home router and running a lab example.  Once that was working, I repeated the procedure with the students observing via Zoom.  Eventually, the students were logging into laptops in the lap, and controlling the circuit boards from there.  One student said this made her feel like a hacker, in a good way.
\\
\vspace{0.25cm}
We see in the written responses that the students appreciated all my effort to give them back some lab activities remotely.  The students suggest more time, in fact, for labs.  This is good to know for Fall 2021 because we are offering the course again.  Normally it is a Spring course, but the math professors and I realized we did not want COSC330/PHYS306 overlapping with other 300-level COSC and PHYS electives.  We now have 14 students enrolled, and I will expand the time these students can work on the labs.  Finally, teaching in person will allow the students to create final projects using the PYNQ-Z1 boards.  I anticipate HDMI video, audio processing, and analog-to-digital conversion (ADC) projects.  Hopefully that will satisfy the students' curiosity and meet the suggestions they shared in course evaluations.

\subsubsection{Digital Signal Processing}

\begin{table}
\footnotesize
\centering
\begin{tabular}{| c | c |}
\hline
\hline
Question & Jan 2019 \\ \hline
10 & 4.75 \\ \hline
11 & 4.75 \\ \hline
12 & 4.75 \\ \hline
13 & 4.75 \\ \hline
14 & 4.75 \\ \hline
15 & 4.6 \\ \hline
16 & 4.5 \\ \hline
\hline
\end{tabular}
\begin{tabular}{| c | c |}
\hline
\hline
Question & Jan 2019 \\ \hline
17 & 4.75 \\ \hline
18 & 4.75 \\ \hline
19 & 4.5 \\ \hline
20 & 4.6 \\ \hline
21 & 4.9 \\ \hline
22 & 4.1 \\ \hline
23 & 4.2 \\ \hline
24 & 5.0 \\ \hline
25 & 4.75 \\ \hline
\hline
\end{tabular}
\caption{\label{tab:eval_dsp} (Left) Course evaluation results for COSC390 pertaining to the course.  (Right) Course evaluation results for COSC390 pertaining to the professor.}
\end{table}

The results for DSP (COSC390) are shown in Tab. \ref{tab:eval_dsp}.  Note that this course number may now correspond to another course, because this course was offered as a special topic within computer science.  Technically, these evaluation data were included in my last PEGP report submitted in Fall 2019 after I taught the course in January 2019.  I include them here because I will be teaching this course once again in the upcoming January term.  The primary feedback the students shared was that the material is really interesting, but that it should be spread over the course of a semester.  Although the students shared that feedback, the results in Tab. \ref{tab:eval_dsp} are generally very good.
\\
\vspace{0.25cm}
Questions 22 and 23 pertain to encouragement of class discussions and accepting different viewpoints.  Although my students know that I am always accepting of different viewpoints, we do not really hold group discussions in advanced STEM courses as the quesions suggest.  What we do have is lab partners. For COSC330/PHYS306, the partners work together on one system (laptop plus PYNQ-Z1) to solve the problems in the laboratory activity.  So in that sense, I do encourage them to discuss the issues within the labs, but with each other.  For the January term version of DSP, we could have partners work on extended examples in the same manner as the COSC330/PHYS306 students.  When I taught this course in January 2019, there were seven students.  Thus, three pairs plus myself working with the additional student may have helped.  I will try this out in the upcoming January term.

\subsubsection{Electromagnetic Theory}

\begin{table}
\footnotesize
\centering
\begin{tabular}{| c | c |}
\hline
\hline
Question & F2020 \\ \hline
10 & 4.9 \\ \hline
11 & 5.0 \\ \hline
12 & 4.9 \\ \hline
13 & 4.9 \\ \hline
14 & 4.6 \\ \hline
15 & 4.6 \\ \hline
16 & 4.4 \\ \hline
\hline
\end{tabular}
\begin{tabular}{| c | c |}
\hline
\hline
Question & F2020 \\ \hline
17 & 4.9 \\ \hline
18 & 4.8 \\ \hline
19 & 4.6 \\ \hline
20 & 4.6 \\ \hline
21 & 4.5 \\ \hline
22 & 4.7 \\ \hline
23 & 4.7 \\ \hline
24 & 4.9 \\ \hline
25 & 4.9 \\ \hline
\hline
\end{tabular}
\caption{\label{tab:eval_330} (Left) Course evaluation results for PHYS330 pertaining to the course.  (Right) Course evaluation results for PHYS330 pertaining to the professor.}
\end{table}

The course evaluation results for PHYS330 are shown in Tab. \ref{tab:eval_330}.  In general, the results are good.  I was aiming for a higher mark on Question 16 (recommend this course to others), which prompted me to search through the written responses for clues. Several students shared that this course is not really appropriate for seven weeks.  I gave examples of how this course is normally distributed through the weeks of a semester in Sec. \ref{sec:adv}.  There were other suggestions, though, that were interesting and useful for next time.  One student related that if there are going to be videos, they would be better if they included traditional teaching in addition to example problems.  Another student suggested that in an online advanced course, warm-ups should be shortened, or done as a group.
\\
\vspace{0.25cm}
Most of my reflection on my teaching lately has been geared towards the introductory physics courses.  Now we see that the students in advanced physics courses have slightly different preferences.  These issues are made easier in a semester system, and I will take them into account the next time I teach PHYS330.  More traditional teaching, and perhaps more CEM, will satisfy the students.  If I notice struggling students, though, I will still be prepard to use some PER modules.  When asked if they would change anything, the students mostly remarked that two modules, or more time, would have been better.  Finally, one student gave me a boost when he wrote: ``I miss the physics boys.  This was nice.''  I miss you too bro.

\end{document}
