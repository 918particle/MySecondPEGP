\documentclass[../../../main.tex]{subfiles}
 
\begin{document}

\begin{table}
\footnotesize
\centering
\begin{tabular}{| c | c |}
\hline
\hline
Question & F2020 \\ \hline
10 & 5.0 \\ \hline
11 & 4.0 \\ \hline
12 & 4.9 \\ \hline
13 & 4.9 \\ \hline
14 & 4.9 \\ \hline
15 & 4.6 \\ \hline
16 & 4.8 \\ \hline
\hline
\end{tabular}
\begin{tabular}{| c | c |}
\hline
\hline
Question & F2020 \\ \hline
17 & 5.0 \\ \hline
18 & 4.9 \\ \hline
19 & 4.9 \\ \hline
20 & 5.0 \\ \hline
21 & 4.9 \\ \hline
22 & 5.0 \\ \hline
23 & 4.7 \\ \hline
24 & 5.0 \\ \hline
25 & 5.0 \\ \hline
\hline
\end{tabular}
\caption{\label{tab:eval_100} (Left) Course evaluation results for INTD100 pertaining to the course.  (Right) Course evaluation results for INTD100 pertaining to the professor.}
\end{table}

Table \ref{tab:eval_100} contains the course evaluation results for my section of INTD100: College Writing Seminar.  I taught this course in the Fall of 2020.  I taught fifteen students who were also my first-year advisees.  The scores are in general high, and I analyze the written remarks below.
\\
\vspace{0.25cm}
Regarding the written evaluations, one student suggested more breakout rooms, and another added to put them more toward the middle of the class session.  This is actually the first time I have seen any student ask for more breakout rooms, for in my other courses the students preferred less usage of them.  One rough patch in my teaching for all my courses as been Moodle and grades.  I have always calculated my grades in spreadsheet programs, and updated the students on their progress.  I do that because in my STEM courses, a student could be doing well on homework (working with friends) and then do poorly on a midterm, only to recover on the final exam.  The projected grade in that case only stresses the student and may not predict well the final outcome.  Some students wrote about more clarity in the grading system and posting projected grades to Moodle.  I will think about how to best honor that type of request.
\\
\vspace{0.25cm}
When asked what they would change about the course, many students wrote that they liked it the way it was.  Some gave the common remarks that more asynchronous days is preferable, more time in the module or class, and to start the class after 8 am.  Some students shared that they were unsure about specifics regarding assignments.  I agree that my assignment descriptions were sometimes more open-ended because I am inexperienced at assigning and grading writing.  For shorter assignments, I would tell them exactly what to do.  For longer assignments, I could reflect more on how to create an essay prompt that structures the assignment without telling them exactly what to write.  I chose a shorter final essay length after speaking with Charles Eastman about it, when he cautioned that some students have not written long pieces.  In such a short module, I am happy with the results the students produced.

\end{document}
