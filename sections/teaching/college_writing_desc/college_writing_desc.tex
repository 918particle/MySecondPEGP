\documentclass[../../../main.tex]{subfiles}
 
\begin{document}

In Fall 2021, I taught my first section of INTD100: College Writing Seminar.  Teaching INTD100 felt the most different from any course I have taught at Whittier College.  I imagine describing my section of INTD100 will sound wrong or different to a professional writing or English instructor.  To better conform with what has been done in the past in INTD100, I met with Charles Eastman during Summer 2020.  Charlie gave me helpful advice regarding the recommended length of assignments, and how to assess writing.  The theme of the course was scientific and technical writing, with emphasis on clarity, conciseness, passive voice, and the use of carefully planned structure for larger tracts of writing.
\\
\vspace{0.15cm}
Although the course was focused on technical writing, we began with summer reading from \textit{Notes of a Native Son} by James Baldwin.  Given the events of Summer 2020, I had to incorporate \textit{something} that spoke to the need for social justice.  My family watched \textit{I Am Not Your Negro}, directed by Raoul Peck, which focused on Balwin's writings.  I decided to actually read the book.  Balwin's contrasting descriptions of burroughs around New York City resonated with me, especially his writing about Harlem.  The reason has to do with where my family and many of our students live.  I asked my advisees (who were also my students in INTD100) to get the book and read a particular essay.  I asked them to do five things: 1) practice summarizing the events of the essay in 120 words, 2) again, but using only 20 words, 3) to discuss an extended metaphor the author uses to describe the gut feeling of experiencing racism, segregation, and poverty in a crowded city, 4) to analyze the use of tangible evidence regarding a riot that took place in Harlem, and finally 5) a bonus assignment to draw links between the first chapter and a different chapter about the author's experience writing for newspapers in Harlem.
\\
\vspace{0.15cm}
The students completed this assignment reasonably well, but remained virtually silent during planned time for group discussions.  We had some discussion in breakout rooms, and icebreaker activities helped.  After icebreakers and initial discussions about the summer reading, we moved forward with the main syllabus.  Our first topic was making writing more concise and clear.  I do grade papers occasionally in my STEM and liberal arts courses, and I find that students use more words than necessary.  I created videos in which I edit example paragraphs of scientific writing.  The students see how the writing becomes more precise by eliminating unnecessary words.  I showed them an open-access visual tool for creating two-dimensional outlines called \verb+coggle.it+.  The students began reading articles about the discovery of things like gravitational waves, and creating the outline of the article using \verb+coggle.it+.  In this way, they learn how technical writing is structured.
\\
\vspace{0.15cm}
Having built the main ideas of conciseness and outline structure, we developed skills like organizing details into proper hierarchy (general to specific).  The students learned how to select their own scientific articles and tracts of writing for practice and analysis.  I added videos about these sub-skills.  In small groups, the students began to generate summaries of scientific topics like COVID19 outbreaks, gravity waves, the first picture of a black hole, and neutrino physics, using the skills they had learned.  Although I gave them examples of sources, the students began to locate their own.  One group chose to describe the purpose of the IceCube neutrino detector (Sec. \ref{sec:scholarship}), and absolutely nailed it.  They demonstrated clarity, conciseness, and correctly cited sources while accurately summarizing the topic.  The work of the other groups was also good, and it appeared the students were succeeding\footnote{Writing examples included in the supporting materials.}.  Finally, we read an essay on leadership (the same one I assigned in INTD255), and I asked the students to write an essay about a time they had to be a leader.
\\
\vspace{0.15cm}
Following advice from Charles Eastman, subsequent class sessions were spent editing example tracts of the students' writing.  I remember from my conversation with him to \textit{let the students' writing drive the discussion.}  We concluded our module with technical description and specific writing structures like laboratory reports.  The students applied these skills to a final 1000-word essay on a scientific topic encountered in the course.  Their writing included diagrams, citations, summarizing content, and was writing with conciseness and clarity.  We concluded the course by playing games together online, since these students were also my first-year advisees.

\end{document}
