\documentclass[../../../main.tex]{subfiles}
 
\begin{document}

The course evaluations for liberal arts courses described in Sec. \ref{sec:lib_eval} are shared below.

\subsubsection{Safe Return Doubtful: History and Current Status of Modern Science in Antarctica}

\begin{table}
\footnotesize
\centering
\begin{tabular}{| c | c |}
\hline
\hline
Question & F2019 \\ \hline
10 & 4.8 \\ \hline
11 & 4.7 \\ \hline
12 & 4.7 \\ \hline
13 & 4.7 \\ \hline
14 & 4.8 \\ \hline
15 & 4.4 \\ \hline
16 & 4.6 \\ \hline
\hline
\end{tabular}
\begin{tabular}{| c | c |}
\hline
\hline
Question & F2019 \\ \hline
17 & 4.9 \\ \hline
18 & 4.8 \\ \hline
19 & 4.7 \\ \hline
20 & 4.8 \\ \hline
21 & 4.9 \\ \hline
22 & 4.8 \\ \hline
23 & 4.9 \\ \hline
24 & 4.9 \\ \hline
25 & 4.8 \\ \hline
\hline
\end{tabular}
\caption{\label{tab:eval_255} (Left) Course evaluation results for INTD255 pertaining to the course.  (Right) Course evaluation results for INTD255 pertaining to the professor.}
\end{table}

The course evaluation results for \textit{Safe Return Doubtful: History and Current Status of Modern Science in Antarctica} are shown in Tab. \ref{tab:eval_255}.  This course represented the first time I have taught anything outside the STEM area.  Although I included content from my research, and some of the other topics were familiar, I knew I was teaching outside of my comfort zone.  I also knew from conversations with the students early in the course that any physics or math I included had to be kept to simple.
\\
\vspace{0.25cm}
The numerical results in Tab. \ref{tab:eval_255} are generally high.  I begin by reflecting on the comments of the students who seemed to like the course.  One student remarked that ``This course was by far the most interesting course I have taken at Whittier College ... it taught life skills while also teaching about the physics of movement and energy as it pertains to humans.  It also was very introspective and made the students think much about their own lives and experiences and relate them to the course.''  This comment is uplifting because I was aiming to teach an interesting course that involved science, history, and life skills.  The life skills upon which we reflected as a class are about knowing one's full capabilities and limitations, and one's ability to lead.  Another student remarked that ``your teaching style was great; I normally do not relate to as much information given in lecture format, but I did in this course ... ''  That is also nice to hear, because sometimes I felt like a square peg teaching a round hole.  \textit{Are any of the students appreciating this?  Am I doing it right?}  Apparently, the students thought I was doing it right.  Another student summarized their experience with ``I am not very good at science or math but you made it very easy for me to understand.''  This last comment is important to remember, given that the students shared a variety of preferences for the math detail included.
\\
\vspace{0.25cm}
Another student wrote: ``I thought that Professor Hanson was a very good enthusiastic professor. He always listened to what his students had to say and tried to help them in the best way possible.'' By Fall 2019, I had learned to become attuned to the specific set of students in a class.  I included the physics of friction (a simple linear formula) in the course, and I demonstrated how the formula is used.  The effects of friction, force, and work controlled the race to the South Pole, and it was interesting how the \textit{Netsilik} solved the problem in Canada.  But another student was right to say that ``... this class has no prerequisites, don't assume that everyone has taken a physics class before.''  About one week into the class one student told me in office hours that he didn't expect to see a physics equation and that he was worried.  I am always surprised to see such fear of a simple linear equation.  However, I have learned by now how to guide students first to a sense of \textit{order}, and then \textit{shared meaning} (see Sec. \ref{sec:teaching_philosophy}).
\\
\vspace{0.25cm}
The students were reassured and later wrote wonderful essays and journal entries.  I understand how students feel when they have to do a bit of math after we do not require them to do any math for several years.  I was paying attention to the perceived difficulty level, and for most students I think the balance worked.  For one student, any math at all would have been too much: ``The (professor) assumed I would know how to do the math and didn't explain how to go through the equations very well.''  Well, I did give the students a review unit at the beginning to slowly prepare them to apply numbers.  I would increase and simplify the review next time, while not increasing the technical content, which is necessary to understand what is happening in \textit{The Last Place on Earth.}  A final remark from a student summarizes how the students and I both felt: ``This course was great because it was untraditional and very insightful.  We learned physics through stories of exploration and I believe this method made students much more interested and involved.''  Finally, I must have been doing something right, because Prof. Warren Hansen attended all of my class sessions and told me I did a great job.  He even recommended more exploration literature from which to draw next time.

\subsubsection{A History of Science in Latin America}

\begin{table}
\footnotesize
\centering
\begin{tabular}{| c | c |}
\hline
\hline
Question & S2021 \\ \hline
10 & 4.8 \\ \hline
11 & 4.8 \\ \hline
12 & 4.8 \\ \hline
13 & 4.6 \\ \hline
14 & 4.8 \\ \hline
15 & 4.6 \\ \hline
16 & 4.6 \\ \hline
\hline
\end{tabular}
\begin{tabular}{| c | c |}
\hline
\hline
Question & S2021 \\ \hline
17 & 4.8 \\ \hline
18 & 4.8 \\ \hline
19 & 4.8 \\ \hline
20 & 5.0 \\ \hline
21 & 4.4 \\ \hline
22 & 4.7 \\ \hline
23 & 5.0 \\ \hline
24 & 5.0 \\ \hline
25 & 4.8 \\ \hline
\hline
\end{tabular}
\caption{\label{tab:eval_290} (Left) Course evaluation results for INTD290 pertaining to the course.  (Right) Course evaluation results for INTD290 pertaining to the professor.}
\end{table}

The course evaluation results for \textit{A History of Science in Latin America} are shown in Tab. \ref{tab:eval_290}.  Even though this course had to be taught in the module system, the students learned a great deal and seemed to appreciate the course theme.  Their final projects indicated that they understood the connection between the Scientific Revolution and Latin America, and that science in Latin America has always had pre-Columbian roots.  The numbers in Tab. \ref{tab:eval_290} are high on average, and I include some further analysis of the students' written remarks below.  Most of the students shared very positive comments about the general topic and style of the course.  One wrote ``You did an fantastic job making the course material engaging. It’s difficult to do over zoom, and a lot of professors struggle with it ...'' Another wrote: ``I really loved this class ... While meeting every day at 8 am seems like it would be a drag, I actually felt that meeting so frequently was the key to how much I learned in this class ... I also could tell how passionate the professor was about both the material and teaching his students, which made me want to show up and do well in the class.''
\\
\vspace{0.25cm}
This was the first time I had taught outside the STEM area via remote instruction.  I am not surprised that students shared a variety of technical comments about course structure, because at this point in the calendar I was really feeling the burn of the modules and balancing teaching and family.  The students understood, and it turned out some students in the course were also caring for children and younger siblings.  Some students wrote the pace was rushed or slow, and shared remarks similar to INTD255 about mathematics content.  Part of this course, however, required arithmetic practice so the students could engage with \textit{ethnomathematics}, the idea that pre-Columbian civilizations had independent numerical systems.  Many students admitted they would have preferred this course be taught in person and for a semester.  I noticed the students' comments that Zoom breakout rooms were not useful.  Once I found out, we dialed back the use of breakout rooms.  Some students would have preferred only three synchronous days per week instead of the four we used.  I did not know I could do that.  I felt I owed the students as much face-to-face time as I could give them.  Another student suggested that the course schedule be set up by week instead of by unit.  At a teaching conference I attended in 2017 (focused on STEM), I was taught the exact opposite, that setting the course stucture by unit rather than weeks minimizes stress on the students.  My students in INTD290 probably preferred the fixed schedule, along with less synchronous time.  One student shared something useful that also occurred to me after the course was done:

\begin{quote}
Professor Hanson, I had such a great time in this course! I really enjoyed how you blended history with your field of study in a way that worked really well with the aim of the course. I definitely wish that we were able to do this course on campus, but I think that you were able to transition it to online very well. I know that there were a lot of adjustments to the course schedule, but I would have liked to have a more solid timeline on how to prepare for the final presentation. We were introduced to the concept of Digital Storytelling early on, but it wasn't until Sonia Chaidez came in the last few weeks of class that that final project option was fully explained to us. It would have been more beneficial to explain digital storytelling halfway through the course so that students could have more time and information to decide between doing a paper or doing a presentation. Overall, I know that you had to balance your teaching and personal life throughout this course, and I appreciate the help and availability that you offered us. I felt that I was very supported and that my work was valued in this course.
\end{quote}

This student was not alone in liking the digital storytelling projects.  The students did a really nice job on them, and I only wish I had inquired if Sonia Chaidez was free earlier in the module to deliver her introduction to WeVideo.  I created an example video for the students involving traveling on an Antarctic expedition (see Sec. \ref{sec:scholarship}), and shared it with about 10 days go in the module.  Had I done this halfway through the module, the students would have had double the time to prepare their projects, but only been exposed to half of the course content.  In person, and in a semester-long format, the WeVideo tutorial could be introduced at the half-way point and the students would have a nice balance of time and exposure to course topics to really enhanced their WeVideo projects.  Overall, I am proud of what we accomplished in this course, as it was offered for the first time under strange circumstances and still received high marks from the students.  I conclude with this interesting thought by one of the students about digital storytelling:

\begin{quote}
My favorite part of the course was that majority of the class did not decide to write a basic final paper, but challenged ourselves to
make a digital story, and I think going forward for this course, this should be the mandated final. It makes things different, and gives them a break from writing redundant papers.
\end{quote}

\end{document}
