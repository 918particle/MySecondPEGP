\documentclass[../../../main.tex]{subfiles}
 
\begin{document}

The introductory STEM courses I have taught are now introduced, with connections to departmental goals and learning focuses listed in Sec. \ref{sec:methods}.
\\
\vspace{0.15cm}
\textbf{\textit{Algebra-based physics (135A/B)}}. Algebra-based physics, PHYS135A/B, is a two-semester integrated lecture/laboratory sequence covering Newton's Laws to electromagnetism\footnote{See supplemental material for example syllabi.}.  PHYS135 is a requirement for majors such as KNS and CHEM.  Students practice problem-solving with algebra, trigonometry, and vectors.  I employ a mixture of TT and PER methods to satisfy \textbf{departmental goals 1, 4, and 6}.
\\
\vspace{0.15cm}
The first learning focus for non-majors is \textbf{curiosity}, with the measureable goals stated in Sec. \ref{sec:methods}.  To satisfy the goal of increasing curiosity, students may present at the outset of class a recent science article.  I encourage the students with extra credit, and I help them to practice communication of scientific ideas (\textbf{Departmental goal 7}).  Once the students overcome nerves, they begin to volunteer and choose content connected to their major.
\\
\vspace{0.15cm}
A second method is to require the students to design a final project in small groups.  The OpenStax textbooks contain many workable examples.  Each group must first submit a proposal in the middle of the semester.  I meet with them to refine the idea.  After data collection, the students practice presenting in office hours.  The students are in control, and the project gives them an avenue for their physics curiosity.  Making this assignment a presentation goes toward \textbf{Departmental goal 7}.  Course evaluation data show that the students' curiosity for physics increases (Question 15).
\\
\vspace{0.15cm}
The second introductory focus is \textbf{improvement of analysis skill}.  I utilize PI modules and PhET simulations strategically.  PI (Peer Instruction) modules were first developed by Eric Mazur \cite{mazur2013peer}, and have better measured performance than TT-only courses.  Physics concepts can be illustrated with PhET (Physics Education Technology) simulations, or used to perform laboratory activities we cannot consruct \cite{phet}.  These two modules are my main PER tools for boosting student problem-solving.  PI modules work because students learn efficiently when explaining ideas to peers.  I also get the opportunity to help the struggling students during discussions.  After short table discussions, students re-submit answers to the exercise they began by themselves.  The answer distribution shifts toward the correct response (see Fig. \ref{fig:exampleData}).  If 70\% of students answer correctly, we are trained to move forward.  I have added the concept of WAT\footnote{e.g. ``What?'' A meme indicating confusion.}.  WAT corresponds to answer E on the students' devices.  If I observe a WAT, I revert to a step-by-step example.  \textit{This strategy ensures inclusivity}, in that we take care to leave no one behind.
\\
\vspace{0.15cm}
The second-half of the lecture/laboratory format is the lab activity or PhET module.  An example of the interplay between labs and PhET occurs in PHYS135B and PHYS180 (electromagnetism).  The students build DC electric circuits.  If the circuit is constructable in our lab, they complete an LA module to measure voltages and electric current to verify Ohm's Law\footnote{Ohm's law states that the current observed is proportional to the voltage in the circuit.}.  If the circuit cannot be easily built in our lab, we simulate it virtually with PhET software.  Whenever possible, we first simulate the circuit in PhET, and then physically construct it to compare simulation and experiment.  The PI modules, PhET modules, and traditional lecture content form a flexible and diverse strategy for improving the students' analysis skill (\textbf{Departmental goals 1, 4, and 6}).
\\
\vspace{0.15cm}
My third introductory course learning focus is \textbf{applications to society}.  The obvious routes are the applications in the OpenStax texts \cite{openstax1} regarding kinesiology and medicine.  The students experience PI modules and example problems with topics such as motion/work/energy in the human body, and nerve cells as DC circuits.  The modules I select depends on the students' majors.  Including content specifically pertaining to the students' majors is highly beneficial to keep students engaged\footnote{See supplemental material for an example of such a unit.}.
\\
\vspace{0.15cm}
\textbf{\textit{Calculus-based physics (150/180)}}. Calculus-based physics, PHYS150/PHYS180, is a two-semester lecture/laboratory formatted sequence that covers calculus-based kinematics, mechanics, work/energy, and electromagnetism\footnote{See supplemental material for example syllabi.}.  The format of these courses is similar to algebra-based physics, mixing PER and traditional content (\textbf{departmental goals 1, 4, and 6}).  I employ PI modules \cite{mazur2013peer} and PhET modules \cite{phet}, and applied calculus concepts\footnote{MATH141/142 may be taken concurrently.}.  There are now PhET tools that visualize calculus concepts.
\\
\vspace{0.15cm}
My PHYS150/180 classes are taught in the same fashion as PHYS135A/B, but include the calculus intrinsic to introductory physics.  Calculus and Newton's Laws were developed by the same people, and are interconnected.  I occasionally pose a calculus problem during the warm-up phase to familiarize the students with a technique that helps solve homework problems.  Occasionally the physics requires concepts that the students will first encounter in Calculus III.  Since this course is taken usually after PHYS180, I only include vector calculus concepts after gauging the comfort level of the students\textit{As a rule, we do not place calculus concepts on exams that the students have not encountered in pre-requisite or concurrent courses.}.
\\
\vspace{0.15cm}
\textbf{\textit{Elementary Statistics (MATH080)}}. Elementary statistics is an introductory course on descriptive and predictive statistics, probability and probability distributions, confidence intervals, and modeling data.  My section of MATH080 was created after discussions with Profs. Radoniqi and Kronholm in Spring 2020.  After a discussion with Prof. Radoniqi during a teaching workshop at Wardman Collaboratory, we felt that elementary statistics would be a good summer course offering for the students.  MATH080 is a course many students need to graduate.  It was not standard procedure in the Department of Mathematics to teach online courses until Spring 2020.  Once \textit{all} courses were going online, Prof. Kronholm informed me the Math Department was giving me the chance MATH080 in Summer Session II, 2020.
\\
\vspace{0.15cm}
I was grateful for the opportunity to teach a math course during summer session.  My approach was to teach MATH080 like PHYS135A/B.  I made no assumptions about the students' math preparation.  The students engaged in warm-ups, TT and PER modules, and PhET modules.  Alhtough PhET stands for Physics Education Technology, that organization now has statistics simulations.  There are modules that are particularly useful for illustrating probability distributions.  Our theme for the course was to \textit{keep it simple.}  After the students completed their final projects, they shared very positive course reviews\footnote{Example final project presentations are included in the supporting materials.}.

\end{document}
