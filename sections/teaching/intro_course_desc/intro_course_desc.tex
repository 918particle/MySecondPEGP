\documentclass[../../../main.tex]{subfiles}
 
\begin{document}

The introductory STEM courses I have taught thus far at Whittier College are now introduced, with connections to departmental goals and learning focuses listed in Sec. \ref{sec:methods}.
\\
\vspace{0.25cm}
\textbf{\textit{Algebra-based physics (135A/B)}}. Algebra-based physics, PHYS135A/B, is a two-semester integrated lecture/laboratory sequence covering Newton's Laws to electromagnetism\footnote{See supplemental material for example syllabi.}.  PHYS135 is a requirement for majors such as KNS and CHEM.  Students practice problem-solving with algebra, trigonometry, and vectors.  I employ a mixture of TT and PER methods to satisfy \textbf{departmental goals 1, 4, and 6}.  I have modified the balance of TT and PER in alignment with department and FPC recommendations.
\\
\vspace{0.25cm}
The first learning focus for non-majors is \textbf{curiosity}, with the measureable goals stated in Sec. \ref{sec:methods}.  To satisfy the goal of increasing curiosity, students may present at the outset of class a recent science journal article pertaining to physics.  I incentivise this with extra credit, and I help them to practice oral communication of scientific ideas (\textbf{Departmental goal 7}).  Once the students overcome nerves and try speaking in front of peers, they begin to choose content connected to their major.  It is wonderful to see the students teach their peers.
\\
\vspace{0.25cm}
A second method is to require the students to design a final project in small groups.  The OpenStax textbooks contain many workable examples.  Each group must first submit a proposal in the middle of the semester.  I then help them refine it and ensure they have proper equipment.  After data collection, the students practice presenting in office hours.  The students design, build, and execute their projects, which gives them an avenue for their physics curiosity.  Making this assignment an oral presentation also goes toward \textbf{Departmental goal 7}.  Course evaluation data show that the students \textit{report an increase in their curiosity for physics}.
\\
\vspace{0.25cm}
The second introductory focus is \textbf{improvement of analysis skill}.  I utilize PI modules and PhET simulations strategically.  PI (Peer Instruction) modules were first developed by Eric Mazur \cite{mazur2013peer}, and have better measured performance than TT only courses.  Physics concepts can be illustrated with PhET (Physics Education Technology) simulations, or used to perform laboratory activities we cannot consruct (e.g. altering the strength of gravity)\cite{phet}.  These two modules are my main PER tools for boosting student problem-solving.  Finally, I learned to use JITT modules \cite{jitt} at a workshop for new physics professors given by the American Association of Physics Teachers (AAPT) in 2017.  The students shared in 2017-2018 that they prefer step-by-step examples to JITT.
\\
\vspace{0.25cm}
PER shows that students learn efficiently from peers explaining their reasoning.  Table discussions encourage this type of learning and give me the chance to help the struggling students.  Spending time with struggling students helps me build a relationship of trust, which alleviates some anxiety.  After short table discussions, students submit their answers again.  We observe the answer distribution shift toward the correct one (see Fig. \ref{fig:exampleData}).  Further, if 70\% of students answer correctly, we move forward.  Thus, we accelerate the pace if the students understand.  This creates the possibility of a few students being left behind, so I have added the concept of WAT\footnote{e.g. ``What?'' A meme indicating confusion.}.  WAT corresponds to answer E.  If I observe a WAT, I revert to a TT example.  \textit{This strategy ensures inclusivity}, in that we strive to leave no one behind in class.
\\
\vspace{0.25cm}
The second-half of the lecture/laboratory format is the lab activity or PhET module.  An example of the interplay between labs and PhET occurs in PHYS135B and PHYS180 (electromagnetism).  The students build DC electric circuits.  If the circuit is constructable in our lab, they complete an LA module to measure voltages and electric current to verify Ohm's Law\footnote{Ohm's law states that the current observed is proportional to the voltage in the circuit.}.  If the circuit cannot be easily built in our lab, we simulate it virtually with PhET software.  Whenever possible, we first simulate the circuit in PhET, and then physically construct it to compare simulation and experiment.  The PI modules, PhET modules, and traditional lecture content form a flexible and diverse strategy for improving the students' analysis skill (\textbf{Departmental goals 1, 4, and 6}).
\\
\vspace{0.25cm}
My third introductory course learning focus is \textbf{applications to society}.  The obvious routes are the applications in the OpenStax texts \cite{openstax1} regarding kinesiology and medicine.  The students experience PI modules and example problems with topics such as motion/work/energy in the human body, nerve cells as DC circuit simulation, and lightning/weather.  Which modules I select depends on the students' majors.  Learning what interests the students and including content specifically pertaining to their majors is highly beneficial to keep students engaged.  Dropping the JITT module also frees more class-preparation time to add material I know particular students will enjoy\footnote{See supplemental material for an example of such a unit.}.
\\
\vspace{0.25cm}
Two final methods for my third learning focus are article discussions and term-papers.  For the third learning focus, these were the least used during remote instruction.  I hope to bring them back this semester.  Article discussions involve a student selecting an online article to present to the class before the TT module for extra credit on homework.  Students practice oral communication of technical ideas, and summarizing quickly in front of a group.  Occasionally, I suggest high-impact articles and offer extra credit on the midterm, which causes a flood of volunteers.  Some brilliant term-papers have also emerged, including the history of the first measurement of the distance to the Sun\footnote{Included in the supplemental material.}.  The story of these first measurements is connected to the first explorations of Antarctica, and I included the astronomy/Antarctica connection in INTD255 (the Antarctica course).  Students opt to create term-papers more rarely, but it does provide them a venue to practice technical writing (\textbf{Departmental goal 7}).
\\
\vspace{0.25cm}
\textbf{\textit{Calculus-based physics (150/180)}}. Calculus-based physics, PHYS150/PHYS180, is a two-semester lecture/laboratory formatted sequence that covers calculus-based kinematics, mechanics, work/energy, and electromagnetism\footnote{See supplemental material for example syllabi.}.  The format of these courses is similar to algebra-based physics, mixing PER and traditional content (\textbf{departmental goals 1, 4, and 6}).  I employ PI modules \cite{mazur2013peer} and PhET modules \cite{phet}.  In addition, these courses require tools from calculus\footnote{MATH141/142 may be taken concurrently.}.  Students new to calculus benefit from PhET tools, which help to visualize calculus concepts like vector addition and vector fields.
\\
\vspace{0.25cm}
My PHYS150/180 classes are taught in the same fashion as PHYS135A/B, but include the calculus intrinsic to introductory physics.  Calculus and Newton's Laws were developed concurrently, often by the same people, making them interconnected.  I occasionally pose a calculus problem during the warm-up or reading assessment phase of class, because I need to familiarize the students with a technique that helps solve physics problems in the current chapter.  Occasionally the physics requires concepts that the students will first encounter in Calculus III, or MATH241 (which covers electromagnetic fields).  I gauge the comfort level of the students, and typically restrict my calculus content to traditional examples or an occasional PI module.  \textit{As a rule, we do not place calculus concepts on exams that the students have not encountered in pre-requisite or concurrent courses}.
\\
\vspace{0.25cm}
\textbf{\textit{Elementary Statistics (MATH080)}}. Elementary statistics is an introductory math course involving descriptive and predictive statistics, probability and probability distributions, confidence intervals, and modeling data.  My section of MATH080 was created after discussions with Profs. Fatos Radoniqi and Bill Kronholm in Spring 2020.  After Prof. Radoniqi called for new summer session courses, I contemplated offering a summer data science course.  It turns out I was not alone, as Profs. Glenn Piner and Mark Kozek are now offering such courses.  After discussion with Prof. Radoniqi after a teaching workshop at Wardman Collaboratory, we felt that elementary statistics is a course similar to an introducion to data science that would really serve the students because it is a course many need to graduate.  It was not standard procedure in the Department of Mathematics to teach online courses until Spring 2020.  As we slowly realized \textit{all} courses were going online, Prof. Kronholm informed me the Math Department was going to give it a chance.
\\
\vspace{0.25cm}
I was grateful for the opportunity to teach a math course in Summer Session II of 2020.  I taught this course as if it was algebra-based physics.  I made no assumptions about the mathematics preparation of the students, and instead used the lessons learned from PHS135A/B and PHYS150/PHYS180.  The students engaged in warm-ups, TT and PER modules, and in particular PhET modules.  Alhtough PhET stands for Physics Education Technology, that organization bas branched out to math and other subjects.  There are statistics modules that were particularly useful for illustrating probability distributions.  Our theme for the course was to \textit{keep it simple.}  After the students completed their final projects, they shared very positive reviews about the course\footnote{Example final project presentations are included in the supporting materials.}.

\end{document}
