\documentclass[../../../main.tex]{subfiles}

\begin{document}

In Sec. \ref{sec:teaching_philosophy}, I remarked that physics provides \textit{order} and \textit{shared meaning} within the liberal arts perspective.  We have known students require evidence-based teaching methods to master the \textit{order} of physics.  Within order, there is problem-solving, analytical thinking, experimentation, and data analysis.  To apply physics concepts to other disciplines (\textit{shared meaning}), students need a variety of PER modules to apply the order they learn.  PER modules must be balanced with TT modules that provide examples for the students to copy and remix.  The students use laboratory activities (LA) to confirm basic physics concepts, and to practice analyzing graphical and numerical data.  During remote instruction, the students experienced physics labs and simulations in online LA modules.  I review each type of module below in Secs. \ref{sec:per} - \ref{sec:ola}.
\\
\vspace{0.25cm}
Students are categorized as \textit{non-majors} or \textit{physics majors}.  Non-majors encounter physics for two semesters in either \textit{calculus-based} or \textit{algebra-based} courses.  Classical physics at the undergraduate introductory level is built upon single-variable calculus, with some multi-variable or vector calculus introduced in the second semester.  However, students who have not taken calculus can still learn using tools from algebra and trigonometry.  \textit{Non-major} students therefore take PHYS135A/B, while those majoring in physics, engineering, math, and integrated computer science take PHYS150/180.
\\
\vspace{0.25cm}
Three focuses are relevant for teaching at the introductory level:
\begin{enumerate}
\item \textbf{Curiosity}.  Good instruction for non-majors should \textit{entice curiosity}, which begins by having an encounter with students where they are in their knowledge, and asking them to think more quantitatively.  Before the pandemic, I would give colloquia and seminars at other schools, and public lectures to children and families in East Los Angeles.  Experiencing people's curiosity forms a starting point, from which we build order.  I have given lectures at Los Nietos Middle School and colloquia here at Whittier College, and invited speakers from UC Irvine to give colloquia as well.  I planned to continue this practice at a Family Science Night at Granada Middle School, but the pandemic forced us to cancel.  Within this teaching focus, I have three measurable goals:

\begin{itemize}
\item Measurably increase student interest in physics as measured by questions 15 and 18 on the evaluation forms.
\item Teach the students to satisfy curiosity through self-designed experiments and pre-designed lab activities.
\item Coach the public speaking skills of the students to empower them to present results to peers.
\end{itemize}

\item \textbf{Improvement of Analysis Skill}.  The order within physics requires analytical skill.  We as physicists help the students develop their problem-solving abilities.  We apply PER modules in introductory courses to train students, while providing a healthy mixture of traditional lecture content and step-by-step examples.  This involves calculations as simple as converting between units (i.e. kilograms to pounds) to plotting the trajectory of a particle in a vector field.  Within this teaching focus, I have two specific goals:

\begin{itemize}
\item Measurably increase the ability of the students to solve word problems (questions 12, 14, 19, and 20 on the evalutations).
\item Teach the students to measure with precision the correct result in laboratory settings.
\end{itemize}

\item \textbf{Applications to Society}. Whittier College students advance in their technical careers if they can qualitatively explain phenomenon using physics.  In recent years, our OER resources \cite{openstax1} \cite{openstax2} have included medical and kinesiological topics.  My students engage in special units, including human muscle motion (in PHYS135A) and nerve systems (in PHYS135B and PHYS180).  I also help the students design experiments which apply to their field. Another tool within this learning focus is the inclusion of student-led summaries of scientific articles, which encourage class discussions about the broader implications for society.  Within this teaching focus, I seek to achieve two measurable goals:

\begin{itemize}
\item Empower the students to present and discuss articles they find relevant or interesting due to the societal impact (see Supplemental Material)
\item Manage and aid in student-designed experiments that are presented to the class, relevant to society (see Supplemental Material)
\end{itemize}

\end{enumerate}

\subsection{Physics Education Research (PER) Modules}
\label{sec:per}

\subsection{Traditional Teaching Modules}
\label{sec:tt}

\subsection{Laboratory Modules}
\label{sec:la}

\subsection{Online Modules}
\label{sec:ola}

\end{document}
