\documentclass[../../../main.tex]{subfiles}

\begin{document}

The pandemic has tested all of us, and I reflected on these experiences in Sec. \ref{sec:family}.  However, after reflecting on \textit{just the teaching}, I realized something startling.  My students experienced \textit{greater success} in the period from January 2019 through Spring 2021, compared to Fall 2017 through Fall 2018.  I have several explanations for why this is the case.
\\
\vspace{0.15cm}
FPC always suggests adjustments, and I am grateful for the professional candor.  There have been three basic suggestions.  First, the pace of my content needs to be controlled.  Second, I needed to increase the number of example problems to give students a starting point.  Third, I needed to include more traditional lecture content.  Traditional content is a term used in physics education research (PER) to refer to the style in which a new equation is first introduced or derived on the board, then solved in examples and displayed in graphical form.  I impemented these suggestions, and the course evaluations showed clear, positive results.
\\
\vspace{0.15cm}
Another reason my courses went more smoothly during the pandemic is related to the way physics is taught in our department.  First, introductory courses are taught using PER modules that work well online.  For example, if we build a lesson around a physics simulation integrated within our textbook, the students experience it the same way at home or in person.  Second, introductory courses in our department are taught in an integrated lecture/laboratory format.  We selected a service called Pivot Interactives that provided interactive, online versions of in-person laboratory activities.  For most of our advanced courses, these same strategies worked as long as we created asynchronous videos of TT modules.  The exception was advanced \textit{laboratory} courses, and I share how I responded in Sec. \ref{sec:adv}.
\\
\vspace{0.15cm}
Finally, our department was well-positioned because we use open educational resources (OER).  \textit{Using OER fosters equity and inclusion, flexibility, and is a strategy that should adopted whenever practical.}  The textbook is free and open-access on any platform in all of my courses, with few exceptions.  I helped give three OER workshops through Wardman Library.  We explained that $\approx 20$\% of students struggle to buy textbooks\footnote{This study was done on students at another school before the pandemic.  According to Wardman Library research, students in focus groups at Whittier College say the same thing qualitatively.}.  Further, our department uses online homework software.  The students have access to all content via Moodle, the open-access book, and online homework at all times.  I use free appointment-booking software that automatically syncs with my schedule.  The students responded positively, knowing I could accommodate their schedule.  Teaching physics, however, goes far beyond issues of problem solving, access, or the pandemic.  In the next section, I reflect on the deeper place physics holds within the liberal arts.

\subsubsection{Physics within the Liberal Arts: Order and Shared Meaning}

Philosophical reflection confuses many physicists.  It's \textit{subjective} ... is what we hear.  When we must engage in philosophy, even the philosophy of teaching, physicists turn to an old friend: plagiarism\footnote{Relax, the jokes are just to keep you awake.  Thanks for doing all this reading.}.  Although I have been reflecting on the values of \textit{order and shared meaning}, the words of my former supervisor Prof. John Beacom (The Ohio State University) already encapsulate them nicely:
\\
\vspace{0.15cm}
\textit{Lost in Space, by John Beacom, TEDx @ The Ohio State University:} \url{https://youtu.be/d6eMdixkoRI}
\\
\vspace{0.15cm}
It is customary to locate physics within the liberal arts tradition by stating that the oldest questions of humanity are questions of physics.  How old is the Universe?  How large is it?  Of what is it made?  \textbf{This custom does not serve the moment}.  It does not address the sense of division, tribalism, and the increase in anti-scientific rhetoric.  One of the reasons physics holds a place within the liberal arts worldview is that it provides \textit{order, and shared meaning.}
\\
\vspace{0.15cm}
Physicists use the word \textit{order} in several ways.  The sense in which I use it here is illustrated by a simple experiment.  Take out your keys, raise them a short distance, and drop them.  How long does it take them to fall?  Is it the same time duration if you repeat it\footnote{Seriously, try it!  You're stuck reading this anyways, right?  Might as well learn some \#\%!*.}? The point is not to understand gravity.  The point is you are having a personal encounter with physics, and it is meaningful to you.  Imagine the entire faculty was together in the Shannon Center to repeat the experiment with two rules.  We hold the keys at the height of the chair in front of us, and we let go simultaneously.  Without these rules, we hear a cacophony.  With the rules, we here a uniform burst of sound.
\\
\vspace{0.15cm}
Studying physics creates \textit{order} from our individual experiences.  We are a diverse group of people, and yet the experiment allows us to reveal order within the Universe.  Gravity does not know who is dropping, and the mass of the keys does not matter.  Our collective experience forms a pattern, and as soon as we admit one law of physics, there could be more.  The Universe is ordered, and the order carries deep meaning.  We can explain the past, control the present, and predict the future together.
\\
\vspace{0.15cm}
Notice another facet of physics: order gives rise to \textit{shared} meaning.  When we learn about gravity by dropping our keys at our desk, and later find out others have the same experience, we develop a shared understanding of the world that bridges divisions.  Physics simply \textit{is}, apart from us.  Moreover, physics is consistent everywhere and over time.  The consistency means that joining the cross-cultural traditions of scientists extending back to the Enlightenment and beyond allows us to build on the shared meaning of our ancestors.  Attending even one national meeting of the American Physical Society (APS) shows us that physics is a discipline that attracts people of all faiths, races, cultures, and ethnicities.
\\
\vspace{0.15cm}
FPC asked me to answer a basic question: \textit{Are there things your physics courses offer a major in business, history, or music that other disciplines cannot?}  The answers are order and shared meaning.  A business major understands how microeconomics drives customer behavior in a sector dealing with scarcity.  Microeconomics predicts optimal prices given simple precepts.  Human beings can be irrational, however, so what \textit{forces} people to behave predictably?  The forces of physics.  Human beings live within an \textit{ordered} Universe that forces us to act if we wish to thrive.  Order arises within the economy just as it does in physical systems: many sub-systems obeying shared rules.
\\
\vspace{0.15cm}
Physics offers a music major shared meaning.  Imagine a student of music trying to understand melodic styles cross-culturally, and she finds that music from one side of a continent sounds different from the other side.  The laws of physics confer shared meaning to the styles, in the sense all humans detect \textit{sound} in the same way.  If a melody contains more than one note, sound waves of different frequencies combine to form \textit{harmony.}  Though people from two different cultures might identify different frequencies as harmonious, people from all cultures perceive \textit{the existence of harmony} via the laws of physics.  Shared meaning arises because physics explains human perception of harmony.  I heard a whale researcher remark in a documentary that ``singing is older than humans,'' because whales evolved before us and they sing.  Since song is made of sound waves, I would add that singing is older than most \textit{animals.}
\\
\vspace{0.15cm}
Physics offers both order and shared meaning to a major in history.  The example I share here is taken directly from my course regarding the history of science in Antarctica.  The leader who created the expedition that led the first humans to the South Pole in 1911 was also the man who led the first complete expedition through The Northwest Passage above Canada: Roald Amundsen.  Amundsen was a Norwegian captain reknowned for his tenacity and curiosity.  The Northwest Passage required more than one year, because the sea would freeze.  The ship had to be anchored before the sea froze.  Once the ship was frozen, the sailors explored the area before the Spring thaw.
\\
\vspace{0.15cm}
During that time, they encountered the \textit{Netsilik} tribe, who greeted them like old friends despite the fact that they had never seen white men before.  The Norwegians realized the Netsilik were partners in survival against the harsh terrain. Amundsen paid attention to how the Netsilik used physics: by melting snow and pouring the hot water onto the bottom of their dogsled runners, the water would freeze to become a layer of ice.  Due to the reduced friction, the Netsilik sleds took less work to pull.  By understanding this physics together, the Norwegians and Netsilik developed shared meaning.  Amundsen later used this trick to upgrade his sleds in Antarctica, and his dogs pulled sleds faster.  His group won the ``race'' to the South Pole, beating their English competitors by several weeks.  This example of shared meaning is one of many I share with my students.

\subsubsection{Concluding Remarks about Teaching Philosophy}

If physics provides a sense of order and shared meaning, \textit{teaching physics} is about the growth of the students towards mastering and applying the order and the shared meaning.  The goal of the professor is to formulate the order of physics into specific equations, testing them through experimentation, and to cause the students to master the equations through problem-solving.  The student usually encounters confusion, followed by an ability to solve specific examples.  Finally, the professor leads the students to shared meaning by showing them how the concepts apply \textit{in general} to other disciplines.
\\
\vspace{0.15cm}
Teaching physics begins by defining the ``system'' under study, with measurable properties like displacement, velocity, acceleration, mass, and charge.  \textit{Classical physics} is a description of the motions, forces and energies that govern all systems.  With the addition of temperature and heat, \textit{thermodynamics} may be added to classical physics.  Students who do not major in physics usually encounter only classical physics.  Physics \textit{majors} progress to \textit{modern physics}, which adds the subjects of relativity and quantum mechanics.  The bulk of PER is done in the context of serving non-majors, and thus the named modules (PI, TT, JITT, and PhET) are usually applied to introductory courses.
\\
\vspace{0.15cm}
A good instructor builds the system of classical physics in students' minds, and leads students to more advanced applications.  At each phase, the instructor also guides laboratory experimentation to provide students proof that the concepts work.  Upon examining my teaching, I have found the correct ``solution'' to build order for the students in our introductory courses is to control the pace of the modules, including many concrete examples, and increasing the proportion of traditional lecture content.  I will discuss how the module system affected this plan in Secs. \ref{sec:intro} and \ref{sec:adv}.
\\
\vspace{0.15cm}
A similar approach has led to good results in my advanced courses.  I have taught an advanced theoretical physics course, a cross-listed physics and computer science course, and a course on digital signal processing (DSP).  I employ the same strategies in these courses as I do in introductory courses, but emphasize TT modules over PER.  The pandemic restricted my ability to provide laboratory activities in advanced courses, but I responded by utilizing a mixture of technology and teamwork with the students.  In my cross-listed computer science and physics course, the students' success and enthusiasm greatly improved relative to the first course version.  I will offer DSP again this January term, and students are already asking how to register.
\\
\vspace{0.15cm}
Finally, the students appear to have learned and grown in my \textit{CON2}, \textit{CUL3}, and College Writing Seminar courses.  In my INTD100 section, the students practiced and improved their technical writing.  The history of science in Antarctica course (\textit{CON2}) was about exploration, self-exploration, and shared-meaning.  The history of science in Latin America course (\textit{CUL3} and \textit{CON2}) was about the history of the discovery of natural order across cultures.  The students also encountered shared meaning by studying people from mesoamerican cultures \textit{doing science and math in their own way.}  Humans from all regions and times in history feel the call to explore the Universe.

\end{document}
