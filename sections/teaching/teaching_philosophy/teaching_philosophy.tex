\documentclass[../../../main.tex]{subfiles}

\begin{document}

The past eighteen months have tested us at Whittier College.  To ensure the safety of the community, we had to alter the nature of interactions with students.  The pandemic has affected all of us in different ways.  Some of us caught the virus or cared for loved ones that did, and others have had to work harder than they ever thought possible.  I reflected on these experiences in Sec. \ref{sec:family}.  However, after reflecting on my experiences \textit{regarding just the teaching}, I realized something startling.  My students and I experienced \textit{greater success} in the period from January 2019 through Spring 2021, compared to Fall 2017 through Fall 2018.  I have several explanations for why this is the case.
\\
\vspace{0.25cm}
The first set of reasons have to do with the work that FPC asked of me in the past, and I am grateful for the professional candor.  There were three basic ideas to implement.  First, the pace of my content needed to be adjusted, in order to maximize overall student success.  Second, I needed to increase the number of step-by-step example problems, in order to give struggling students a starting point.  Third, I needed to include more traditional lecture content in my integrated lecture/laboratory formatted classes.  Traditional content is a term used in physics education research (PER) to refer to the classical teaching style in which a new equation is first introduced or derived on the board, then solved in examples and displayed in graphical form.  Traditional teaching (TT) modules are compared and contrasted with PER modules. I impemented these ideas, and the course evaluations showed major increases in every category.  I provided clear, graphical evidence, and the results were undeniable.
\\
\vspace{0.25cm}
My second group of thoughts about why my courses went more smoothly during the pandemic have to do with the way physics is taught in our department.  There are three reasons my department was well-positioned to make the online transition.  First, introductory courses are taught using PER modules in our department.  We use PER teaching styles that work well in an online or in-person setting.  For example, if we build a lesson around a physics simulation integrated within our textbook, the students are running the simulation in the same way at home as they would in person.  Second, introductory courses in our department are taught in an integrated lecture/laboratory format.  For laboratory activities, we selected a service called Pivot Interactives.  The Pivot Interactives website provided video versions of the same laboratory activities we would have done in class.  Students still collected data, and still analyzed results.  For most of our advanced courses, these same strategies worked, as long as we created asynchronous videos of TT modules\footnote{This was my experience in teaching PHYS330: Electromagnetic Theory, for example.}.  The exceptions were advanced laboratory courses.  I will share what I did in that situation in Sec. \ref{sec:adv}.
\\
\vspace{0.25cm}
The third reason our department was well-positioned has to do with open educational resources (OER).  Using OER fosters equity and inclusion, flexibility, and is a strategy that should adopted whenever practical.  In all of my introductory courses, and some of my advanced courses, the textbook is free and open-access on any platform.  When I was a college student, physics texts cost \$100 and had to be purchased in person to obtain the right version.  I served as faculty speaker at two OER workshops at Wardman Library, where we learned that $\approx 20$\% of students struggle to buy textbooks\footnote{This study was done on students at another school before the pandemic.  According to Wardman Library research, students in focus groups at Whittier College say the same thing qualitatively.}.  Further, we use online homework administration software in our department.  Thus, the students have access to all content via Moodle, the open-access book, and online homework at all times regardless of their financial status.  I learned to use free appointment-booking software that automatically synced with my schedule.  The goal was to stay as flexible as possible for students who had to work or care for younger siblings.  Sometimes I would meet with students while they were on their break at their jobs to help with homework.  The students responded positively to the flexibility.  But teaching physics goes far beyond issues of problem solving, access, or the pandemic.  In the next section, I reflect on the deeper place physics holds within the liberal arts worldview.

\subsubsection{Physics within the Liberal Arts: Order and Shared Meaning}

Philosophical reflection confuses many physicists.  It's not \textit{objective} ... not \textit{testable} ... is what we hear.  When we must engage in philosophy, even the philosophy of teaching, physicists turn to an old friend: plagiarism\footnote{Relax, the jokes are just to keep you awake.  Thanks for doing all this reading.}.  Although I have been reflecting this past year on \textit{order and shared meaning}, the words of my colleague Prof. John Beacom, from the Center for Cosmology and Astro-Particle Physics (CCAPP), already encapsulate how I view the place of physics within higher education.
\\
\vspace{0.25cm}
\textit{Lost in Space, by John Beacom, TEDx @ The Ohio State University:} \url{https://youtu.be/d6eMdixkoRI}
\\
\vspace{0.25cm}
When properly locating physics within the liberal arts world, it is customary to begin by stating that the oldest questions humanity has asked are questions of physics.  How old is the Universe?  How large is it?  Of what is it made?  However, I do not think this custom serves the moment.  From my home in East Los Angeles and knowing the community where I was born, the sense of division, tribalism, and the increase in anti-scientific rhetoric lead me to respond in a different way.  Physics has long had a place within the liberal arts worldview because it provides as least two foundational ideas: \textit{order and shared meaning.}
\\
\vspace{0.25cm}
Physicists use the word \textit{order} in several ways, but the sense in which I use it here is illustrated by a simple experiment.  Take out your keys.  Raise them a short distance above your desk or lap and drop them.  How long does it take them to fall?  Is it the same time duration if you repeat it?  Though you are now doing a physics experiment, the point is not to understand gravity.  The point is you are having a personal encounter with physics through experimentation.  Your experience is via your observations, which are meaningful to you.  Imagine the entire Whittier College faculty was together again in the Shannon Center for the Arts.  We are going to do the experiment together, but with two rules: we hold the keys at the height of the chair in front of us, and we let go when someone gives the signal on stage.  Without these two rules, we hear a cacophony of keys.  When agreeing to follow them, we here a uniform burst of sound as the keys drop simultaneously to the floor.
\\
\vspace{0.25cm}
Following simple procedure creates \textit{order} from our individual experiences.  The Whittier College faculty is a diverse group of people with different family backgrounds, ethnicities, first languages, and more.  Yet the simple experiment allows us to reveal together a piece of order within the Universe.  Gravity does not know who is doing the dropping, and the time duration does not depend on the mass or shape of the keys.  We accept that we should all get equal answers.  As John says in the TEDx talk in the link above: \textit{As soon as you admit there's one law of physics, there could be many.}  The Universe is not total chaos, it is ordered.  The order carries deep meaning: we can explain the past, control the present, and predict the future together.  Our common experience is identical.  After more experimentation, we find the results are identical no matter where we are in the Shannon Center Auditorium.  It is a \textit{universal experience.}
\\
\vspace{0.25cm}
Notice another facet of physics: order gives rise to \textit{shared meaning.}  When we learn about gravity by dropping our keys at our desk, and later find out others have the same experience, we develop a shared understanding of the world that bridges whatever divisions we choose to create amongst ourselves.  Physics simply \textit{is}, apart from us.  Moreover, physics appears to be consistent everywhere and over time.  The consistency means that joining the cross-cultural traditions of scientists extending back to the Enlightenment and beyond allows us to build on the shared meaning of our ancestors.  Attending even one national meeting of the American Physical Society (APS) shows us that physics is a discipline that attracts people of all faiths, races, cultures, and ethnicities.
\\
\vspace{0.25cm}
In our last communication, FPC asked me to answer a basic question: \textit{Are there things your physics courses offer a major in business, history, or music that other disciplines cannot?}  The answers are order and shared meaning.  Take the example of a business major, who understands how microeconomics might drive customer behavior in a sector dominated by small businesses dealing with scarce resources.  Given simple precepts, microeconomics should predict optimal production and prices.  But how is that possible, given that human beings can be irrational?  Is there anything \textit{forcing} people to behave predictably?  The answer is the forces of physics.  Human beings live within an \textit{ordered} Universe that forces us to act if we wish to thrive.  Order arises within the economy in the same way that it does in the Universe: many interacting systems all obeying the same rules.  Consider the cost of production.  What lower limits are there on the production efficiency of a product?  These limits are controlled by physics in that it takes finite time and energy to build a product.  Thus, a natural order arises.
\\
\vspace{0.25cm}
Are there things my physics courses offer a music major?  Yes: shared meaning.  Imagine a student of music trying to understand melodic styles cross-culturally, and she finds that music from one side of a continent sounds different from the other side.  Do the laws of physics confer shared meaning?  Yes, in the sense that \textit{any music is sound, and all humans detect sound in the same way}.  If a melody contains more than one note, sound waves of different frequencies combine to form \textit{harmonies.}  Though one group of humans might perceive one set of frequencies as harmony, different from that of another, all groups of humans (and animals, for that matter) perceive \textit{the existence of harmony} via the laws of physics.  Shared meaning arises because the laws of physics dictate that harmonies are possible, albeit in near infinite variety.  I heard a whale researcher remark in a documentary that ``singing is older than humans,'' meaning we should not be surprised that animals sing.  If song is made of sound waves, then I would add that singing is older than \textit{animals.}
\\
\vspace{0.25cm}
Are there things my courses offer a history major?  Yes: both order and shared meaning.  The example I share here is taken directly from one of my courses, INTD255.  The course is entitled \textit{Safe Return Doubtful: History and Current Status of Modern Science in Antarctica.}  It turns out the leader who created the expedition that led the first humans to the South Pole in 1911 was also the man who led the first complete expedition through The Northwest Passage above Canada: Roald Amundsen.  Amundsen was a Norwegian ship captain reknowned for his tenacity and curiosity.  The Northwest Passage required more than one year, because the seas between the tracts of land in Northern Canada freeze.  The ship had to be anchored as the seas began to freeze.  Once the ship was frozen, the sailors had the winter to explore the area before the Spring thaw.
\\
\vspace{0.25cm}
During that time, they encountered the \textit{Netsilik} tribe, who greeted them like old friends despite the fact that they had never seen white men before.  The Norwegians realized the Netsilik were partners in survival against the harsh northern climate. Amundsen paid attention to how the Netsilik used physics: by melting snow and pouring the hot water onto the bottom of the runners of their dogsleds, they lowered the coefficient of friction between the runners and the snow as the water froze.  The laws of physics (order) are the same for the Netsilik and the Norwegians, so the Netsilik sleds took less work to pull.  By understanding this physics together, the Norwegians and Netsilik developed shared meaning.  Amundsen later used this trick to upgrade his sleds in Antarctica, and his dogs pulled the sleds faster.  His group won the ``race'' to the South Pole, beating their English competitors by several weeks.  This example of shared meaning, order, the laws of physics, and history is one of many I share with my students.

\subsubsection{Concluding Remarks about Teaching Philosophy}

If physics provides a sense of order and shared meaning to liberal arts students, \textit{teaching physics} is about the growth of the students towards mastering and applying the order, and identifying and applying the shared meaning.  Success is measured by the varying degree to which the student can retain, understand, and apply the concepts.  The goal of the professor is to formulate the order of physics into specific equations, testing them through experimentation, and to cause the students to master the equations through problem-solving.  The student usually encounters confusion, then the ability to solve specific examples.  Finally, the professor leads the students to shared meaning by showing them how the concepts apply \textit{in general} to other disciplines.
\\
\vspace{0.25cm}
Teaching physics begins by defining the specific ``system'' under study, with measurable properties like displacement, velocity, acceleration, mass, and charge.  \textit{Classical physics} is a description of the motions, forces and energies that govern all systems.  With the addition of temperature and heat, \textit{thermodynamics} may be added to classical physics.  Students who do not major in physics usually encounter just classical physics.  Physics \textit{majors} progress to \textit{modern physics}, which adds the subjects of relativity and quantum mechanics to the toolkit.  We often distinguish between \textit{physics majors} and \textit{non-majors}, who encounter different types of material.  The bulk of PER is done in the context of serving non-majors, and thus the named modules (PI, TT, JITT, and PhET) are usually applied to introductory courses.
\\
\vspace{0.25cm}
Quality physics instruction involves using the modules to impart basic concepts to the students and grow their successes from the building blocks.  The instructor must be able to build the system of classical physics in students' minds, and then be able to lead students to more advanced applications.  At each phase, the instructor must be able to guide laboratory experimentation, while at the same time demonstrating how physics formulas are used to solve problems.  Upon examining my teaching, I have found the correct ``solution'' for our classical and introductory physics courses to be keeping the pace of the modules under control, including more concrete examples, and increasing the proportion of traditional lecture content.  I will discuss how the module system affected this plan in Secs. \ref{sec:intro} and \ref{sec:adv}.
\\
\vspace{0.25cm}
In my advanced courses, a similar approach has led to good results.  I have taught an advanced theoretical physics course, a cross-listed physics and computer science course with equal emphasis on lecture and laboratory activities, and an engineering course involving the mathematics of signals.  I employ the same active learning modules in these courses as I do in introductory courses, but scale back the PER modules and increase the TT ones.  The pandemic restricted my ability to provide laboratory activities in advanced courses, but I responded by utilizing a mixture of technology and teamwork with the students.  I note that in my cross-listed computer science and physics course, the students' successes and enthusiasm greatly improved relative to the first time we offered it.  In our final January term, I will once more offer my course on signals, and students are already asking how to register for it.
\\
\vspace{0.25cm}
Finally, the students appear to have learned and grown successfully in my \textit{CON2}, \textit{CUL3}, and College Writing Seminar courses.  Although teaching College Writing Seminar was a stretch for me, the students practiced and improved their technical writing.  I was happy to serve the INTD100 program, and will periodically return to it.  The history of science in Antarctica course (\textit{CON2}) was about exploration, self-exploration, and shared-meaning.  It is also connected to my field of research which makes it easier for me to use the content to enlighten the students.  The history of science in Latin America course (\textit{CUL3} and \textit{CON2}) was about the history of the discovery of order within the Universe across cultures, and how our ancestors discovered that order with science.  The students also encountered shared meaning by studying people from mesoamerican cultures \textit{doing science and math in their own way.}  Humans from all regions and times in history feel the call to do mathematics and science.

\end{document}
