\documentclass[../../../main.tex]{subfiles}

\begin{document}

The pandemic has tested our teaching abilities, and we have had to work harder.  I reflected on these experiences in Sec. \ref{sec:family}.  However, after reflecting on my experiences \textit{regarding just the teaching}, I realized something startling.  My students and I experienced \textit{greater success} in the period from January 2019 through Spring 2021, compared to Fall 2017 through Fall 2018.  I have several explanations for why this is the case.
\\
\vspace{0.15cm}
FPC asked of me to make adjustments in the past, and I am grateful for the professional candor.  There were three basic ideas to implement.  First, the pace of my content needed to be adjusted.  Second, I needed to increase the number of step-by-step example problems to give struggling students a starting point.  Third, I needed to include more traditional lecture content.  Traditional content is a term used in physics education research (PER) to refer to the teaching style in which a new equation is first introduced or derived on the board, then solved in examples and displayed in graphical form.  I impemented these ideas, and the course evaluations showed major increases in every category.  I provided clear, graphical evidence, and the results were undeniable.
\\
\vspace{0.15cm}
Another reason my courses went more smoothly during the pandemic is related to the way physics is taught in our department.  First, introductory courses are taught using PER modules.  We use PER modules that work well in an online or in-person setting.  For example, if we build a lesson around a physics simulation integrated within our textbook, the students are running the simulation in the same way at home as they would in person.  Second, introductory courses in our department are taught in an integrated lecture/laboratory format.  We selected a service called Pivot Interactives for online labs.  The Pivot Interactives website provided video versions of the same laboratory activities we would have done in person.  For most of our advanced courses, these same strategies worked, as long as we created asynchronous videos of TT modules.  The exception was advanced \textit{laboratory} courses, and I share how I responded in Sec. \ref{sec:adv}.
\\
\vspace{0.15cm}
Finally, our department was well-positioned because we use open educational resources (OER).  Using OER fosters equity and inclusion, flexibility, and is a strategy that should adopted whenever practical.  In all of my introductory courses, and some of my advanced courses, the textbook is free and open-access on any platform.  Ilectured at two OER workshops at Wardman Library, where we learned that $\approx 20$\% of students struggle to buy textbooks\footnote{This study was done on students at another school before the pandemic.  According to Wardman Library research, students in focus groups at Whittier College say the same thing qualitatively.}.  Further, we use online homework administration software.  The students have access to all content via Moodle, the open-access book, and online homework at all times regardless of financial status.  I use free appointment-booking software that automatically syncs with my schedule.  I stayed flexibile, meeting with students while they were on a break at their jobs.  The students responded positively to the flexibility.  Teaching physics, however, goes far beyond issues of problem solving, access, or the pandemic.  In the next section, I reflect on the deeper place physics holds within the liberal arts worldview.

\subsubsection{Physics within the Liberal Arts: Order and Shared Meaning}

Philosophical reflection confuses many physicists.  It's not \textit{objective}, not \textit{testable} ... is what we hear.  When we must engage in philosophy, even the philosophy of teaching, physicists turn to an old friend: plagiarism\footnote{Relax, the jokes are just to keep you awake.  Thanks for doing all this reading.}.  Although I have been reflecting this past year on \textit{order and shared meaning}, the words of my colleague Prof. John Beacom, from the Center for Cosmology and Astro-Particle Physics (CCAPP), already encapsulate order and shared meaning:
\\
\vspace{0.15cm}
\textit{Lost in Space, by John Beacom, TEDx @ The Ohio State University:} \url{https://youtu.be/d6eMdixkoRI}
\\
\vspace{0.15cm}
It is customary to locate physics within the liberal arts tradition by stating that the oldest questions of humanity are questions of physics.  How old is the Universe?  How large is it?  Of what is it made?  This custom does not serve the moment, for it does not address the sense of division, tribalism, and the increase in anti-scientific rhetoric.  One of the reasons physics holds a place within the liberal arts worldview is that it provides \textit{order, and shared meaning.}
\\
\vspace{0.15cm}
Physicists use the word \textit{order} in several ways.  The sense in which I use it here is illustrated by a simple experiment.  Take out your keys, raise them a short distance, and drop them.  How long does it take them to fall?  Is it the same time duration if you repeat it?  The point is not to understand gravity.  The point is you are having a personal encounter with physics through experimentation, and it is meaningful to you.  Imagine the entire faculty was together in the Shannon Center to repeat the experiment with two rules.  We hold the keys at the height of the chair in front of us, and we let go simultaneously.  Without these rules, we hear a cacophony.  With the rules, we here a uniform burst of sound.
\\
\vspace{0.15cm}
Following procedure creates \textit{order} from our individual experiences.  The Whittier College faculty is a diverse group of people.  Yet the experiment allows us to reveal together order within the Universe.  Gravity does not know who is dropping, and the mass of the keys does not matter.  We accept that our experiences form a pattern, and as soon as we admit one law of physics, there could be more.  The Universe is not total chaos, it is ordered.  The order carries deep meaning: we can explain the past, control the present, and predict the future together.  Our experience is part of a \textit{universal experience.}
\\
\vspace{0.15cm}
Notice another facet of physics: order gives rise to \textit{shared meaning.}  When we learn about gravity by dropping our keys at our desk, and later find out others have the same experience, we develop a shared understanding of the world that bridges divisions.  Physics simply \textit{is}, apart from us.  Moreover, physics appears to be consistent everywhere and over time.  The consistency means that joining the cross-cultural traditions of scientists extending back to the Enlightenment and beyond allows us to build on the shared meaning of our ancestors.  Attending even one national meeting of the American Physical Society (APS) shows us that physics is a discipline that attracts people of all faiths, races, cultures, and ethnicities.
\\
\vspace{0.15cm}
FPC asked me to answer a basic question: \textit{Are there things your physics courses offer a major in business, history, or music that other disciplines cannot?}  The answers are order and shared meaning.  A business major understands how microeconomics drives customer behavior in a sector dealing with scarcity.  Microeconomics predicts optimal prices given simple precepts.  Human beings can be irrational, however, so what \textit{forces} people to behave predictably?  The forces of physics.  Human beings live within an \textit{ordered} Universe that forces us to act if we wish to thrive.  Order arises within the economy in the same way that it does in physical systems: many sub-systems obeying common rules.
\\
\vspace{0.15cm}
Physics offers a music major shared meaning through the physics of sound.  Imagine a student of music trying to understand melodic styles cross-culturally, and she finds that music from one side of a continent sounds different from the other side.  The laws of physics confer shared meaning to the styles, in the sense all humans detect sound in the same way.  If a melody contains more than one note, sound waves of different frequencies combine to form \textit{harmonies.}  Though one group of humans might perceive one set of frequencies as harmony, different from that of another, all groups of humans perceive \textit{the existence of harmony} via the laws of physics.  Shared meaning arises because the laws of physics dictate that harmonies are possible.  I heard a whale researcher remark in a documentary that ``singing is older than humans,'' meaning we should not be surprised that animals sing.  If song is made of sound waves, then I would add that singing is older than most \textit{animals.}
\\
\vspace{0.15cm}
Physics offers both order and shared meaning to a major in history.  The example I share here is taken directly from my course regarding the history of science in Antarctica.  It turns out the leader who created the expedition that led the first humans to the South Pole in 1911 was also the man who led the first complete expedition through The Northwest Passage above Canada: Roald Amundsen.  Amundsen was a Norwegian ship captain reknowned for his tenacity and curiosity.  The Northwest Passage required more than one year, because the seas between the tracts of land in Northern Canada freeze.  The ship had to be anchored before the sea froze.  Once the ship was frozen, the sailors explored the area before the Spring thaw.
\\
\vspace{0.15cm}
During that time, they encountered the \textit{Netsilik} tribe, who greeted them like old friends despite the fact that they had never seen white men before.  The Norwegians realized the Netsilik were partners in survival against the harsh northern climate. Amundsen paid attention to how the Netsilik used physics: by melting snow and pouring the hot water onto the bottom of their dogsled runners, they lowered the coefficient of friction against the snow as the water froze.  The Netsilik sleds took less work to pull, and the Norwegians learned.  By understanding this physics together, the Norwegians and Netsilik developed shared meaning.  Amundsen later used this trick to upgrade his sleds in Antarctica, and his dogs pulled the sleds faster.  His group won the ``race'' to the South Pole, beating their English competitors by several weeks.  This example of shared meaning, order, the laws of physics, and history is one of many I share with my students.

\subsubsection{Concluding Remarks about Teaching Philosophy}

If physics provides a sense of order and shared meaning, \textit{teaching physics} is about the growth of the students towards mastering and applying the order and the shared meaning.  The goal of the professor is to formulate the order of physics into specific equations, testing them through experimentation, and to cause the students to master the equations through problem-solving.  The student usually encounters confusion, then the ability to solve specific examples.  Finally, the professor leads the students to shared meaning by showing them how the concepts apply \textit{in general} to other disciplines.
\\
\vspace{0.15cm}
Teaching physics begins by defining the specific ``system'' under study, with measurable properties like displacement, velocity, acceleration, mass, and charge.  \textit{Classical physics} is a description of the motions, forces and energies that govern all systems.  With the addition of temperature and heat, \textit{thermodynamics} may be added to classical physics.  Students who do not major in physics usually encounter just classical physics.  Physics \textit{majors} progress to \textit{modern physics}, which adds the subjects of relativity and quantum mechanics to the toolkit.  The bulk of PER is done in the context of serving non-majors, and thus the named modules (PI, TT, JITT, and PhET) are usually applied to introductory courses.
\\
\vspace{0.15cm}
The instructor must be able to build the system of classical physics in students' minds, and then be able to lead students to more advanced applications.  At each phase, the instructor must be able to guide laboratory experimentation, while at the same time demonstrating how physics formulas are used to solve problems.  Upon examining my teaching, I have found the correct ``solution'' for our classical and introductory physics courses to be keeping the pace of the modules under control, including many concrete examples, and increasing the proportion of traditional lecture content.  I will discuss how the module system affected this plan in Secs. \ref{sec:intro} and \ref{sec:adv}.
\\
\vspace{0.15cm}
A similar approach has led to good results in my advanced courses.  I have taught an advanced theoretical physics course, a cross-listed physics and computer science course, and a course on digital signal processing.  I employ the same active learning strategies in these courses as I do in introductory courses, but emphasize TT modules over PER.  The pandemic restricted my ability to provide laboratory activities in advanced courses, but I responded by utilizing a mixture of technology and teamwork with the students.  In my cross-listed computer science and physics course, the students' successes and enthusiasm greatly improved relative to the first time we offered it.  In our final January term, I will once more offer my course on signals, and students are already asking how to register.
\\
\vspace{0.15cm}
Finally, the students appear to have learned and grown successfully in my \textit{CON2}, \textit{CUL3}, and College Writing Seminar courses.  In my INTD100 section, the students practiced and improved their technical writing.  I was happy to serve the INTD100 program, and I can return to it periodically.  The history of science in Antarctica course (\textit{CON2}) was about exploration, self-exploration, and shared-meaning.  The history of science in Latin America course (\textit{CUL3} and \textit{CON2}) was about the history of the discovery of natural order across cultures.  The students also encountered shared meaning by studying people from mesoamerican cultures \textit{doing science and math in their own way.}  Humans from all regions and times in history feel the call to do mathematics and science.

\end{document}
