\documentclass[../../../main.tex]{subfiles}
 
\begin{document}

Advising and mentoring students resembles our teaching practice, because we must create a sense of \textit{order and shared meaning} in the mind of the student surrounding the curriculum.  The curriculum must be broken into its constituent parts, and the student must be shown how the parts fit together into a whole.  Good advising also follows the teaching form of drawing out the ideas of the student and examining them as a student-teacher partnership.  This partnership takes on different forms, depending on the type of student.  What follows is a reflection on the type of advising that is appropriate given the types of students we encounter.
\\
\vspace{0.25cm}
Physics professors often classifiy students into two broad categories: \textit{non-majors} and \textit{majors} (see Sec. \ref{sec:teaching_philosophy}).  Most of our advisees as teachers fall into the first category.  In fact, according to Tab. \ref{tab:oer} (Sec. \ref{sec:teaching}), I have instructed 402 students since Fall 2017, and 206 students were non-majors taking algebra-based physics.  This implies that that 51 percent of my students are non-majors.  Thus, I am a accustomed to teaching and advising outside of my field, because it is the norm for my field.  I have taken first-year advisees on two occasions: Fall 2019, and Fall 2020 (Tab. \ref{tab:advisees}), for a total of about 30 advisees.
\\
\vspace{0.25cm}

\begin{table}
\centering
\begin{tabular}{| c | c |}
\hline
\hline
Semester & \textbf{Number of First Year Advisees} \\ \hline
Fall 2019 & 15 \\ \hline
Fall 2020 & 14 \\ \hline
\hline
All semesters & \textbf{Physics, ICS, and 3-2 Majors} \\ \hline 
& Cassady Smith (Physics '20) \\ \hline
& John Paul G\'{o}mez-Reed (Math/ICS '21) \\ \hline
& Nicolas Clarizio (Physics, Business Admin. '19) \\ \hline
& Raymond Hartig (Physics and Math '23) \\ \hline
& Adam Wildanger (3-2 Engineering '21) \\ \hline
& Natasha Waldorf (ICS/Physics '24) \\ \hline \hline
All semesters & \textbf{Whittier Scholars Program Majors} \\ \hline
& Nicolas Bakken-French (WSP '21) \\ \hline
\end{tabular}
\caption{\label{tab:advisees} A summary of my advisees, broken into three categories: first-year advisees, STEM majors, and WSP majors.}
\end{table}

Advising non-majors follows a basic progression: introducing them to the curriculum and campus (\textit{order}), beginning a conversation surrounding major selection (\textit{shared meaning}), and future course selection.  First-year students need careful instructions on how to best take advantage of our liberal arts curriculum.  If the students are first-generation, they might need an introduction to the structure of college in general.  In Fall 2019, along with the help of a wonderful student coordinator named Kat Garrison (KPOET radio), we introduced my new advisees to the curriculum.  We find another classification useful for first year advisees: those are are certain of their major selection immediately, and those are still deciding.
\\
\vspace{0.25cm}
One advisee of mine named Shengyi Liu felt comfortable taking 19 credits, and he had the desire to go to medical school.  He knew what major to choose and he was highly organized.  Thus, I shifted my advising to discussions about passing college writing seminar, letters of recommendation, and research.  An example of a student that was not at all sure about major selection was my advisee Andrea Wainwright.  She excelled at French in high school, but I could tell from our conversations that she was interested in art history and history as well.  I led her towards the Whittier Scholars Program, in addition to encouraging that she master French in our time with us.
\\
\vspace{0.25cm}
Some students require more conversation to locate their main reason for attending college.  Some athletes want to play varsity sports, but are unsure what courses make sense for them.  For these students, I have begun the practice of completing a LinkedIn Profile with them.  We use the job search feature to locate geographically the firms they feel match their employment goals.  Then I have them examine the skills required, and we try to arrive at a major selection that aligns with these skills.  This approach works because it helps the students to \textit{order} their thinking in a practical way.  It also creates a sense of \textit{shared meaning}, because the studenta and I share an understanding of their goals after graduation.
\\
\vspace{0.25cm}
Though advising majors is a smaller fraction of the work I have done in the area of advising and mentoring, it is a more involved process.  In my experience, people who wish to major in the STEM areas listed in Tab. \ref{tab:advisees} already have a firm idea of major selection before they arrive at Whittier College.  Part of the challenge is to assess the mathematics and computer programming skill they gained in high school, and guide them to the correct introductory courses.  The next task is to draw from the student their ideas about the purpose of their major selection, and whether it falls under any broad category like theoretical physics, experimental physics or engineering, or business applications.
\\
\vspace{0.25cm}
The connections between research (both mine and the students') and course/major selection also becomes apparent in these conversations.  Students in the STEM area often base their major selection on the kind of research or job role they prefer in the long run.  It is my role as an educator to introduce them to the intellectual variety of their chosen area.  Sometimes this aligns with my research, and sometimes it does not.  There have been times a student has drifted towards another professor (John Paul, for example, worked with Prof. Fred Park for a while).  Sometimes I develop shared interests with a student, and we develop a project idea together that is related to my research experience but not perfectly aligned.  Finally, and rarely, a student knows that my research is what they want to do, and they are eager to get started (Raymond Hartig, for example, plans to attend graduate school for physics).

\end{document}