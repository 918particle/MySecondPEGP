\documentclass[../../../main.tex]{subfiles}
 
\begin{document}

Advising and mentoring students resembles our teaching practice, because we must create a sense of \textit{order and shared meaning} in the mind of the student surrounding the curriculum.  The curriculum must be broken into its constituent parts, and the student must be shown how the parts fit together into a whole.  Good advising also follows the teaching form of discovering the perspective of the student in a student-teacher partnership.  This partnership takes on different forms, depending on the type of student.  It is important to note that I have already given examples of \textit{developing a sense of belonging} with the first-year students in Secs. \ref{sec:teaching} and \ref{sec:service}, so I will move forward with details about my advising practices.  What follows is a reflection on the type of advising that is appropriate given the types of students we encounter.
\\
\vspace{0.25cm}
Physics professors often classifiy students into two broad categories: \textit{non-majors} and \textit{majors} (see Sec. \ref{sec:teaching_philosophy}).  Most of our advisees as teachers fall into the first category.  Table \ref{tab:advisees} contains the distribution of my advisees since 2019.  In fact, according to Tab. \ref{tab:oer} (Sec. \ref{sec:teaching}), I have instructed 402 students since Fall 2017, and 206 students were non-majors taking algebra-based physics.  This implies that that 51 percent of my students are non-majors.  Thus, I am accustomed to teaching and advising outside of my field.  Like teaching, advising students outside of my field of study requires emphasis on different goals than my students that are STEM majors.  I have taken first-year advisees on two occasions: Fall 2019, and Fall 2020 (Tab. \ref{tab:advisees}), for a total of about 30 advisees.
\\
\vspace{0.25cm}

\begin{table}
\centering
\begin{tabular}{| c | c |}
\hline
\hline
Semester & \textbf{Number of First Year Advisees} \\ \hline
Fall 2019 & 15 \\ \hline
Fall 2020 & 14 \\ \hline
\hline
All semesters & \textbf{Physics, ICS, and 3-2 Majors} \\ \hline 
& Cassady Smith (Physics '20) \\ \hline
& John Paul G\'{o}mez-Reed (Math/ICS '21) \\ \hline
& Nicolas Clarizio (Physics, Business Admin. '19) \\ \hline
& Alex Ortiz-Valenzuela (3-2 Engineering/Physics  '22) \\ \hline
& Raymond Hartig (Physics and Math '23) \\ \hline
& Adam Wildanger (3-2 Engineering/Physics '21) \\ \hline
& Matthew Buchanan Garza (ICS/Physics '23) \\ \hline
& Natasha Waldorf (ICS/Physics '24) \\ \hline \hline
All semesters & \textbf{Whittier Scholars Program Majors} \\ \hline
& Nicolas Bakken-French (WSP '21) \\ \hline
\end{tabular}
\caption{\label{tab:advisees} A summary of my advisees, broken into three categories: first-year advisees, STEM majors, and WSP majors.  There are some first year advisees who have chosen ICS/Math for their major, for whom I remain a mentor.  One example is Emily List (ICS/Math '23).}
\end{table}

Advising non-majors follows a basic progression: introducing them to the curriculum and campus (\textit{order}), beginning a conversation surrounding major selection (\textit{shared meaning}), and future course selection.  First-year students need careful instructions on how to best take advantage of our liberal arts curriculum.  If the students are first-generation, they might need an introduction to the structure of college in general.  In Fall 2019, along with the help of a wonderful student coordinator named Kat Garrison (KPOET radio), we introduced my new advisees to the curriculum.  We find another classification useful for first year advisees: those are are certain of their major selection immediately, and those are still deciding.
\\
\vspace{0.25cm}
One advisee of mine named Shengyi Liu felt comfortable taking 19 credits, and he had the desire to go to medical school.  He knew what major to choose and he was highly organized.  Thus, I shifted my advising to discussions about passing college writing seminar, letters of recommendation, and research.  An example of a student that was not at all sure about major selection was my advisee Andrea Wainwright.  She excelled at French in high school, but I could tell from our conversations that she was interested in art history and history as well.  I led her towards the Whittier Scholars Program, in addition to encouraging that she master French in our time with us.  Another first-year mentee of mine, Wyatt Killien, was at first interested in physics.  After our conversations, we realized it was not actually physics that he wanted, but merely physics as a means to an end in graphical design and digital art.  After he discussed the change with his family, we directed him to the Art and Digital Design program.
\\
\vspace{0.25cm}
Some students require more conversation to identify their main reason for attending college.  Student-athletes, for example, want to play varsity sports but are unsure what courses make sense for them.  Others have only the general area but not a specific idea for a major.  For the students who have more uncertainty about their path to a degree, I have begun the practice of completing a LinkedIn Profile with them.  We use the job search feature to locate geographically the firms they feel match their employment goals.  Then I have them examine the skills required, and we try to arrive at a major selection that aligns with these skills.  This approach works because it helps the students to \textit{order} their thinking in a practical way.  It also creates a sense of \textit{shared meaning}, because the students and I share an understanding of their goals after graduation.  A side benefit of this procedure is that we remain connected on LinkedIn after graduation, and we can contact each other.
\\
\vspace{0.25cm}
Though advising STEM majors is a smaller fraction of the work I have done in the area of advising and mentoring, it is a more involved process.  In my experience, people who wish to major in the STEM areas listed in Tab. \ref{tab:advisees} already have a firm idea of major selection before they arrive at Whittier College.  Part of the challenge is to assess the mathematics and computer programming skill they gained in high school, and guide them to the correct introductory courses.  The next task is to draw from the student their ideas about the purpose of their major selection, and whether it falls under any broad category like theoretical physics, experimental physics or engineering, or business applications.  I pay extra close attention when I have an advisee in the 3-2 program, because the core requirements plus liberal arts requirements must all be satisfied in three years.
\\
\vspace{0.25cm}
The connections between research (both mine and the students') and course/major selection also becomes apparent in these conversations.  Students in the STEM area often base their major selection on the kind of research or job role they prefer in the long run.  For this reason, completing LinkedIn profiles with my STEM students is also a good idea.  For Cassady Smith and John Paul G\'{o}mez-Reed, I worked closely with the Career Center staff to help them craft resum\'{e}s.  Recently, I have restarted that practice, since my students will be applying for internships with the Navy research laboratory (see Sec. \ref{sec:scholarship}).  However, it is also my role as an educator to introduce them to the intellectual variety of their chosen area.  \textit{Sometimes the interests of the student align with my research, and sometimes they do not.  I advise them nonetheless, and do my best to meet the student where they are to guide them forward.}
\vspace{0.25cm}
\\
There have been times a student has drifted towards another professor (John Paul, for example, worked with Prof. Fred Park for a while because he was interested in learning more about machine learning).  Sometimes I develop shared interests with a student, and we develop a project idea together that is related to my research experience but not perfectly aligned.  Finally, and rarely, a student knows that my research is what they want to do, and they are eager to get started (Raymond Hartig, for example, plans to attend graduate school for physics).  In the supporting material, I have included two letters from two of my advisees regarding projects we developed together from start to finish as a team.  The first is from Raymond Hartig regarding our novel mathematical physics model of Askaryan radiation.  I know that Raymond wants to double-major in math and physics and then apply to graduate school, so I have been coaching him for that process.  The second note is from Nicolas Bakken-French, who graduate from the Whittier Scholars Program after creating a holistic study of glaciology with me that included travel around the world.

\end{document}