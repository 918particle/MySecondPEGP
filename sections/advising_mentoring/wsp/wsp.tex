\documentclass[../../../main.tex]{subfiles}
 
\begin{document}

I have had a wonderful time recruiting students for the Whittier Scholars Program.  There are two moments that stand out for me.  The first happened when I accompanied Nicolas Bakken-French to his final meeting with Profs. Rehn and Kjellberg, where his planned scholarship was approved.  After the meeting, we were walking back to my office and Nick was electrified.  As we parted ways, he gave me a big hug and thanked me for helping him gain this opportunity to graduate with a custom major in Environmental Analysis and Glaciology.  We had argued that there is a useful connection between understanding the \textit{science} of glaciers and glacial melt, the cultural impact of glaciers on communities, and the environmental science and policy decisions that affect glaciers.
\\
\vspace{0.15cm}
The second moment that stands out is when I was meeting with my first-year advisee Emma Walston to talk about course selection.  She knew that she wanted to continue with French, but I was also hearing remarks like ``Well, I'm also interested in art history, but I just love the renaissance, and oh, there's history as well ...''  At some point, I interjected: ``You know we have a special program that allows you to design your own major, right?''  Her eyes widened.  ``Oh, that sounds awesome, I think that is for me!''  For some students who simply love learning and want to know everything, and for whom one major is not enough, I know to direct them to WSP101: The Individual, Identity, and Community.  One near-term goal I have is to serve on the Whittier Scholars Advisory board, so that when students take that first step, I can help them form their educational plans in detail.

\subsection{Organization of Field Deployments}

In many ways, my student, Nicolas Bakken-French, was an exceptional case.  Nicolas simply walked into my office with confidence, and said ``I heard you do research in Antarctica.  Is there a way I can help?''  I could tell he had a fire in his belly, also called the \textit{Nansen spirit} in my INTD255 course about Antarctic science and exploration.  Fridtjof Nansen was the first human to cross Greenland, and held the record for furthest North latitude before the North Pole was finally reached by his proteg\'{e} Roald Amundsen.  Exploring Antarctica is not that different from operating in Nordic climates, and it turns out Nick had already been doing that for several years when visiting the Norwegian half of his family.
\\
\vspace{0.15cm}
I knew right away to try to secure him a spot on the next ARIANNA expedition.  We hoped to send him with our recently built drone (thanks to Nick Clarizio, Physics and Business '19), that he could use it to take aerial photos of glaciers feeding into the Ross Ice Shelf.  My ARIANNA colleagues at UC Irvine agreed to add him to the roster.  There are two stages required.  First, one must pass medical checks, which are extensive the first time.  Nick finished the process, which includes multiple medical appointments for blood and cardiovascular tests.  The second stage is to acquire travel documents and airline tickets.  We found resources to support Nicks' air travel.  I met with VP Andrew Wallis (International Programs) to create a plan for keeping Nick enrolled as a student while he was away.  Most Antarctic expeditions take place between November and February, when the sun is up.  We coordinated with the instructors of Nick's courses to ensure the content he would miss could be completed remotely\footnote{Just a few months later, we \textit{all} had to create fully remote content.}.  Nick was taking my INTD255 course (the Antarctica course), and I had hoped that he could video call the class from McMurdo to present on our work there.
\\
\vspace{0.15cm}
Then the ARIANNA budget was cut.  No expedition was to take place that year out to Moore's Bay to finish data collection from the detector.  The pandemic wiped out the next season, and we were dead in the water.  Those that work in polar research know that events like these are not abnormal.  We encounter plenty of similar stories in my INTD255 course, where expeditions are put on hold if finances or logistics are not yet in order.  I worked with Career Services to reorganize Nick's internship credits so that he could help with a physics project at UCI.  He helpd test a new heating element that can melt slots into glacial surface snow.  The slots are ncessary to insert instrumentation uniformly as glacial probes.  After Nick returned from UCI, we set out a roadmap of programs to which he could apply to gain nordic exploration and scientific experience.  The plan merged climate science, the cultural impacts of glaciers, and the science of glaciology.  Nick went on to travel to Iceland, the National Outdoor Leadership School (NOLS), and the Juneau Ice Field of Alaska.
\\
\vspace{0.15cm}
The idea began to come together as a photo-journalistic experience.  Nick was to learn the scientific trade from the experts running the programs at the sites, and bring back photos with a dual purpose.  The photos were to provide a geological record of glacial retreat, and also evoke a sense of \textit{aliveness} from an artistic perspective.  In \textit{The Secret Lives of Glaciers,} by Prof. Jackson (National Geographic Society Explorer and TED Fellow), the author writes about experiencing a glacier as if it is alive.  Glaciers breath, move, sweat, and make sounds.  They influence agriculture and fisheries.  Human cultures near glaciers have always ascribed some degree of life to glaciers.  The holistic view of glaciology and culture was an idea Nick encountered in Iceland, and he applies it to glaciers and cultures in Latin America, Alaska, Wyoming, and Iceland in his thesis.

\section{The Finished Product}

My role in helping Nick to finish his degree was two-fold.  First, I had to guide him through the process of counting credits, and ensuring that Prof. Rehn, myself, and Nick were all on the same page.  Since Nick was denied the chance to go to Antarctica, we made sure that every credit he took online and in the field counted towards his WSP major.  Second, it was my job to make sure the final thesis was polished in both the scientific and writing senses.  Regarding the first round of polishing, I made sure that Nick applied my INTD100 techniques to his thesis.  Scientific graphics are used when appropriate, to enhance and clarify the logic and flow of ideas.  The captions are edited to include real units, which are explained in the text.  Concise sentences are used to explain the logical chain of ideas that surround glaciological claims.
\\
\vspace{0.15cm}
One example is found in the section that explains how we infer global average temperature over 800,000 years from Antarctic ice cores.  The specifics of the measurement involve analyzing radioactive isotope ratios in air bubbles trapped within the ice at varying core depth.  The radioactive isotope analysis is mapped onto years in the past by using layering and ice physics.  We wanted the language for the thesis to reach a broad audience, so we practiced generalizing statements filled with scientific jargon while keeping the logic intact.  Overall, we think the thesis is accessible to readers with diverse training and engagement with glaciology and climate science.
\\
\vspace{0.15cm}
The thesis begins by explaining the way in which Greenlandic and Antarctic glaciers and ice shelves will influence global climate in the next 50-100 years.  We provide case studies of the effect of glaciers on agriculture in Wyoming, fisheries in Alaska, native cultures in Latin America and Iceland, and biodiversity in California.  One strategic choice we made was to focus the detailed analysis on Iceland.  Iceland provided the holistic picture in one location, though all of Nick's expeditions were valuable.  With the Iceland expedition, we were able to communicate the message without expanding the thesis length.  We included cultural persectives on glaciers in Southeastern Iceland, near the Vatnaj\"{o}kull ice cap.  The icecap is the largest in Europe, and it feeds many glaciers that dominate the fertile landscape near the ocean.  The people farm the land between the ice cap and the ocean, and glaciers can retreat or advance to change the landscape.  The older generation views glacial melt as just part of the natural cycle, whereas others can see that the shift will become permanent.  Some even have taken advantage of the situation to form ``last chance'' tourism companies, bringing in eco-tourists eager to glimpse the mighty glaciers before they disappear.
\\
\vspace{0.15cm}
My second role as Nick's advisor for the thesis was to help Nick polish the writing.  Unfortunately, this took place during Sring 2021.  I was stretched very thin, and I think we could have done a better job polishing writing.  Normally, when I help to write a paper, I have a process for outlining, building, and polishing.  There was less time for this than I would have liked.  Nick was due to go to Alaska for the summer to explore the Juneau ice fields, and I was also preparing for the ONR summer research.  The shining jewel of the dissertation, however, are Nick's photographs.  We include them in the end, and I hope their truth and beauty make up for a few awkward sentences.  I introduced WeVideo to Nick so that he could weave them into a digital storytelling piece for the WSP senior presentation.  Nick has told me recently that these photos and the rest of the work will someday be added to a book underway with his colleagues from Iceland.  We look forward to sharing this book with the Whittier College community.

\end{document}
