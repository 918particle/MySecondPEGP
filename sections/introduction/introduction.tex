\documentclass[../../main.tex]{subfiles}
 
\begin{document}

\section{Beginning}

Friends,
\\
\vspace{0.25cm}
I have compiled a report on my progress as a liberal arts educator in the Department of Physics and Astronomy for the period of 2019 through 2021.  The following is a reflection on the development of my educational and scholarly practices, and the service I have performed for the College as a mentor, advisor, and committee member.  I strive to perfect my teaching abilities, and I am pleased to report that my students are learning and growing.  In our last communication, after my supplemental PEGP from 2019, you concurred that my practices are serving the students well, and that meant a great deal to me.  One note that stood out was a request to enrich my teaching philosphy by reflecting on how it serves the liberal arts.  The given example was about the utility of physics to non-STEM students.  I have put a lot of thought into this enrichment, I have progress to share with you.
\\
\vspace{0.25cm}
I have included in my teaching philosphy (Sec. \ref{sec:teaching_philosophy}) my vision for the intersection of broader liberal arts education with physics, mathematics, computer science, and engineering, as I teach all of these.  Further, I have created and taught new liberal arts courses in the \textit{Connections 2} and \textit{Culture 3} categories, as well as a College Writing Seminar on scientific and technical writing.  I used these courses to show my students how physics, mathematics, and engineering intersect with the history of our ancestors and how we all use scientific modes of thought to thrive in a variety of contexts.  In my College Writing Seminar, we sharpened the skills of conciseness, precision and clarity, and organization in writing.  Though these skills apply to physics, they are useful in all writing in which abstract or difficult ideas are communicated.
\\
\vspace{0.25cm}
At a PEGP workshop, a colleague emphasized weaving a narrative about who we are.  I have shared my academic origins, and my vision for multi-disciplinary teaching and scholarship at Whittier College (Sec. \ref{sec:origin}).  I have also written about my family and working during the pandemic (Sec. \ref{sec:family}).  I hope these sections provide you with proper insight about our family and how we hope to serve Whittier College.  We thank the FPC in advance for what is sure to be a difficult year of service.  We also express our gratitude for allowing us to postpone the PEGP for one year.  This helped my family avoid a difficult situation.  My spouse is considered an essential worker, and was called back after a shortened maternity leave.  I have been working as a full-time professor and a full-time parent \textit{by myself} each day since that happened.  Thankfully, my \textit{suegra} (mother-in-law) has stepped in to help.  I look forward to seeing you this Fall, and from our family to yours, we hope you are well.
\\
\vspace{0.25cm}
Sincerely,
Prof. Jordan C. Hanson

\subsection{Academic Origins}
\label{sec:origin}

We all share a common origin as professors, for we first attend college and then graduate school, where we encounter ideas that inspire us.  When I was an undergraduate at Yale University, my family nudged me towards engineering.  My curiousity kept returning me to fundamental physics courses.  In my heart, I knew that I wanted to offer fresh discoveries about the Universe to people, and I fell in love with enlightening people about Nature.  In particular, I learned that the laws of physics morph and merge into one another as the energy of matter increases to relativistic scales where particles move near the speed of light.  I also learned that deep from within the Universe originates a flux of sub-atomic particles ten thousand times more energetic than any human has ever created, with mysterious origin: \textit{the cosmic rays}.  The physics that controls their acceleration has remained unknown at energies that large for a over a century, and it could reveal new fundamental laws of Nature.  I applied for graduate school in the hopes of one day becoming a professor of physics.  I hoped to show that within the cosmic ray flux there are neutrinos of world-record breaking energies.
\\
\vspace{0.25cm}
The University of California at Irvine (UCI) is a pioneering institution in the field of ultra-high energy neutrino (UHE-$\nu$) research, beginning with the Radio Ice Cherenkov Experiment (RICE) \cite{PhysRevD.85.062004} and Antartic Impulsive Transient Antenna (ANITA) \cite{PhysRevD.99.122001} collaborations.  UHE-$\nu$ carry information about the highest-energy accelerators in the cosmos, and could reveal fundamental physics unexplored on Earth.  Sources of UHE-$\nu$ may be classified as \textit{astrophysical} or \textit{cosmogenic}.  Cosmogenic UHE-$\nu$ are the decay products of cosmic rays interacting with ambient photons from the Big Bang.  Cosmogenic UHE-$\nu$ could reveal the source of UHE cosmic rays because the UHE-$\nu$ travel in a straight line from the source of the decay.  Astrophysical UHE-$\nu$ produced by black holes would reveal the underlying acceleration physics \cite{Astro2020_1}.  Both types probe the electroweak interaction at energies inaccessible to human beings \cite{Astro2020_2}.  UHE-$\nu$ observations would, for example, provide insight into quantum mechanics at record energies \cite{PhysRevD.83.113009}.
\\
\vspace{0.25cm}
Detection of UHE-$\nu$ has thus been a long-time goal of the physics community.  When UHE-$\nu$ have energies aboev a certain threshold, they create cascades of particles in matter that radiate in the radio-frequency (RF) bandwidth, a process known as the Askaryan effect \cite{ask1} \cite{ask2} \cite{PhysRevLett.86.2802} \cite{PhysRevLett.99.171101, PhysRevD.101.083005}.  The IceCube Collaboration published the observations of extra-solar neutrinos using optical techniques at record-breaking energies \cite{PhysRevLett.111.021103}, and later showed that the UHE-$\nu$ flux is strikingly close to theoretical predictions based on what we know about cosmic rays \cite{Aartsen_2015} \cite{WB}.  IceCube analyses have not found UHE-$\nu$ events with energy greater than $10^{15}$ electron-volts \cite{PhysRevD.98.062003}, and the authors of \cite{PhysRevD.98.062003} conclude that Askaryan-class detectors are the logical next step.  My colleagues in the field have decided to upgrade IceCube to include RF detectors in a project known as IceCube Generation 2, or IceCube Gen2 (\url{https://icecube.wisc.edu/}).
\\
\vspace{0.25cm}
Askaryan-class detectors improve UHE-$\nu$ prospects because Askaryan radiation is in the radio bandwidth \cite{10.1016/j.astropartphys.2017.03.008}.  UHE-$\nu$ must strike some material in the Earth's crust that produces an observable radio pulse.  It turns out that radio waves travel $\approx 1$ km in \textit{Antarctic ice} \cite{10.3189/2015jog14j214} \cite{10.3189/2015jog15j057} \cite{barwick_besson_gorham_saltzberg_2005}.  Thus, we create stations comprised of RF antenna channels, supporting electronics, and solar panels \cite{10.1109/tns.2015.2468182}.  Stations are triggered to read out station channel signals when a given number of channel signals cross a threshold \cite{sst}.  The dataset is comprised of RF waveforms representing the signals in all the stations, and the data can be used to reconstruct UHE-$\nu$ interactions \cite{10.1088/1475-7516/2019/11/030} \cite{10.1088/1748-0221/15/09/p09039}.  This type of detector is called an \textit{in-situ} array.  As a graduate student at UCI, I led two Antarctic expeditions to create a prototype \textit{in-situ} array: the Antarctic Ross Ice Shelf Antenna Neutrino Array (ARIANNA).
\\
\vspace{0.25cm}
We began by measuring the ice shelf thickness and radio transparency in Moore's Bay, Antactica \cite{icrc}.  We deployed prototype ARIANNA stations in two separate missions.  I designed systems that managed station power consumption and recorded environmental data \cite{10.1109/tns.2015.2468182} \cite{10.1016/j.nima.2010.09.032}.  We demonstrated with computer simulations that a 30 x 30 km$^2$ array reached target UHE-$\nu$ sensitivity.  Further, the sensitivity doubled in Moore's Bay through \textit{reflected} events, in which the RF signal reflects from the ice-ocean boundary beneath the ice shelf.  We completed the Hexagonal Radio Array prototype in Moore's Bay, and published upper limits on the UHE-$\nu$ flux \cite{10.1016/j.astropartphys.2015.04.002}.  We also observed cosmic rays \cite{cr} (though we cannot determine their original direction), and completed a second UHE-$\nu$ search \cite{4_5}.  UHE-$\nu$ interact more rarely in dense matter, and thus the flux is lower than that of the cosmic rays.
\\
\vspace{0.25cm}
As a post-doctoral fellow at the University of Kansas, I published the first complete analysis of the ice in Moore's Bay \cite{10.3189/2015jog14j214}. I also published the first complete calibration of the ARIANNA RF chain \cite{10.1016/j.astropartphys.2014.09.002}. Using the results, we showed simulations of our detector accurately modeled Askaryan signal strength.  We created UHE-$\nu$ signal \textit{template waveforms} that account for both the theoretical Askaryan signal and the aforementioned calibration.  These templates now serve as the primary UHE-$\nu$ search criterion when cross-correlated with data collected in Antarctica \cite{10.1016/j.astropartphys.2015.04.002} \cite{4_5}.  As a CCAPP Fellow at The Ohio State University\footnote{Center for Cosmology and Astro-Particle Physics}, I improved upon the templates by developing a purely analytic theory of Askaryan pulses \cite{10.1016/j.astropartphys.2017.03.008}.
\\
\vspace{0.25cm}
Once I joined Whittier College, I turned my attention to the complex path taken by radio pulses through Antarctic ice.  The path is curved because the speed depends on the ice density, which in turn changes with depth.  I worked out ray-tracing solutions predicting the ray-path of RF signals through ice with the observed index of refraction \cite{Barwick:2018497}.  These calculations became a central component of our current state-of-the-art simulation software that we use to predict the sensitivity of our systems \cite{10.1140/epjc/s10052-020-7612-8} \cite{10.1140/epjc/s10052-019-6971-5}.  Meanwhile, another student and I designed firmware upgrades to auto-calibrate the RF channel thresholds and presented the results at SCCUR, twice \cite{sccur1} \cite{sccur2}.  These tools facilitate expansion and automation of our detector.  The pandemic has prevented deployment of these upgrades in ARIANNA, but they will be incorporated in IceCube Gen2.  For both projects, I included undergraduate students.  The first was a young lady who went on to become a physics researcher for the LIGO project (gravity waves).  The second was a student of color and Whittier native who majored in ICS/Math, and who is applying to graduate schools for engineering.
\\
\vspace{0.25cm}
I recently returned to the theory of Askaryan radiation, and have begun to study computational electromagnetism (CEM).  For the first time, I have created an analytic time-domain model of Askaryan radiation \cite{time}.  We are happy to report that the work will be published in Physical Review D.  This achievement was made possible by a collaboration with a wonderful undergraduate student who has become a good friend over the past two years.  I describe the importance of this result in Sec. \ref{sec:scholarship}.  Regarding CEM, I have won two Summer Faculty Research Fellowships with the Office of Naval Research (ONR), in which we apply CEM to radar design.  This is an example of the liberal arts mindset in action: I was able to identify a connection between two seemingly unconnected fields, and form a mutually beneficial partnership.
\\
\vspace{0.25cm}
Using CEM, my student and I have created a 3D printed radar antenna design \cite{10.1016/j.cpc.2009.11.008}.  Knowing that Whittier College cannot afford to subscribe to every IEEE engineering journal, I selected an open-access journal named Electronics Journal so that our students have access to the research.  I view choices like these as part of our mission to foster equity and inclusion.  Our paper won Top 10 Most Notable Papers in the Electronics Journal for 2020-21 (see Sec. \ref{sec:scholarship}).  Recently, my colleagues at the naval laboratory fabricated the 3D printed design, and they have provided powerful lab equipment to Whittier College for testing it.  It is worth mentioning that this equipment is prohibitively expensive, and thus our partnership with my naval colleagues is opening new doors scientifically.  If we succeed, this research has applications to UHE-$\nu$ physics (by creating new and better antennas), 5G communication, and radar applications.  I describe in Sec. \ref{sec:scholarship} my vision for a partnership with the Navy, and how this will benefit our students.
\\
\vspace{0.25cm}
Finally, I would like to share with you my recent venture into the Whittier Scholars Program (WSP).  A student heard of my scholarship regarding Antarctica, and sat down in my office one afternoon.  He showed me photographs of glaciers he had taken while visiting family in Norway, and said that he'd like to perform a comparative photographic analysis with historical photos of glaciers all over the world, in order to assess the loss of ice due to global warming.  We lept into a partnership that sent him to Norway, Iceland, Alaska, and the National Outdoor Leadership School (NOLS).  He began by taking one of my new \textit{Connections 2} courses about the history and current status of science in Antarctica.  The research was at the intersection of glaciology, physics, climate science, and environmental social justice.
\\
\vspace{0.25cm}
The work came together as my student gained experience living in the field.  I did everything in my power to get him added to one of my Antarctic expeditions, but alas, that particular mission was canceled due to budget and the pandemic.  We had hoped to include photos of the glaciers near ARIANNA.  These are the same glaciers on the Ross Ice Shelf passed by Robert Falcon Scott and Roald Amundsen as they raced for the discovery of the South Pole.  I have been there twice and taken photos, but there were no similar photos to which we could compare.  I helpd my student gain admission to the WSP, and our final project gave a holistic view of the environmental, agricultural, and cultural impact of glaciers around the world.  My student, who graduated this Spring, told me that he is beginning a book with colleagues he met in Iceland, and that this book will feature our work.  I enjoyed the project so much I have decided to help support WSP by serving on the WSP Advisory Board.  My offer was accepted and I will begin this semester.  Thus, I have come full circle regarding the FPC invitation to serve in the liberal arts.  I will share more on this in Sec. \ref{sec:service}.

\subsection{My Family, East Los Angeles, and COVID-19}
\label{sec:family}

When I first came to Whittier, I lived down the street from campus on Bright Avenue.  I had met a wonderful young woman and we fell in love.  In the summer of 2019 we married, and I moved to East Los Angeles to live with her family.  My wife's family is quite extraordinary.  Her family is originally from Jalisco, Mexico.  The family immigrated to Los Angeles, and dealt with gangs and poverty where they originally lived.  My wife and all six of her siblings worked hard in school and went to college.  We strongly value higher education in our family.  My wife and I share our Catholic faith, and we care about our children's education.
\\
\vspace{0.25cm}
Even before I met my spouse, I knew that becoming fluent in Spanish would be helpful living in Whittier.  I already spoke a little, and in my first year joining the Whittier community I decided to formalize my Spanish skill by taking Spanish 120. Prof. Doreen O'Conner-G\'{o}mez was kind enough to let me audit her course that Fall.  She remarked that this was the first time she had seen a STEM professor audit a language course.  It turned out to be wonderfully necessary in my family, because our older generation usually does not speak English at home.  Our family as a whole is highly diverse, with Mexican, Romanian, American, and Filipino roots.  Given the dark and divisive trends that have arisen within our broader culture, as a Christian and someone who considers himself a loving person, I felt a genuine desire to share this with you.
\\
\vspace{0.25cm}
In my first years as part of the Whittier community, I recognized the same diversity in the families of my students.  Many of our students speak Spanish at home with their parents, but English at school.  There have been times when I have helped the mother or guardian of a student navigate campus by speaking Spanish, and it has made them more comfortable.  A short glance at the Factbook compiled by our institutional research staff reveals two important numbers about our students.  About seventy percent of our students are students of color, and about forty percent are first-generation.  My spouse and every single one of her siblings are all first-generation students.  Back when we were teaching in person, sometimes colleagues would explain to me over lunch about ``the first-generation experience,'' assuming that the white male physicist across from them was new to the concept.  I would always smile inwardly, since my entire family has shared this experience with me.
\\
\vspace{0.25cm}
Given these experiences, I am keenly aware of the importance of our curricular theme of \textit{belonging.}  Despite the challenges brought upon Whittier by the pandemic, I have put effort into making that theme a reality.  I socialize on Zoom with my first-year advisees and research students in order to make them all feel that they belong.  I ensure that I account for equity and inclusion in each decision I make.  One stark example was when a student in my section of INTD100 connected to class via Zoom \textit{while at work in CostCo.}  I learned to arrange my schedule to account for students' jobs, knowing that many were supporting themselves and loved ones.  Being inflexible would have just selected the wealthy students to participate, thus introducing inequity.  Regarding belonging, I am often reminded of a basic fact: \textit{even though my heritage is different from my family and my community, they have accepted me as one of their own.}  In the Gospels, we find the Golden Rule to treat others as we would like to be treated.  I am called therefore to ensure the students feel that they belong.

\subsubsection{Inspiration for New Courses}

Inspired by my family, and the theme of belonging, I have created two new courses that serve our current liberal arts curriculum in the time since my last full PEGP.  One is entitled \textit{A History of Science in Latin America,} which was assigned a \textit{Culture 3} and \textit{Connections 2} designation.  I could tell there was a hunger for a course like this.  Our students needed to see that \textit{all} of our ancestors performed science.  One aspect of that course was the idea of \textit{central} and \textit{peripheral} sources of science.  Taking STEM courses alone might lead a student to believe that European and American cities have been central to scientific progress from the Enlightenment onward, and that Latin American communities were \textit{peripheral}.  Peripheral, in this sense, refers to a community that merely adopts discoveries from the central ones and produces few of its own discoveries.  By honestly covering the colonial period in Latin America, we found examples in which Latin American communities were \textit{central} and their European counterparts were \textit{peripheral.}  In fact, a more accurate description of scientific progress in Latin America would be a full two-way exchange of knowledge (Sec. \ref{sec:teaching}).  I invited colleagues from the Wardman Library to introduce our students to digital storytelling during class time, and the students used this experience to create final projects that wove together topics such as their cultural heritage, history, mathematics, and scientific discoveries.
\\
\vspace{0.25cm}
The second liberal arts course I have created that was inspired by the theme of belonging, and my research, is entitled \textit{Safe Return Doubtful: History and Current Status of Modern Science in Antarctica.}  At first glance, it might be difficult to discern the connection between themes like inclusion and belonging and Antarctica.  This course is a metaphor for self-exploration.  We address three main areas, interwoven throughout the semester.  First, we address the history of the race to discover the South Pole in the early 20th century.  Second, we cover current scientific endeavors in Antarctica.  Third, we perform weekly journal activities that invite the students to look inside themselves and to discover their potential for exploration.  All three areas are connected to each other, and to the theme of belonging.  What we find when examining who raced to be first to the South Pole was that the winner of the race was a person who took indigenous science seriously.  It turns out this was the same person who completed the Northwest Passage.  While completing the Northwest Passage, the explorers encountered the \textit{Netsilik} people.  Rather than assuming there was no new knowledge to be gained from them, the European explorers observed how the Netsilik \textit{used physics and engineering strategies} to travel through the harsh environment.  These strategies were adapted to the Ross Ice Shelf (where one of our detectors is located), and \textit{that group won the race to the South Pole.}  Thus the course connects back to the central theme of inclusion and belonging, as it pertains to survival and exploration.  The students learned that their survival in new areas of life will be enhanced if they are willing to include and analyze knowledge earned by those \textit{indigenous} to that area.

\subsubsection{Keeping a Sense of Humor under Quarantine}

4. Being neighbors with students and staff
5. Getting through the pandemic

\subsubsection{Growing our Family during the Pandemic}

A

\subsubsection{Working from Home with Children}

A

\subsubsection{Tenacity and Adaptation}

A

\section{Progress at Whittier College, and Scope of this Report}

\end{document}
