\documentclass[../../main.tex]{subfiles}
 
\begin{document}

\section{Beginning}

Friends,
\\
\vspace{0.25cm}
I have compiled a report on my progress as a liberal arts educator in the Department of Physics and Astronomy for the period of 2019 through 2021.  The following is a reflection on the development of my educational and scholarly practices, and the service I have performed for the College as a mentor, advisor, and committee member.  I strive to perfect my teaching abilities, and I am pleased to report that my students are learning and growing.  In our last communication, after my supplemental PEGP from 2019, you concurred that my practices are serving the students well, and that meant a great deal to me.  One note that stood out was a request to enrich my teaching philosphy by reflecting on how it serves the liberal arts.  The given example was about the utility of physics to non-STEM students.  I have put a lot of thought into this enrichment, I have progress to share with you.
\\
\vspace{0.25cm}
I have included in my teaching philosphy (Sec. \ref{sec:teaching_philosophy}) my vision for the intersection of broader liberal arts education with physics, mathematics, computer science, and engineering, as I teach all of these.  Further, I have created and taught new liberal arts courses in the \textit{Connections 2} and \textit{Culture 3} categories, as well as a College Writing Seminar on scientific and technical writing.  I used these courses to show our students how physics, mathematics, and engineering intersect with the history of our ancestors and how we all use scientific modes of thought to thrive.  In my College Writing Seminar, we sharpened the skills of conciseness, precision and clarity, and organization in writing.  Though these skills apply to physics, they are useful in all writing in which abstract or difficult ideas are communicated.
\\
\vspace{0.25cm}
At a PEGP workshop, a colleague emphasized weaving a narrative about who we are.  I have shared my academic origins, and my vision for multi-disciplinary teaching and scholarship at Whittier College (Sec. \ref{sec:origin}).  I have also written about my family and working during the pandemic (Sec. \ref{sec:family}).  I hope these sections provide you with helpful insight about our family and how we hope to serve Whittier College.  We thank the FPC in advance for what is sure to be a difficult year of service.  We also express our gratitude for allowing us to postpone the PEGP for one year.  This helped my family avoid a difficult situation.  My spouse is considered an essential worker, and was called back after a shortened maternity leave.  I have been working as a full-time professor and a full-time parent \textit{by myself} each day since that happened.  Thankfully, my \textit{suegra} (mother-in-law) has stepped in to help after we got vaccinated.  I look forward to seeing you this Fall, and from our family to yours, we hope you are well.
\\
\vspace{0.25cm}
Sincerely,
Prof. Jordan C. Hanson

\section{Academic Origins}
\label{sec:origin}

We all share a common theme as professors, in that we encounter ideas that inspire us in college and graduate school.  When I was an undergraduate at Yale University, my family was nudging me towards engineering.  My curiousity, however, kept returning me to physics courses.  In my heart, I knew that I wanted to offer fresh discoveries about the Universe to people, and I fell in love with enlightening others.  I learned that the laws of physics morph and merge into one another as the energy of matter increases to relativistic scales where particles move near the speed of light.  I also learned that deep from within the Universe originates a mysterious flux of sub-atomic particles ten thousand times more energetic than any human has ever created: \textit{the cosmic rays}.  The physics that explains their origin has remained unknown for a over a century, and it could reveal new fundamental laws of Nature.  I applied for graduate school in the hopes of one day becoming a professor of physics.
\\
\vspace{0.25cm}
The University of California at Irvine (UCI) is a pioneering institution in the field of cosmic-ray research.  In particular, my colleagues at UCI began focusing on the study of cosmic ray \textit{neutrinos}, also known as ultra-high energy neutrinos (UHE-$\nu$), beginning with the Radio Ice Cherenkov Experiment (RICE) \cite{PhysRevD.85.062004} and Antartic Impulsive Transient Antenna (ANITA) \cite{PhysRevD.99.122001} collaborations. Neutrinos do not have electric charge, while cosmic rays do.  Neutrinos propagate in straight lines through the Universe, while any electromagnetic field would bend the trajectory of the charged cosmic rays, due to the Lorentz and Coulomb forces.  Thus, UHE-$\nu$ could reveal the locations of the cosmic ray accelerators, thereby teaching us about fundamental physics unexplored on Earth \cite{Astro2020_1} \cite{Astro2020_2}.  UHE-$\nu$ observations would provide insight into quantum mechanics at record-breaking energies \cite{PhysRevD.83.113009}.
\\
\vspace{0.25cm}
Detection of UHE-$\nu$ has been a goal of the physics community for three decades.  When UHE-$\nu$ have energies above a certain threshold, they create cascades of particles in matter that radiate in the radio-frequency (RF) bandwidth, a process known as the Askaryan effect \cite{ask1} \cite{ask2} \cite{PhysRevLett.86.2802} \cite{PhysRevLett.99.171101} \cite{PhysRevD.101.083005}.  The IceCube Collaboration published the observations of extra-solar neutrinos using optical techniques at record-breaking energies \cite{PhysRevLett.111.021103}, and later showed that the flux is strikingly close to theoretical predictions \cite{Aartsen_2015} \cite{WB}.  IceCube analyses have not found UHE-$\nu$ events, however, with energy greater than $10^{15}$ electron-volts \cite{PhysRevD.98.062003}.  It is above this energy that the UHE-$\nu$ could one day reveal the source of cosmic rays and new physics.  The authors of \cite{PhysRevD.98.062003} conclude that Askaryan-class detectors are the logical next step.  My colleagues have decided to upgrade IceCube to include RF detectors in a project known as IceCube Generation 2, or IceCube Gen2 (\url{https://icecube.wisc.edu/}).
\\
\vspace{0.25cm}
Askaryan-class detectors improve UHE-$\nu$ prospects because Askaryan radiation is in the RF bandwidth \cite{10.1016/j.astropartphys.2017.03.008}.  UHE-$\nu$ must strike some material in the Earth's crust that produces an observable radio pulse.  It turns out that radio waves travel $\approx 1$ km in \textit{Antarctic ice} \cite{10.3189/2015jog14j214} \cite{10.3189/2015jog15j057} \cite{barwick_besson_gorham_saltzberg_2005}.  Thus, we can create stations comprised of RF antenna channels, supporting electronics, and solar panels \cite{10.1109/tns.2015.2468182}, separated by 1 km to cover enormous volumes of ice.  This is important because the expected UHE-$\nu$ flux is low.  When a potential signal arrives, stations are triggered to read out the RF channel data \cite{sst}.  The overall dataset is then comprised of RF waveforms representing signals from all the stations.  The data will be used to reconstruct UHE-$\nu$ interactions \cite{10.1088/1475-7516/2019/11/030} \cite{10.1088/1748-0221/15/09/p09039}.  This type of detector is called an \textit{in-situ} array.  As a graduate student at UCI, I led two Antarctic expeditions to create a prototype \textit{in-situ} array: the Antarctic Ross Ice Shelf Antenna Neutrino Array (ARIANNA).
\\
\vspace{0.25cm}
We would leave Los Angeles International airport, and land in Auckland, New Zealand.  After collecting our gear, we transferred to the local airline to fly to Christchurch, on the southern island of New Zealand.  Usually there would be a week of down time in a bed and breakfast, while we waited for the weather in the Southern Ocean to clear.  We would equip our cold weather gear, tinker with our equipment, and read.  At the right moment, we filed on to an Army C-17 cargo plane that had both wheels and skiis for landing gear.  After a five-hour flight, we would land on the frozen ocean near Ross Island, not far from the berth of the original Antarctic explorers.  A truck brought us in to McMurdo Station, the home of the United States Antarctic Program (USAP).  There we would receive safety training and logistics to prepare for our expedition into the field.  A helicopter would transport us to ARIANNA.  Our gear, which arrived by boat in New Zealand, and then by plane to McMurdo, followed us in a separate helicopter.
\\
\vspace{0.25cm}
We began by measuring the ice shelf thickness and radio transparency in Moore's Bay, Ross Ice Shelf, Antactica \cite{icrc}.  We deployed prototype ARIANNA stations in two separate missions.  I designed systems that managed station power consumption and recorded environmental data \cite{10.1109/tns.2015.2468182} \cite{10.1016/j.nima.2010.09.032}.  We demonstrated with computer simulations that a 30 x 30 array would reach target UHE-$\nu$ sensitivity.  Further, the sensitivity doubled in Moore's Bay through \textit{reflected} events, in which signals reflect from the ocean beneath the ice shelf.  We completed the seven-station prototype array, and published upper limits on the UHE-$\nu$ flux \cite{10.1016/j.astropartphys.2015.04.002}.  We also observed cosmic rays \cite{cr} (though we cannot determine their original direction), and completed a second UHE-$\nu$ search \cite{4_5}.  UHE-$\nu$ interact more rarely in dense matter, and thus the visible flux is lower than that of cosmic rays.  With much larger number of stations in IceCube Gen2, we hope to capture the precious UHE-$\nu$ signals.
\\
\vspace{0.25cm}
As a post-doctoral fellow at the University of Kansas, I published the first complete analysis of the ice in Moore's Bay \cite{10.3189/2015jog14j214}.  This research was an intersection of glaciology and physics, for we need to understand our detector and our detector is an ice shelf.  I also published the first complete calibration of the ARIANNA RF chain \cite{10.1016/j.astropartphys.2014.09.002}. Using the results, we showed simulations of our detector accurately modeled Askaryan signal strength.  We created UHE-$\nu$ signal \textit{template waveforms} that account for both the theoretical Askaryan signal and the aforementioned calibration.  These templates now serve as the primary UHE-$\nu$ search criterion when cross-correlated with data collected in Antarctica \cite{10.1016/j.astropartphys.2015.04.002} \cite{4_5}.  As a CCAPP Fellow at The Ohio State University\footnote{Center for Cosmology and Astro-Particle Physics}, I improved upon the templates by developing a new analytic theory of Askaryan pulses \cite{10.1016/j.astropartphys.2017.03.008}.
\\
\vspace{0.25cm}
Once I joined Whittier College, I turned my attention to the complex path taken by Askaryan pulses through Antarctic ice.  The path is curved because the speed of light depends on the ice density, which changes with depth.  An undergraduate student and I worked out solutions for the ray-path of the signals through ice given the density profile \cite{Barwick:2018497}.  These calculations became a central component of our current software that we use to predict detector sensitivity to UHE-$\nu$ \cite{10.1140/epjc/s10052-020-7612-8} \cite{10.1140/epjc/s10052-019-6971-5}.  Meanwhile, another student and I designed firmware upgrades to auto-calibrate the RF channel thresholds and presented the results at SCCUR, twice \cite{sccur1} \cite{sccur2}.  These tools facilitate expansion and automation of our detector.  The pandemic has prevented deployment of these upgrades in ARIANNA, but they will be incorporated in IceCube Gen2.  For both projects, I included undergraduate students.  The first was a young lady who went on to become a physics researcher for the LIGO project (gravity waves).  The second was a student of color and Whittier native who majored in ICS/Math, and who is applying to graduate schools for engineering and machine learning.
\\
\vspace{0.25cm}
I recently returned to the theory of Askaryan radiation, and have begun to study computational electromagnetism (CEM).  For the first time, I have created an analytic time-domain model of Askaryan radiation \cite{time}.  We are happy to report that the work will be published in Physical Review D, and that it will be incorporated into IceCube Gen2 software.  This achievement was made possible by a collaboration with a wonderful undergraduate student who has become a good friend over the past two years amid the pandemic.  I describe the importance of this result in Sec. \ref{sec:scholarship}.  Regarding CEM, I have won two Summer Faculty Research Fellowships with the Office of Naval Research (ONR), in which we apply CEM to radar design.  This is an example of the liberal arts mindset in action: I was able to identify a connection between two seemingly unconnected fields, and form a mutually beneficial partnership.
\\
\vspace{0.25cm}
Using CEM, my student and I have created a 3D printed radar design \cite{10.1016/j.cpc.2009.11.008}.  Knowing that Whittier College cannot afford to subscribe to every IEEE engineering journal, I selected an open-access journal named Electronics Journal so that our students have access to the research.  I view choices like these as part of our mission to foster equity and inclusion.  Our paper won Top 10 Most Notable Papers in the Electronics Journal for 2020-21 (see Sec. \ref{sec:scholarship}).  Recently, my colleagues at the naval laboratory fabricated the 3D printed design, and they have provided powerful lab equipment to Whittier College for testing it.  It is worth mentioning that this equipment is prohibitively expensive, and thus our partnership with my naval colleagues is opening new doors scientifically.  If we succeed, this research has applications to UHE-$\nu$ physics (by creating new and better antennas), 5G communication, and radar applications.  I describe in Sec. \ref{sec:scholarship} my vision for a partnership with the Navy, and how this will benefit our students.
\\
\vspace{0.25cm}
Finally, I would like to share with you my recent venture into the Whittier Scholars Program (WSP).  A student heard of my scholarship regarding Antarctica, and sat down in my office one afternoon.  He showed me photographs of glaciers he had taken while visiting family in Norway, and said that he'd like to perform a comparative photographic analysis with historical photos of glaciers all over the world, in order to assess the loss of ice due to global warming.  We lept into a partnership that sent him to Norway, Iceland, Alaska, and the National Outdoor Leadership School (NOLS).  He began by taking one of my new \textit{Connections 2} courses about the history and current status of science in Antarctica.  The research was at the intersection of glaciology, physics, climate science, and environmental social justice.
\\
\vspace{0.25cm}
The work came together as my student gained experience living in the field.  I did everything in my power to add him to one of my Antarctic expeditions.  Alas, that particular mission was cancelled due to budget cuts and the pandemic.  We had hoped to include photos of the glaciers near ARIANNA.  These are the same glaciers passed by Robert Falcon Scott and Roald Amundsen as they raced for the discovery of the South Pole.  I have been there twice and taken photos, but there were no similar photos to which we could compare.  I helpd my student gain admission to the WSP, and our final project gave a holistic view of the environmental, agricultural, and cultural impact of glaciers around the world.  My student, who graduated this Spring, told me that he is beginning a book with colleagues he met in Iceland, and that this book will feature our work.  I enjoyed the project so much I have decided to help support WSP by serving on the WSP Advisory Board.  My offer was accepted and I will begin this semester.  Thus, I have come full circle regarding the FPC invitation to serve in the liberal arts.  I will share more on this in Sec. \ref{sec:service}.

\section{My Family, East Los Angeles, and COVID-19}
\label{sec:family}

When I first came to Whittier, I lived down the street on Bright Avenue.  I had met a wonderful young woman and we fell in love.  In the summer of 2019 we married, and I moved to East Los Angeles to live with her family.  My wife's family is quite extraordinary.  Her family is originally from Jalisco, Mexico.  The family immigrated to Los Angeles, and dealt with gangs and poverty where they originally lived.  My wife and all six of her siblings worked hard in school and went to college.  We strongly value higher education in our family.  My wife and I share our Catholic faith, and we care about our children's education.
\\
\vspace{0.25cm}
Even before I met my spouse, I knew that becoming fluent in Spanish would be helpful living in Whittier.  I already spoke a little, and in my first year joining the Whittier community I decided to formalize my Spanish skill by taking Spanish 120. Prof. Doreen O'Conner-G\'{o}mez was kind enough to let me audit her course that Fall.  She remarked that this was the first time she had seen a STEM professor audit a language course.  It turned out to be wonderfully necessary in my family, because our older generation usually does not speak English at home.  Our family as a whole is highly diverse, with Mexican, Romanian, American, and Filipino roots.  Given the dark and divisive trends that have arisen within our broader culture, and as a Christian and someone who considers himself a loving person, I felt a genuine desire to share this with you.
\\
\vspace{0.25cm}
In my first years as part of the Whittier community, I recognized the same diversity in the families of my students.  Many of our students speak Spanish at home with their parents, but English at school.  There have been times when I have helped the mother or guardian of a student navigate campus by speaking Spanish, and it has made them more comfortable.  As part of a statistics course I taught for the Whittier Summer Session II (2020), I was gathering data from the Whittier College Factbook.  It reveals two important numbers about our students.  About seventy percent of our students are students of color, and about forty percent are first-generation.  My spouse and every single one of her siblings are all first-generation students.  Back when we were teaching in person, sometimes colleagues would explain to me over lunch about ``the first-generation experience,'' assuming that the white male physicist was new to the concept.  I would always smile inwardly, since my entire family has shared this experience with me.
\\
\vspace{0.25cm}
Given these experiences, I am keenly aware of the importance of our curricular theme of \textit{belonging.}  Despite the challenges brought upon Whittier by the pandemic, I have put effort into making that theme a reality.  I socialize on Zoom with my first-year advisees and research students in order to make them all feel that they belong.  I ensure that I account for equity and inclusion in each decision I make.  One stark example was when a student in my section of INTD100 connected to class via Zoom \textit{while at work in CostCo.}  I learned to arrange my schedule to account for students' jobs, knowing that many were supporting themselves and loved ones.  Being inflexible would have made class accessible only for the wealthy students.  Regarding belonging, I am often reminded of a basic fact: \textit{even though my heritage is different from my family and my community, they have accepted me as one of their own.}  In the Gospels, we find the Golden Rule to treat others as we would like to be treated.  I am called therefore to ensure the students feel that they belong.

\subsection{Inspiration for New Courses}

Inspired by my family, and the theme of belonging, I have created two new courses that serve our current liberal arts curriculum.  One is entitled \textit{A History of Science in Latin America,} which was assigned a \textit{Culture 3} and \textit{Connections 2} designation.  I could tell there was a hunger for this course.  Our students needed to see that \textit{all} of our ancestors performed science.  One aspect of that course was the ideas of \textit{central} and \textit{peripheral} scientific communities.  Taking STEM courses alone might lead a student to believe that European and American cities have been central to scientific progress, and that Latin American communities have been \textit{peripheral}.  Peripheral, in this sense, refers to a community that merely adopts discoveries from the central ones and rarely produces progress.  By honestly covering the colonial period in Latin America, we found examples in which Latin American communities were \textit{central} and their European counterparts were \textit{peripheral.}  A more accurate description of scientific progress for Latin America and Europe would be a full two-way exchange of knowledge (Sec. \ref{sec:teaching}).  I invited colleagues from the Wardman Library to introduce my students to digital storytelling.  The students used this skill to create final projects that wove together their cultural heritage, history, mathematics, and scientific discovery.
\\
\vspace{0.25cm}
The second liberal arts course I have created was inspired by the theme of belonging, and my research.  It is called \textit{Safe Return Doubtful: History and Current Status of Modern Science in Antarctica.}  At first glance, the connection between themes like inclusion and belonging and Antarctica is not obvious.  This course is a metaphor for self-exploration.  We address three main areas, interwoven throughout the semester.  First, we address the history of the race to discover the South Pole in the early 20th century.  Second, we cover current scientific endeavors in Antarctica.  Third, we perform weekly journal activities that invite the students to look inside themselves and to discover their potential for exploration.  The connection to inclusion and belonging emerges as we learn that the winner of the race for the South Pole was a person who took indigenous science seriously.  This was the same captain who completed the Northwest Passage, before trying the South Pole.  I share the rest of the story in Sec. \ref{sec:teaching}. Thus the course connects inclusion and belonging to survival and exploration.  The students learn through history, science, and self-reflection that their survival in new areas of life will be enhanced if they are willing to learn from those different from themselves.

\subsection{Keeping a Sense of Humor under Quarantine}

My fellow tenure-track colleagues and I sometimes discuss if it's appropriate to add a ``COVID-19 Impact Statement'' to our PEGP.  At first I thought, no, just stick to the formal stuff.  Keep it short.  But I also thought it would be a laugh.  So here goes.  For those with a sense of humor, this next section is for you.  
\\
\vspace{0.25cm}
For those without a sense of humor, I have to ask, like, how are you still here?  After we duct-taped together a way to teach our students online in spring 2020, we watched the world lose its freaking mind that summer\footnote{After watching \textit{I Am Not Your Negro}, directed by Raoul Peck, and reading \textit{Notes of a Native Son} by James Baldwin, maybe it's more apt to say that we should have lost our minds sooner. I ended up making a chapter of Notes of A Native Son the summer reading assignment for my College Writing Seminar.}.  Then, the module system, for \textit{a year.}  But hooray!  The vaccines arrived.  Funny thing about vaccines, though, is that you have to go \textit{get them.}  Ugh, who has time, right?  Seriously, a few people in my family flatly refuse.  Here's a fun exercise: try teaching a science course on Zoom and hearing your phone buzzing from an argument about how \textit{the scientists are wrong} in Spanish.  Focus, focus ... \textit{just get the blasted shot already} ... ``Ok students, let's talk about ... friction!  Am I right?''  Ay yay yay.
\\
\vspace{0.25cm}
Teaching students remotely was like watching those YouTube channels where people crash into stuff.  What I mean is, students would log in to class while driving.  Had to make a rule against that.  Everyone survived, but ... wow.  I thought one of my students was driving on the wrong side of the road while Zooming, but it's ok, he was in India.  Another rule I had to make for class: wear clothes.  I don't wanna see that.  This is a family establishment.  After 2.1 seconds of research, I found the Zoom button to have the camera off by default.
\\
\vspace{0.25cm}
My spouse always says that she has the worst luck.  I always reassure her: ``Have faith honey.  We'll be alright.''  After years of searching, we find each other, marry, and then BAM.  Pandemic.  Whoops.  Our daughter was born right in the exact middle (like, literally within the error of the mean) of the first wave of COVID-19.  Whyyy.  The nurses weren't going to let me in the hospital.  For the birth of my child.  What.  Actually they weren't really that keen on letting my wife in either.  \textit{Just hang out in the parking garage}, they told my wife, \textit{who was in labor.}  After the required amount of suffering took place, they let us come inside to give birth.
\\
\vspace{0.25cm}
Sometimes I had to fight for my students.  We made sure to purchase enough bandwidth from Spectrum, but sometimes it still felt choppy.  After running some checks, I realized I needed to call Spectrum.  They told me, ``What do you expect?  It's slow all over the city.''  I got all idealistic: ``This is about access to education!  For first-generation students no less!'' They finally sent a techician, who came to our door holding small metal piece.  ``Did you know there was a 3 dB attenuator in your coaxial line?''  I half-choked on my coffee.  I knew that the ONLY JOB of this component is to cut signal power by a factor of two.  Whyyy.  So we removed the Education Blocking Device, and the signal was greatly improved.
\\
\vspace{0.25cm}
Working from home during Summer 2020 did make it easier to care for our daughter.  We were quarantined, but I managed to convince the ONR that we could perform the research project remotely.  My spouse is a dentist, and the state provided maternity leave.  For added spice, they took it away, though, after 12 weeks.  Right at the beginning of Fall semester.  No vaccines were available yet, so I had to just teach and parent alone.  The upside is that I got to spend more time with the baby.  My spouse bravely went back to work to help pay her student loans.  She treats patients who are supposed to test negative for COVID-19, but the positives sneak past the guard.  So basically The Hunger Games for dentists, who tend to be around a lot of, you know, mouths and throats.
\\
\vspace{0.25cm}
Here's another fun exercise: try teaching college-level physics with a six-month-old pooping in your lap.  Keep composure.  Another one: the baby is napping in her seat, and all is quiet and ready for class.  My chihuahua looks at me like he wants to bark at the dogs outside.  \textit{Don't you do it, Lobo!  I swear...} Does it anyways.  So I trained him not to bark, but my neighbors responded by buying a rooster.  They. Bought. A. Rooster.  In the city.  Not chickens!  Chickens I would understand for the eggs.  We love eggs.  Fun fact about cities: they sell alarm clocks.  Quirky thing about roosters: they don't have a snooze.  This particular rooster was peculiar though, because it would crow twice.  Once at dawn, and once whenver my baby was napping during class.  Bad rooster.
\\
\vspace{0.25cm}
Despite its quirks, we citizens of East Los Angeles take great pride in maintaining the community.  Except on 4th of July.  Then we blow it to smithereens.  As far as I can tell, the goal is to trigger as many car alarms as you can.  On our block the record is twelve.  I mean, why spend your money on ``real'' fireworks when you can just fill a rice-cooker with black powder?  It's easy.  Nothing wakes you up from grading math homework from summer session like shrapnel.
\\
\vspace{0.25cm}
Joking aside, my students were wonderfully understanding when I had to teach with the baby on my lap.  The same is true for my colleagues in committee meetings.  I like to think she brightened people's day a little.  A colleague from another institution who gave birth recently lamented to me that it has been \textit{so hard} lately, for she and her husband (both physicists working from home) hadn't had child care in \textit{six whole weeks!}  I died a little inside.  It had been almost nine months for me flying solo.  Once we all got vaccinated, my \textit{suegra} (mother-in-law) who lives next door, started to come each day to help.  Que santa, no? (What a saint, no?).  My family has supported us, and we are so grateful.
\\
\vspace{0.25cm}
I hope you are all safe and sound.  It turns out that some people in my extended family in the Midwest were not so lucky.  My cousin's husband, who was a well-loved football coach and mentor to many junior college students, already needed a lung transplant before COVID-19 arrived.  He finally got the lung transplant a few years ago and recovered.  And then someone gave him the virus, and he's gone now, along with his father.  We pray for them each night now.  If any of you have lost loved ones, we will pray for them as well.  I've had students miss class to go to funerals.  Even though we have all suffered, I still believe it's important to keep a smile on.  We are still here, and Whittier College needs us.  Soon the students will return, and Whittier College will grow and thrive.

\end{document}
