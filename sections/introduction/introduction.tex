\documentclass[../../main.tex]{subfiles}
 
\begin{document}

\section{Beginning}

Friends,
\\
\vspace{0.15cm}
I have compiled a report on my progress as a liberal arts educator for the period of 2019-2021.  The following is a reflection on the development of my educational and scholarly practices, and the service I have performed for the College as a mentor, advisor, and committee member.  I strive to perfect my teaching abilities, and I am pleased to report that my students are learning and growing.  In our last communication, after my supplemental PEGP from 2019, you concurred that my students are being served well, and that meant a great deal to me.  I understood your suggestion to enrich my teaching by reflecting on how it serves the liberal arts.  The given example was about the utility of physics to non-STEM students.  I have put a lot of thought into this enrichment, I have progress to share with you.
\\
\vspace{0.15cm}
I have included in my teaching philosphy (Sec. \ref{sec:teaching_philosophy}) my vision for the intersection of broader liberal arts education with physics, mathematics, computer science, and engineering, as I teach all of these.  Further, I have created and taught new liberal arts courses in the \textit{Connections 2} and \textit{Culture 3} categories, as well as a College Writing Seminar on scientific and technical writing.  I used these courses to show our students how physics, mathematics, and engineering intersect with the history of our ancestors and how we all use scientific modes of thought to thrive.  In my College Writing Seminar, we sharpened the skills of conciseness, precision and clarity, and organization in writing.  Though these skills apply to physics, they are useful in all writing in which abstract or difficult ideas are communicated.
\\
\vspace{0.15cm}
At a PEGP workshop, a colleague emphasized the importance of weaving a personal narrative.  Thus, I have shared a section about my academic origins and my vision for multi-disciplinary teaching and scholarship (Sec. \ref{sec:origin}).  I have also written about my family and working during the pandemic (Sec. \ref{sec:family}).  I hope these sections prove useful, and I understand that you all are pressed for time.  Thank you in advance for what is sure to be a difficult year of service.  My family is also grateful that we were allowed to postpone the PEGP for one year.  It helped us to avoid a difficult situation.  My spouse is considered an essential worker, and was called back after a shortened maternity leave.  Since that time, have been working as a full-time professor and a full-time parent \textit{by myself}.  Thankfully, my \textit{suegra} (mother-in-law) has stepped in to help after we got vaccinated.  I look forward to seeing you this Fall, and we hope you are well.
\\
\vspace{0.15cm}
Sincerely,
Prof. Jordan C. Hanson

\section{Academic Origins}
\label{sec:origin}

As professors, we all share the experience of being inspired in college and graduate school.  When I was an undergraduate at Yale University, my family nudged me towards engineering.  My curiousity kept returning me to physics.  In my heart, I knew that I wanted to help create discoveries, and I fell in love with enlightening others.  I learned that the laws of physics morph and merge into one another as systems move near the speed of light.  I learned that deep from within the Universe originates a mysterious flux of sub-atomic particles ten thousand times more energetic than any human has ever created: \textit{the cosmic rays}.  The physics of cosmic ray origins has remained unknown for a century, and it could reveal new fundamental laws of Nature.  I applied for graduate school in the hopes of one day becoming a professor of physics.
\\
\vspace{0.15cm}
The University of California at Irvine (UCI) is a pioneering institution in the field of cosmic-ray research.  In particular, my colleagues at UCI study of cosmic ray \textit{neutrinos}, also known as ultra-high energy neutrinos (UHE-$\nu$), beginning with the Radio Ice Cherenkov Experiment (RICE) \cite{PhysRevD.85.062004} and Antartic Impulsive Transient Antenna (ANITA) \cite{PhysRevD.99.122001} collaborations. Neutrinos do not have electric charge, while cosmic rays do.  Neutrinos propagate in straight lines through the Universe, while any electromagnetic field exerts force on the charged cosmic rays, bending the trajectory.  Thus, unchanged UHE-$\nu$ trajectories could reveal the locations of the cosmic ray accelerators, thereby teaching us about unexplored laws of physics\cite{Astro2020_1} \cite{Astro2020_2} \cite{PhysRevD.83.113009}.
\\
\vspace{0.15cm}
Detection of UHE-$\nu$ has been a goal of the physics community for three decades.  When UHE-$\nu$ have energies above a certain threshold, they create cascades of particles in matter that radiate in the radio-frequency (RF) bandwidth, a process known as the Askaryan effect \cite{ask1} \cite{ask2} \cite{PhysRevLett.86.2802} \cite{PhysRevLett.99.171101} \cite{PhysRevD.101.083005}.  The IceCube Collaboration has published the observations of extra-solar neutrinos at energies below the Askaryan threshold that could originate from unique objects near our galaxy \cite{PhysRevLett.111.021103}.  IceCube analyses have not found UHE-$\nu$ events, however, with energy greater than $10^{15}$ electron-volts \cite{PhysRevD.98.062003}.  It is above this energy that the UHE-$\nu$ would reveal the source of cosmic rays and new physics.  The authors of \cite{PhysRevD.98.062003} conclude that Askaryan-class detectors are the logical next step.  We have decided to upgrade IceCube to include RF detectors in a project known as IceCube Generation 2, or IceCube Gen2 (\url{https://icecube.wisc.edu/}).
\\
\vspace{0.15cm}
Askaryan-class detectors improve UHE-$\nu$ prospects because Askaryan radiation is in the RF bandwidth \cite{10.1016/j.astropartphys.2017.03.008}.  UHE-$\nu$ must strike some material in the Earth's crust that produces the radio pulse.  It turns out that radio waves travel $\approx 1$ km in \textit{Antarctic ice} \cite{10.3189/2015jog14j214} \cite{10.3189/2015jog15j057} \cite{barwick_besson_gorham_saltzberg_2005}.  Thus, we can create RF detectors separated by 1 km to cover enormous volumes of ice \cite{10.1109/tns.2015.2468182}.  The volume is necessary because the expected UHE-$\nu$ flux is low.  When a potential signal arrives, stations will capture the radio pulse \cite{sst}.  The dataset will then be comprised of RF waveforms representing signals from all the stations.  The data will be used to reconstruct UHE-$\nu$ interactions \cite{10.1088/1475-7516/2019/11/030} \cite{10.1088/1748-0221/15/09/p09039}.  This type of detector is called an \textit{in-situ} array.  As a graduate student at UCI, I led two Antarctic expeditions to build a prototype \textit{in-situ} array: the Antarctic Ross Ice Shelf Antenna Neutrino Array (ARIANNA).
\\
\vspace{0.15cm}
We began by measuring the ice shelf thickness and radio transparency in Moore's Bay, the location of ARIANNA \cite{icrc}.  We deployed prototype stations in two separate missions.  I designed systems that managed station power consumption and recorded environmental data \cite{10.1109/tns.2015.2468182} \cite{10.1016/j.nima.2010.09.032}.  We demonstrated with simulations that a $30 \times 30$ array would reach target UHE-$\nu$ sensitivity.  Further, the sensitivity doubled in Moore's Bay through \textit{reflected} events, in which RF signals reflect from the ocean beneath the ice shelf.  Over several years, we completed the prototype array, and published upper limits on the UHE-$\nu$ flux \cite{10.1016/j.astropartphys.2015.04.002}.  We observed cosmic rays \cite{cr} (though we cannot determine their original direction), and completed a second UHE-$\nu$ search \cite{4_5}.  Quantum mechanics dictates that UHE-$\nu$ interact more rarely in dense matter, meaning the detectable flux is lower than that of cosmic rays.  With IceCube Gen2, we will increase the number of stations by more than an order of magnitude, and we expect to finally capture the precious UHE-$\nu$ signals.
\\
\vspace{0.15cm}
As a post-doctoral fellow at the University of Kansas, I published the first complete analysis of the ice in Moore's Bay \cite{10.3189/2015jog14j214}.  This research was an intersection of glaciology and physics, for we need to understand our detector and our detector is an ice shelf.  I also published the calibration of the ARIANNA RF chain \cite{10.1016/j.astropartphys.2014.09.002}. Using the results, we showed simulations of our detector accurately modeled Askaryan signal strength.  We created UHE-$\nu$ signal \textit{template waveforms} that account for Askaryan physics and the calibration.  These templates now serve as the primary UHE-$\nu$ search criterion when cross-correlated with data collected in Antarctica \cite{10.1016/j.astropartphys.2015.04.002} \cite{4_5}.  As a CCAPP Fellow at The Ohio State University\footnote{Center for Cosmology and Astro-Particle Physics}, I improved upon the templates by developing a new analytic theory of Askaryan pulses \cite{10.1016/j.astropartphys.2017.03.008}.
\\
\vspace{0.15cm}
When I became a professor, I turned my attention to the complex path taken by Askaryan pulses through Antarctic ice.  The path is curved because the speed of light depends on the ice density, which changes with depth.  In 2017-2018, a student and I found solutions for the ray-path \cite{Barwick:2018497}.  These calculations became one of the four main pillars our current software that computes detector sensitivity to UHE-$\nu$ \cite{10.1140/epjc/s10052-020-7612-8} \cite{10.1140/epjc/s10052-019-6971-5}.  Meanwhile, a student and I designed firmware upgrades for the RF detectors and presented the results at SCCUR, twice \cite{sccur1} \cite{sccur2}.  These tools facilitate automation of our \textit{in situ} array.  The pandemic has prevented deployment, but they can be incorporated in IceCube Gen2.  For both projects, I included undergraduate students.  The first was a young lady who went on to become a researcher for the gravity-wave detector LIGO, and who is now a gradaute student at Yale.  The second was a student of color and Whittier native who majored in ICS/Math, and who is currently applying to graduate schools for engineering and machine learning.
\\
\vspace{0.15cm}
I recently returned to the theory of Askaryan radiation, and have begun to study computational electromagnetism (CEM).  For the first time, I have created an analytic time-domain model of Askaryan radiation \cite{time}.  We are happy to report that the work will be published in Physical Review D, and that it will be incorporated into IceCube Gen2 software.  This achievement was made possible by a collaboration with a wonderful undergraduate student who has become a good friend over the past two years.  I describe the importance of this result in Sec. \ref{sec:scholarship}.  Regarding CEM, I have won two Summer Faculty Research Fellowships with the Office of Naval Research (ONR), in which we apply CEM to radar design.  This is an example of the liberal arts mindset in action: I was able to identify a connection between two seemingly unconnected fields, and form a mutually beneficial partnership.
\\
\vspace{0.15cm}
Using CEM, my student and I have created a 3D printed radar design \cite{10.1016/j.cpc.2009.11.008}.  Knowing that Whittier College cannot afford to subscribe to every IEEE engineering journal, I selected an open-access journal named Electronics Journal so that our students have access to the research.  I view choices like these as part of our mission to foster equity and inclusion.  Our paper won Top 10 Most Notable Papers in the Electronics Journal for 2020-21 (see Sec. \ref{sec:scholarship}).  Recently, my colleagues at the naval laboratory fabricated the 3D printed design, and they have provided powerful lab equipment to Whittier College for testing it.  This equipment is prohibitively expensive, and thus our partnership is opening new doors scientifically.  If we succeed, this research has applications to UHE-$\nu$ physics (by creating new and better antennas), 5G mobile, and radar.  I describe the partnership in Sec. \ref{sec:scholarship}, and how it will benefit our students.
\\
\vspace{0.15cm}
Finally, I would like to highlight my contributions to the Whittier Scholars Program (WSP).  I met with a student who showed me photographs of glaciers he had taken while visiting family in Norway.  He shared an idea to perform a comparative photographic analysis with historical photos of glaciers all over the world.  The goal would be to assess the loss of ice due to global warming.  We lept into a partnership that sent him to Norway, Iceland, Alaska, and the National Outdoor Leadership School (NOLS).  He began by taking my new \textit{Connections 2} course about the history and current status of science in Antarctica (INTD255).  The research was at the intersection of glaciology, climate science, comparitive cultural perspective, and environmental justice.  My student, who graduated this Spring, told me that he is beginning a book with colleagues he met in Iceland, and that this book will feature our work.  I enjoyed the project so much I imquired about serving the WSP Advisory Board, and my offer was accepted.  Thus, I have come full circle regarding the FPC invitation to serve in the liberal arts.

\section{My Family, East Los Angeles, and COVID-19}
\label{sec:family}

When I first came to Whittier, I lived down the street on Bright Avenue.  I had met a wonderful young woman and we fell in love.  In the summer of 2019 we married, and I moved to East Los Angeles to live with her family.  My wife's family is quite extraordinary.  Her family is originally from Jalisco, Mexico.  After immigrating to Los Angeles, they were forced to deal with gangs and poverty.  My wife and all six of her siblings studied hard and graduated college.  My wife and I share our Catholic faith, and we strongly value higher education for our children.
\\
\vspace{0.15cm}
Even before I met my spouse, I knew that becoming fluent in Spanish would be helpful living in Whittier.  I already spoke a little, and in my first year joining the Whittier community I decided to formalize my Spanish skill by taking Spanish 120. Prof. Doreen O'Conner-G\'{o}mez was kind enough to let me audit her course.  She remarked that this was the first time she had seen a STEM professor audit a language course.  It turned out to be wonderfully useful, because our older generation usually does not speak English at home.  Our family as a whole is highly diverse, with Mexican, Romanian, American, and Filipino roots.  Given the divisive trends that have arisen within our broader culture, and as someone who tries to be a loving person, I felt the desire to share with you the story of our family for what it says about the true value of acceptance of those who are different.
\\
\vspace{0.15cm}
I recognize the same diversity in the families of my students.  Many of our students speak Spanish at home with their parents, but English at school.  There have been times when I have helped the mother or guardian of a student navigate campus by speaking Spanish, and it has made them more comfortable.  As part of a statistics course I taught for the Whittier Summer Session II (2020), I was gathering data from the Whittier College Factbook.  It reveals two important numbers about our students.  About seventy percent of our students are students of color, and about forty percent are first-generation.  My spouse and every single one of her siblings are all first-generation students.  Before the pandemic, sometimes colleagues would try to teach me about ``the first-generation experience,'' assuming that the white male physicist would not understand.  I always smile inwardly, since my entire family has shared this experience with me.
\\
\vspace{0.15cm}
I am keenly aware of the importance of our curricular theme of \textit{belonging.}  Despite the challenges brought by the pandemic, I have put effort into making that theme a reality.  I socialize on Zoom with my first-year advisees and research students in order to make them all feel that they belong.  I account for equity and inclusion in each decision I make.  One stark example was when a student in my section of INTD100 connected to class via Zoom \textit{while at work in CostCo.}  I learned to arrange my schedule to account for students' jobs, knowing that many were supporting themselves and loved ones.  Inflexibility would have made class accessible only for wealthy students.  Regarding belonging, I am often reminded of a basic fact: \textit{even though my heritage is different from my family, they have accepted me as one of their own.}  In the Gospels, we find the Golden Rule to treat others as we would like to be treated.  I am called therefore to ensure the students feel that they belong.

\subsection{Inspiration for New Courses}

Inspired by my family, and the theme of belonging, I have created two new courses that serve our current liberal arts curriculum.  One is entitled \textit{A History of Science in Latin America}.  There was a hunger for this course, and students needed to see that \textit{all} of our ancestors performed science.  We encounter a false historical contrast of \textit{central} and \textit{peripheral} scientific communities.  Taking STEM courses alone might give the impression that European and American cities have been central to scientific progress, and that Latin American communities have been \textit{peripheral}.  By honestly covering the colonial period in Latin America, we find examples in which Latin American communities were \textit{central}.  A more accurate description would be a full two-way exchange of knowledge (Sec. \ref{sec:teaching}).  I invited Sonia Chaidez to introduce my students to digital storytelling.  The students created final projects that wove together their cultural heritage, history, mathematics, and scientific discovery.
\\
\vspace{0.15cm}
The second liberal arts course I have created was inspired by the theme of belonging, and my research.  It is called \textit{Safe Return Doubtful: History and Current Status of Modern Science in Antarctica.}  At first glance, the connection between themes like inclusion and belonging and Antarctica is not obvious.  This course is a metaphor for self-exploration.  We address three main areas, interwoven throughout the semester.  First, we address the history of the race to discover the South Pole in 1910-11.  Second, we cover current science in Antarctica.  Third, we perform journal activities that invite the students to look inside themselves and to discover their potential for exploration.  The connection to inclusion and belonging emerges as we learn that the winner of the race for the South Pole was a person who took indigenous science seriously.  This was the same captain who completed the Northwest Passage, where he spent time with indigenous Canadians.  I share the rest of the story in Sec. \ref{sec:teaching}.  The course therefore connects inclusion and belonging to survival and exploration.

\subsection{Keeping a Sense of Humor under Quarantine}

My fellow tenure-track colleagues and I sometimes discuss if it's appropriate to add a ``COVID-19 Impact Statement'' to our reports.  At first I thought, no, just stick to the formal stuff.  Keep it short.  But I also thought it would be a laugh.  So here goes.  For those with a sense of humor, this next section is for you.  
\\
\vspace{0.15cm}
For those without a sense of humor, I have to ask, like, how are you still here?  After we duct-taped together a way to teach our students online in spring 2020, we watched the world lose its freaking mind that summer\footnote{After watching \textit{I Am Not Your Negro}, directed by Raoul Peck, and reading \textit{Notes of a Native Son} by James Baldwin, maybe it's more apt to say that we should have lost our minds sooner. I made a chapter of Notes of A Native Son the summer reading assignment for my College Writing Seminar.}.  Then, the module system, for \textit{a year.}  But hooray!  The vaccines arrived.  Funny thing about vaccines, though, is that you have to go \textit{get them.}  Ugh, who has time, right?  Seriously, a few people in my family flatly refuse.  Here's a fun exercise: try teaching a science course on Zoom and hearing your phone buzzing from an argument about how \textit{the scientists are wrong}.  Focus, focus ... \textit{just get the blasted shot already} ... ``Ok students, let's talk about ... friction!  Am I right?''  Ay yay yay.
\\
\vspace{0.15cm}
Teaching students remotely was like watching those YouTube channels where people crash into stuff.  What I mean is, students would log in to class while driving.  Had to make a rule against that.  Everyone survived, but ... wow.  I thought one of my students was driving on the wrong side of the road while Zooming, but it's ok, he was just in India.  Another rule I had to make for class: you can't be naked.  This is a family establishment. Please put a shirt on.  I don't wanna see that.  After seconds of research, I found the Zoom button to have the camera off by default.
\\
\vspace{0.15cm}
My spouse always says that she has the worst luck.  I always reassure her: ``Have faith honey.  We'll be alright.''  After years of searching, we find each other, marry, and then BAM.  Pandemic.  Whoops.  Our daughter was born right in the exact middle (like, literally within the error of the mean) of the first wave of COVID-19.  Whyyyy.  The nurses weren't going to let me in the hospital.  For the birth of my child.  What.  Actually they weren't really that keen on letting my wife in either.  \textit{Just hang out in the parking garage}, they told my wife, \textit{who was in labor.}  After the required amount of suffering took place, they let us come inside to give birth.  The next time you hear someone complain about masks, just think of how close my baby was to being born in a RAV4.
\\
\vspace{0.15cm}
Sometimes I had to fight for my students.  We made sure to purchase enough bandwidth for our house from Spectrum, but sometimes it still felt choppy.  After running some checks, I realized I needed to call Spectrum.  They told me, ``What do you expect?  It's slow all over the city.''  I got all idealistic: ``This is about access to education!  For first-generation students no less!'' They finally sent a techician, who came to our door holding small metal piece.  ``Did you know there was a 3 dB attenuator in your coaxial line?''  I half-choked on my coffee.  I knew that the ONLY JOB of this component is to cut signal power by a factor of two.  Whyyyy.  The Education Blocking Device was removed, and the signal was greatly improved.
\\
\vspace{0.15cm}
Working from home during Summer 2020 did make it easier to care for our daughter.  We were quarantined, but I managed to convince the ONR that we could perform the research project remotely.  My spouse is a dentist, and the state provided maternity leave.  For added spice, they took it away, though, after 12 weeks.  Right at the beginning of Fall semester.  No vaccines were available yet, so I had to just teach and parent alone.  The upside is that I got to spend more time with the baby.  My spouse bravely went back to work to help pay her student loans.  She treats patients who are supposed to test negative for COVID-19, but the positives sneak past the guard.  So basically The Hunger Games for dentists, who tend to be around a lot of, you know, mouths and throats.
\\
\vspace{0.15cm}
Here's another fun exercise: try teaching college-level physics with a six-month-old pooping in your lap.  Keep composure.  Another one: the baby is napping in her seat, and all is quiet and ready for class.  My chihuahua looks at me like he wants to bark at the dogs outside.  \textit{Don't you do it, Lobo!  I swear...} Does it anyways.  Baby wakes up, class paused.  So I trained him not to bark, but my neighbors responded by buying a rooster.  They. Bought. A. Rooster.  In the city.  Not chickens!  Chickens I would understand for the eggs.  We love eggs.  Fun fact about cities: they sell alarm clocks.  Quirky thing about roosters: they don't have a snooze.  I dreamed often of turning that blasted rooster into tacos.
\\
\vspace{0.15cm}
Despite the quirks of life in East Los Angeles, we take great pride in maintaining the community.  Except on 4th of July.  Then we blow it to smithereens.  As far as I can tell, the goal is to trigger as many car alarms as you can.  On our block the record is twelve.  I mean, why spend your money on ``real'' fireworks when you can just fill a rice-cooker with black powder?  It's easy.  Nothing wakes you up from grading math homework from summer session like shrapnel.
\\
\vspace{0.15cm}
Joking aside, my students were wonderfully understanding when I had to teach with the baby on my lap.  The same is true for my colleagues in committee meetings.  I like to think she brightened people's day a little.  A colleague from another institution who gave birth recently lamented to me that it has been \textit{so hard} lately, for she and her husband (both professors working from home) hadn't had child care in \textit{six whole weeks!}  I died a little inside.  It had been almost nine months for me flying solo.  Once we all got vaccinated, my \textit{suegra} (mother-in-law) who lives next door, started to come each day to help.  Que santa, no? (What a saint, no?).  My family has supported us, and we are so grateful.
\\
\vspace{0.15cm}
I hope you are all safe and sound.  It turns out that some people in my extended family in the Midwest were not so lucky.  My cousin's husband, who was a well-loved football coach and mentor to many junior college students, already needed a lung transplant before COVID-19 arrived.  He finally got the lung transplant a few years ago and recovered.  And then someone gave him the virus, and he's gone now, along with his father.  He had two beautiful daughters with my cousin.  We pray for them each night now.  If any of you have lost loved ones, we will pray for them as well.  I've had students miss class to go to funerals.  We have all suffered.  But we can still find hope, joy, and even laughter, knowing our community is still here.  Whittier College needs us.  The students have returned, and Whittier College will once again grow and thrive.

\end{document}
