\documentclass[../../../main.tex]{subfiles}

\begin{document}
\label{sec:first_year}

In the Fall of 2018, I was invited by Prof. Seamus Lagan to help with the first-year orientation.  The goal was to observe and help with the activities Prof. Lagan had planned for his new advisees.  We ran icebreaker activities, discussed majors and classes with the students, went over summer reading assignments, and toured campus.  Some of Prof. Lagan's first-year advisees later became my advisees as they graduated (see Sec. \ref{sec:advising_mentoring}).  It was an important experience, because it served as a trial run for the following year when I became the first-year advisor to my students in PHYS150.  That was my first solo experience in administering the orientation and first-semester advising process, and I'm grateful to my student coordinator and Whittier College staff for helping me through it.  In Fall 2020, I stepped forward to teach College Writing Seminar (INTD100), and take on more first year students as my own.
\\
\vspace{0.15cm}
This group of new first year advisees was added to my roster as I taught INTD100 for the first time.  I applied the lessons of 2018 and 2019 to my second solo run for first-year orientation.  My student coordinator and I did our best to give the students the sense that they \textit{belong} to this community.  One social activity that I find works well for the students is to game with them, as in online gaming and board games.  The majority of my new advisees in 2019 turned out to be athletes, so they were not free to do the field trips to downtown LA, for example.  I found times to meet with them (for both advising and social activities) that worked with their schedules.  In my first PEGP, I wrote about how I participated for more than a year very closely with the student organization CRU (a Christian fellowship for Whittier undergraduates).  Mentoring first-year advisees felt similar to that experience.
\\
\vspace{0.15cm}
The second mentoring experience in 2020 occurred during the height of the pandemic, before anyone was vaccinated.  We had to build a sense of community and belonging via Zoom.  I decided to serve the INTD100 program because it seemed in need of help and not enough instructors were stepping forward.  Though it was difficult building a sense of community with my students remotely, I kept encouraging them to perservere until the end of the semester.  Then, we gamed to our hearts content.  We played online card games, and Among Us, laughing the whole time.  Then we ``went home'' for the holiday break.  I will elaborate more on advising my students' course and major selections in Sec. \ref{sec:advising_mentoring}.

\end{document}
