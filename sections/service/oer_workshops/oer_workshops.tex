\documentclass[../../../main.tex]{subfiles}

\begin{document}
\label{sec:oer_workshops}

I was invited to give two lectures at OER workshops organized by Sonia Chaidez and Azeem Khan\footnote{See supplemental material.}.  I considered these a form of service to my colleagues, in that some of us might not be aware of how much course content in our field my be made open-access using OER.  One number that I keep in my head is that about 1 in 5 students have difficulty buying textbooks (see \ref{sec:equity_inclusion}).  My students use the OpenStax framework whenever possible, and this is true of courses in other STEM areas.  However, our OER workshops also covered the use of \url{https://www.oercommons.org/}, which provides open access resources in the humanities.  I also pointed out that there has been progress in the social sciences within the OpenStax framework, and how the OpenStax tools are used in my classes (see Sec. \ref{sec:methods}).
\\
\vspace{0.25cm}
The OpenStax Tutor system costs the students only \$10.00, and the OpenStax textbooks are free.  Tutor is a flexible reading and homework assignment system that grades student work, and uses artificial intelligence to provide students extra practice in areas where they need to grow.  The OpenStax texts help me teach my courses by providing example projects my students can construct as their final projects.  They also have built-in PhET simulations, to help the students illustrate concepts through experimentation.  I receive statistical reports from the Tutor system, and I act on them by covering those exercises in class with which many students struggled.  All reading and homework assignments are summarized in a calendar and notification system.  The system is an adaptable, feature-rich, and cost-effective for our students.
\\
\vspace{0.25cm}
In my OER lectures, I also gave examples of OER usage in advanced courses.  In Computer Logic and Digital Circuit Design (COSC330/PHYS306), the PYNQ-Z1 by Agilent (\url{https://www.pynq.io}) provides the students an open-source environment with which to learn software and firmware development.  I also gave an example of an open access text outside the OpenStax and OER Commons areas, for Digital Signal Processing (DSP), COSC390.  I write the course software +in octave, an open-source programming language the students install for free.  So DSP is an example of an advanced course that is completely open access, top to bottom.  It is my goal to inspire and lead my colleagues to use OER whenever practical, to help foster equity and inclusion in our educational environment.

\end{document}
