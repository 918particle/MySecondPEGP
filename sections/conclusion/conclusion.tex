\documentclass[../main.tex]{subfiles}
 
\begin{document}

This semester marks the beginning of my fifth year with Whittier College.  These years have been filled with both wonderfully uplifting experiences, and also wrenching heartbreaks.  I hope that my writing and analysis in Sec. \ref{sec:teaching} are convincing to the FPC that my teaching practice is growing and thriving.  Despite the rapid transition to online instruction, and the module system, my students and I have demonstrated success in learning.  The numerical and written evidence from course evaluations provide proof that my teaching works, while also providing valuable insight into \textit{why} certain practices work.  It is my prediction that the students will continue to favor projects and project-based learning over assessments like midterms and finals.  This is a trend I noticed throughout last year, and perhaps it will continue in the future.  I have taken seriously the FPC call to broaden my educational horizons by expanding my liberal arts portfolio.  I have created several courses outside my primary field that dealt with deep issues of the history of science, cross-culturally, and the students shared that they learned a great deal.  My participation in the college writing seminar (INTD100) was a challenging experience to which I will return periodically.
\\
\vspace{0.15cm}
Our research program continues to grow.  The beginning of our membership as an IceCube Collaboration institution will provide a new avenue of physics scholarship for our students.  We are excited to resume participation in the development of the world's largest neutrino detector in the quest for UHE-$\nu$.  Our brand new collaboration with the Office of Naval Research will help our students participate in engineering research with a broad range of applications.  Our students are eagerly signing up, and as of this writing, I am helping them to polish resum\'{e}s for application to ONR programs.  The ONR has given me an alternative route to performing quality research in the pandemic-induced pause experienced in the UHE-$\nu$ field.  There is also an interesting synergy between my physics and ONR engineering research: phased-array radar development.  The new research area of CEM will benefit our students' education, and our long-term research goals.
\\
\vspace{0.15cm}
I finally have the opportunity to share with FPC my accomplishments for Whittier College in the area of committee service.  Having served on ESAC, ERC/DLAC, and now EPC, I have received broad experience in how Whittier College operates.  With ESAC, we found a way to set admissions criteria that reflected our genuine desire to admit students we can support, and help to thrive.  Our proposal was ultimately adopted by the faculty after many negotiations and much data analysis.  I am proud of our our committee came together to address a difficult issue, and incorporated feedback from all sides.  With ERC, we found ways to help students archive their work so that their achievements are preserved long after they graduate.  I keep returning, however, to my experience with WSP.  It seems like a natural fit, and I hope to join the WSP Advisory board when it is practical.
\\
\vspace{0.15cm}
I have had wonderful experiences with all of my new advisees, both majors and non-majors.  I began advising before the pandemic, however, the experiences that stand out recently are my dogged attempts to create a sense of belonging for students that were blocked from coming to campus for the first time.  One of my brand new advisees is on the football team, and I finally can see him for the first time in my algebra-based physics class.  Another of mine is in Computer Logic and Digital Circuit Design.  I am helping her to apply for the ONR research program, and she joins a very excited cohort of computer science physics students who are now taking that course.

Jordan C. Hanson, PhD \\
Assistant Professor, Department of Physics and Astronomy \\
Science and Learning Center, 212 \\
Whittier College \\
562.907.5130 \\
jhanson2@whittier.edu

\end{document}
